\documentclass[]{article}
\usepackage{lmodern}
\usepackage{amssymb,amsmath}
\usepackage{ifxetex,ifluatex}
\usepackage{fixltx2e} % provides \textsubscript
\ifnum 0\ifxetex 1\fi\ifluatex 1\fi=0 % if pdftex
  \usepackage[T1]{fontenc}
  \usepackage[utf8]{inputenc}
\else % if luatex or xelatex
  \ifxetex
    \usepackage{mathspec}
  \else
    \usepackage{fontspec}
  \fi
  \defaultfontfeatures{Ligatures=TeX,Scale=MatchLowercase}
\fi
% use upquote if available, for straight quotes in verbatim environments
\IfFileExists{upquote.sty}{\usepackage{upquote}}{}
% use microtype if available
\IfFileExists{microtype.sty}{%
\usepackage[]{microtype}
\UseMicrotypeSet[protrusion]{basicmath} % disable protrusion for tt fonts
}{}
\PassOptionsToPackage{hyphens}{url} % url is loaded by hyperref
\usepackage[unicode=true]{hyperref}
\hypersetup{
            pdftitle={Proposed API for tech.ml.dataset},
            pdfauthor={GenerateMe},
            pdfborder={0 0 0},
            breaklinks=true}
\urlstyle{same}  % don't use monospace font for urls
\usepackage[margin=1in]{geometry}
\usepackage{color}
\usepackage{fancyvrb}
\newcommand{\VerbBar}{|}
\newcommand{\VERB}{\Verb[commandchars=\\\{\}]}
\DefineVerbatimEnvironment{Highlighting}{Verbatim}{commandchars=\\\{\}}
% Add ',fontsize=\small' for more characters per line
\usepackage{framed}
\definecolor{shadecolor}{RGB}{248,248,248}
\newenvironment{Shaded}{\begin{snugshade}}{\end{snugshade}}
\newcommand{\KeywordTok}[1]{\textcolor[rgb]{0.13,0.29,0.53}{\textbf{#1}}}
\newcommand{\DataTypeTok}[1]{\textcolor[rgb]{0.13,0.29,0.53}{#1}}
\newcommand{\DecValTok}[1]{\textcolor[rgb]{0.00,0.00,0.81}{#1}}
\newcommand{\BaseNTok}[1]{\textcolor[rgb]{0.00,0.00,0.81}{#1}}
\newcommand{\FloatTok}[1]{\textcolor[rgb]{0.00,0.00,0.81}{#1}}
\newcommand{\ConstantTok}[1]{\textcolor[rgb]{0.00,0.00,0.00}{#1}}
\newcommand{\CharTok}[1]{\textcolor[rgb]{0.31,0.60,0.02}{#1}}
\newcommand{\SpecialCharTok}[1]{\textcolor[rgb]{0.00,0.00,0.00}{#1}}
\newcommand{\StringTok}[1]{\textcolor[rgb]{0.31,0.60,0.02}{#1}}
\newcommand{\VerbatimStringTok}[1]{\textcolor[rgb]{0.31,0.60,0.02}{#1}}
\newcommand{\SpecialStringTok}[1]{\textcolor[rgb]{0.31,0.60,0.02}{#1}}
\newcommand{\ImportTok}[1]{#1}
\newcommand{\CommentTok}[1]{\textcolor[rgb]{0.56,0.35,0.01}{\textit{#1}}}
\newcommand{\DocumentationTok}[1]{\textcolor[rgb]{0.56,0.35,0.01}{\textbf{\textit{#1}}}}
\newcommand{\AnnotationTok}[1]{\textcolor[rgb]{0.56,0.35,0.01}{\textbf{\textit{#1}}}}
\newcommand{\CommentVarTok}[1]{\textcolor[rgb]{0.56,0.35,0.01}{\textbf{\textit{#1}}}}
\newcommand{\OtherTok}[1]{\textcolor[rgb]{0.56,0.35,0.01}{#1}}
\newcommand{\FunctionTok}[1]{\textcolor[rgb]{0.00,0.00,0.00}{#1}}
\newcommand{\VariableTok}[1]{\textcolor[rgb]{0.00,0.00,0.00}{#1}}
\newcommand{\ControlFlowTok}[1]{\textcolor[rgb]{0.13,0.29,0.53}{\textbf{#1}}}
\newcommand{\OperatorTok}[1]{\textcolor[rgb]{0.81,0.36,0.00}{\textbf{#1}}}
\newcommand{\BuiltInTok}[1]{#1}
\newcommand{\ExtensionTok}[1]{#1}
\newcommand{\PreprocessorTok}[1]{\textcolor[rgb]{0.56,0.35,0.01}{\textit{#1}}}
\newcommand{\AttributeTok}[1]{\textcolor[rgb]{0.77,0.63,0.00}{#1}}
\newcommand{\RegionMarkerTok}[1]{#1}
\newcommand{\InformationTok}[1]{\textcolor[rgb]{0.56,0.35,0.01}{\textbf{\textit{#1}}}}
\newcommand{\WarningTok}[1]{\textcolor[rgb]{0.56,0.35,0.01}{\textbf{\textit{#1}}}}
\newcommand{\AlertTok}[1]{\textcolor[rgb]{0.94,0.16,0.16}{#1}}
\newcommand{\ErrorTok}[1]{\textcolor[rgb]{0.64,0.00,0.00}{\textbf{#1}}}
\newcommand{\NormalTok}[1]{#1}
\usepackage{longtable,booktabs}
% Fix footnotes in tables (requires footnote package)
\IfFileExists{footnote.sty}{\usepackage{footnote}\makesavenoteenv{long table}}{}
\usepackage{graphicx,grffile}
\makeatletter
\def\maxwidth{\ifdim\Gin@nat@width>\linewidth\linewidth\else\Gin@nat@width\fi}
\def\maxheight{\ifdim\Gin@nat@height>\textheight\textheight\else\Gin@nat@height\fi}
\makeatother
% Scale images if necessary, so that they will not overflow the page
% margins by default, and it is still possible to overwrite the defaults
% using explicit options in \includegraphics[width, height, ...]{}
\setkeys{Gin}{width=\maxwidth,height=\maxheight,keepaspectratio}
\IfFileExists{parskip.sty}{%
\usepackage{parskip}
}{% else
\setlength{\parindent}{0pt}
\setlength{\parskip}{6pt plus 2pt minus 1pt}
}
\setlength{\emergencystretch}{3em}  % prevent overfull lines
\providecommand{\tightlist}{%
  \setlength{\itemsep}{0pt}\setlength{\parskip}{0pt}}
\setcounter{secnumdepth}{0}
% Redefines (sub)paragraphs to behave more like sections
\ifx\paragraph\undefined\else
\let\oldparagraph\paragraph
\renewcommand{\paragraph}[1]{\oldparagraph{#1}\mbox{}}
\fi
\ifx\subparagraph\undefined\else
\let\oldsubparagraph\subparagraph
\renewcommand{\subparagraph}[1]{\oldsubparagraph{#1}\mbox{}}
\fi

% set default figure placement to htbp
\makeatletter
\def\fps@figure{htbp}
\makeatother


\title{Proposed API for tech.ml.dataset}
\author{GenerateMe}
\date{2020-05-21}

\begin{document}
\maketitle

\subsection{Introduction}\label{introduction}

\href{https://github.com/techascent/tech.ml.dataset}{tech.ml.dataset} is
a great and fast library which brings columnar dataset to the Clojure.
Chris Nuernberger has been working on this library for last year as a
part of bigger \texttt{tech.ml} stack.

I've started to test the library and help to fix uncovered bugs. My main
goal was to compare functionalities with the other standards from other
platforms. I focused on R solutions:
\href{https://dplyr.tidyverse.org/}{dplyr},
\href{https://tidyr.tidyverse.org/}{tidyr} and
\href{https://rdatatable.gitlab.io/data.table/}{data.table}.

During conversions of the examples I've come up how to reorganized
existing \texttt{tech.ml.dataset} functions into simple to use API. The
main goals were:

\begin{itemize}
\tightlist
\item
  Focus on dataset manipulation functionality, leaving other parts of
  \texttt{tech.ml} like pipelines, datatypes, readers, ML, etc.
\item
  Single entry point for common operations - one function dispatching on
  given arguments.
\item
  \texttt{group-by} results with special kind of dataset - a dataset
  containing subsets created after grouping as a column.
\item
  Most operations recognize regular dataset and grouped dataset and
  process data accordingly.
\item
  One function form to enable thread-first on dataset.
\end{itemize}

All proposed functions are grouped in tabs below. Select group to see
examples and details.

INFO: The future of this API is not known yet. Two directions are
possible: integration into \texttt{tech.ml} or development under
\texttt{Scicloj} organization. For the time being use this repo if you
want to try. Join the discussion on
\href{https://clojurians.zulipchat.com/\#narrow/stream/236259-tech.2Eml.2Edataset.2Edev/topic/api}{Zulip}

Let's require main namespace and define dataset used in most examples:

\begin{Shaded}
\begin{Highlighting}[]
\NormalTok{(}\KeywordTok{require}\NormalTok{ '[techtest.api }\AttributeTok{:as}\NormalTok{ api])}
\NormalTok{(}\BuiltInTok{def}\FunctionTok{ DS }\NormalTok{(api/dataset \{}\AttributeTok{:V1}\NormalTok{ (}\KeywordTok{take} \DecValTok{9}\NormalTok{ (}\KeywordTok{cycle}\NormalTok{ [}\DecValTok{1} \DecValTok{2}\NormalTok{]))}
                      \AttributeTok{:V2}\NormalTok{ (}\KeywordTok{range} \DecValTok{1} \DecValTok{10}\NormalTok{)}
                      \AttributeTok{:V3}\NormalTok{ (}\KeywordTok{take} \DecValTok{9}\NormalTok{ (}\KeywordTok{cycle}\NormalTok{ [}\FloatTok{0.5} \FloatTok{1.0} \FloatTok{1.5}\NormalTok{]))}
                      \AttributeTok{:V4}\NormalTok{ (}\KeywordTok{take} \DecValTok{9}\NormalTok{ (}\KeywordTok{cycle}\NormalTok{ [}\CharTok{\textbackslash{}A} \CharTok{\textbackslash{}B} \CharTok{\textbackslash{}C}\NormalTok{]))\}))}
\end{Highlighting}
\end{Shaded}

\begin{Shaded}
\begin{Highlighting}[]
\NormalTok{DS}
\end{Highlighting}
\end{Shaded}

\_unnamed {[}9 4{]}:

\begin{longtable}[]{@{}llll@{}}
\toprule
:V1 & :V2 & :V3 & :V4\tabularnewline
\midrule
\endhead
1 & 1 & 0.5000 & A\tabularnewline
2 & 2 & 1.000 & B\tabularnewline
1 & 3 & 1.500 & C\tabularnewline
2 & 4 & 0.5000 & A\tabularnewline
1 & 5 & 1.000 & B\tabularnewline
2 & 6 & 1.500 & C\tabularnewline
1 & 7 & 0.5000 & A\tabularnewline
2 & 8 & 1.000 & B\tabularnewline
1 & 9 & 1.500 & C\tabularnewline
\bottomrule
\end{longtable}

\subsection{Functionality}\label{functionality}

\subsubsection{Dataset}\label{dataset}

Dataset is a special type which can be considered as a map of columns
implemented around \texttt{tech.ml.datatype} library. Each column can be
considered as named sequence of typed data. Supported types include
integers, floats, string, boolean, date/time, objects etc.

\paragraph{Dataset creation}\label{dataset-creation}

Dataset can be created from various of types of Clojure structures and
files:

\begin{itemize}
\tightlist
\item
  single values
\item
  sequence of maps
\item
  map of sequences or values
\item
  sequence of columns (taken from other dataset or created manually)
\item
  sequence of pairs
\item
  file types: raw/gzipped csv/tsv, json, xls(x) taken from local file
  system or URL
\item
  input stream
\end{itemize}

\texttt{api/dataset} accepts:

\begin{itemize}
\tightlist
\item
  data
\item
  options (see documentation of
  \texttt{tech.ml.dataset/-\textgreater{}dataset} function for full
  list):
\item
  \texttt{:dataset-name} - name of the dataset
\item
  \texttt{:num-rows} - number of rows to read from file
\item
  \texttt{:header-row?} - indication if first row in file is a header
\item
  \texttt{:key-fn} - function applied to column names (eg.
  \texttt{keyword}, to convert column names to keywords)
\item
  \texttt{:separator} - column separator
\item
  \texttt{:single-value-column-name} - name of the column when single
  value is provided
\end{itemize}

\begin{center}\rule{0.5\linewidth}{0.5pt}\end{center}

Empty dataset.

\begin{Shaded}
\begin{Highlighting}[]
\NormalTok{(api/dataset)}
\end{Highlighting}
\end{Shaded}

\begin{verbatim}
_unnamed [0 0]:
\end{verbatim}

\begin{center}\rule{0.5\linewidth}{0.5pt}\end{center}

Dataset from single value.

\begin{Shaded}
\begin{Highlighting}[]
\NormalTok{(api/dataset }\DecValTok{999}\NormalTok{)}
\end{Highlighting}
\end{Shaded}

\_unnamed {[}1 1{]}:

\begin{longtable}[]{@{}l@{}}
\toprule
:\$value\tabularnewline
\midrule
\endhead
999\tabularnewline
\bottomrule
\end{longtable}

\begin{center}\rule{0.5\linewidth}{0.5pt}\end{center}

Set column name for single value. Also set the dataset name.

\begin{Shaded}
\begin{Highlighting}[]
\NormalTok{(api/dataset }\DecValTok{999}\NormalTok{ \{}\AttributeTok{:single-value-column-name} \StringTok{"my-single-value"}\NormalTok{\})}
\NormalTok{(api/dataset }\DecValTok{999}\NormalTok{ \{}\AttributeTok{:single-value-column-name} \StringTok{""}
                  \AttributeTok{:dataset-name} \StringTok{"Single value"}\NormalTok{\})}
\end{Highlighting}
\end{Shaded}

\_unnamed {[}1 1{]}:

\begin{longtable}[]{@{}l@{}}
\toprule
my-single-value\tabularnewline
\midrule
\endhead
999\tabularnewline
\bottomrule
\end{longtable}

Single value {[}1 1{]}:

\begin{longtable}[]{@{}l@{}}
\toprule
0\tabularnewline
\midrule
\endhead
999\tabularnewline
\bottomrule
\end{longtable}

\begin{center}\rule{0.5\linewidth}{0.5pt}\end{center}

Sequence of pairs (first = column name, second = value(s)).

\begin{Shaded}
\begin{Highlighting}[]
\NormalTok{(api/dataset [[}\AttributeTok{:A} \DecValTok{33}\NormalTok{] [}\AttributeTok{:B} \DecValTok{5}\NormalTok{] [}\AttributeTok{:C} \AttributeTok{:a}\NormalTok{]])}
\end{Highlighting}
\end{Shaded}

\_unnamed {[}1 3{]}:

\begin{longtable}[]{@{}lll@{}}
\toprule
:A & :B & :C\tabularnewline
\midrule
\endhead
33 & 5 & :a\tabularnewline
\bottomrule
\end{longtable}

\begin{center}\rule{0.5\linewidth}{0.5pt}\end{center}

Not sequential values are repeated row-count number of times.

\begin{Shaded}
\begin{Highlighting}[]
\NormalTok{(api/dataset [[}\AttributeTok{:A}\NormalTok{ [}\DecValTok{1} \DecValTok{2} \DecValTok{3} \DecValTok{4} \DecValTok{5} \DecValTok{6}\NormalTok{]] [}\AttributeTok{:B} \StringTok{"X"}\NormalTok{] [}\AttributeTok{:C} \AttributeTok{:a}\NormalTok{]])}
\end{Highlighting}
\end{Shaded}

\_unnamed {[}6 3{]}:

\begin{longtable}[]{@{}lll@{}}
\toprule
:A & :B & :C\tabularnewline
\midrule
\endhead
1 & X & :a\tabularnewline
2 & X & :a\tabularnewline
3 & X & :a\tabularnewline
4 & X & :a\tabularnewline
5 & X & :a\tabularnewline
6 & X & :a\tabularnewline
\bottomrule
\end{longtable}

\begin{center}\rule{0.5\linewidth}{0.5pt}\end{center}

Dataset created from map (keys = column name, second = value(s)). Works
the same as sequence of pairs.

\begin{Shaded}
\begin{Highlighting}[]
\NormalTok{(api/dataset \{}\AttributeTok{:A} \DecValTok{33}\NormalTok{\})}
\NormalTok{(api/dataset \{}\AttributeTok{:A}\NormalTok{ [}\DecValTok{1} \DecValTok{2} \DecValTok{3}\NormalTok{]\})}
\NormalTok{(api/dataset \{}\AttributeTok{:A}\NormalTok{ [}\DecValTok{3} \DecValTok{4} \DecValTok{5}\NormalTok{] }\AttributeTok{:B} \StringTok{"X"}\NormalTok{\})}
\end{Highlighting}
\end{Shaded}

\_unnamed {[}1 1{]}:

\begin{longtable}[]{@{}l@{}}
\toprule
:A\tabularnewline
\midrule
\endhead
33\tabularnewline
\bottomrule
\end{longtable}

\_unnamed {[}3 1{]}:

\begin{longtable}[]{@{}l@{}}
\toprule
:A\tabularnewline
\midrule
\endhead
1\tabularnewline
2\tabularnewline
3\tabularnewline
\bottomrule
\end{longtable}

\_unnamed {[}3 2{]}:

\begin{longtable}[]{@{}ll@{}}
\toprule
:A & :B\tabularnewline
\midrule
\endhead
3 & X\tabularnewline
4 & X\tabularnewline
5 & X\tabularnewline
\bottomrule
\end{longtable}

\begin{center}\rule{0.5\linewidth}{0.5pt}\end{center}

You can put any value inside a column

\begin{Shaded}
\begin{Highlighting}[]
\NormalTok{(api/dataset \{}\AttributeTok{:A}\NormalTok{ [[}\DecValTok{3} \DecValTok{4} \DecValTok{5}\NormalTok{] [}\AttributeTok{:a} \AttributeTok{:b}\NormalTok{]] }\AttributeTok{:B} \StringTok{"X"}\NormalTok{\})}
\end{Highlighting}
\end{Shaded}

\_unnamed {[}2 2{]}:

\begin{longtable}[]{@{}ll@{}}
\toprule
:A & :B\tabularnewline
\midrule
\endhead
{[}3 4 5{]} & X\tabularnewline
{[}:a :b{]} & X\tabularnewline
\bottomrule
\end{longtable}

\begin{center}\rule{0.5\linewidth}{0.5pt}\end{center}

Sequence of maps

\begin{Shaded}
\begin{Highlighting}[]
\NormalTok{(api/dataset [\{}\AttributeTok{:a} \DecValTok{1} \AttributeTok{:b} \DecValTok{3}\NormalTok{\} \{}\AttributeTok{:b} \DecValTok{2} \AttributeTok{:a} \DecValTok{99}\NormalTok{\}])}
\NormalTok{(api/dataset [\{}\AttributeTok{:a} \DecValTok{1} \AttributeTok{:b}\NormalTok{ [}\DecValTok{1} \DecValTok{2} \DecValTok{3}\NormalTok{]\} \{}\AttributeTok{:a} \DecValTok{2} \AttributeTok{:b}\NormalTok{ [}\DecValTok{3} \DecValTok{4}\NormalTok{]\}])}
\end{Highlighting}
\end{Shaded}

\_unnamed {[}2 2{]}:

\begin{longtable}[]{@{}ll@{}}
\toprule
:a & :b\tabularnewline
\midrule
\endhead
1 & 3\tabularnewline
99 & 2\tabularnewline
\bottomrule
\end{longtable}

\_unnamed {[}2 2{]}:

\begin{longtable}[]{@{}ll@{}}
\toprule
:a & :b\tabularnewline
\midrule
\endhead
1 & {[}1 2 3{]}\tabularnewline
2 & {[}3 4{]}\tabularnewline
\bottomrule
\end{longtable}

\begin{center}\rule{0.5\linewidth}{0.5pt}\end{center}

Missing values are marked by \texttt{nil}

\begin{Shaded}
\begin{Highlighting}[]
\NormalTok{(api/dataset [\{}\AttributeTok{:a} \VariableTok{nil} \AttributeTok{:b} \DecValTok{1}\NormalTok{\} \{}\AttributeTok{:a} \DecValTok{3} \AttributeTok{:b} \DecValTok{4}\NormalTok{\} \{}\AttributeTok{:a} \DecValTok{11}\NormalTok{\}])}
\end{Highlighting}
\end{Shaded}

\_unnamed {[}3 2{]}:

\begin{longtable}[]{@{}ll@{}}
\toprule
:a & :b\tabularnewline
\midrule
\endhead
& 1\tabularnewline
3 & 4\tabularnewline
11 &\tabularnewline
\bottomrule
\end{longtable}

\begin{center}\rule{0.5\linewidth}{0.5pt}\end{center}

Import CSV file

\begin{Shaded}
\begin{Highlighting}[]
\NormalTok{(api/dataset }\StringTok{"data/family.csv"}\NormalTok{)}
\end{Highlighting}
\end{Shaded}

data/family.csv {[}5 5{]}:

\begin{longtable}[]{@{}lllll@{}}
\toprule
family & dob\_child1 & dob\_child2 & gender\_child1 &
gender\_child2\tabularnewline
\midrule
\endhead
1 & 1998-11-26 & 2000-01-29 & 1 & 2\tabularnewline
2 & 1996-06-22 & & 2 &\tabularnewline
3 & 2002-07-11 & 2004-04-05 & 2 & 2\tabularnewline
4 & 2004-10-10 & 2009-08-27 & 1 & 1\tabularnewline
5 & 2000-12-05 & 2005-02-28 & 2 & 1\tabularnewline
\bottomrule
\end{longtable}

\begin{center}\rule{0.5\linewidth}{0.5pt}\end{center}

Import from URL

\begin{Shaded}
\begin{Highlighting}[]
\NormalTok{(}\BuiltInTok{defonce}\FunctionTok{ ds }\NormalTok{(api/dataset }\StringTok{"https://vega.github.io/vega-lite/examples/data/seattle-weather.csv"}\NormalTok{))}
\end{Highlighting}
\end{Shaded}

\begin{Shaded}
\begin{Highlighting}[]
\NormalTok{ds}
\end{Highlighting}
\end{Shaded}

\url{https://vega.github.io/vega-lite/examples/data/seattle-weather.csv}
{[}1461 6{]}:

\begin{longtable}[]{@{}llllll@{}}
\toprule
date & precipitation & temp\_max & temp\_min & wind &
weather\tabularnewline
\midrule
\endhead
2012-01-01 & 0.000 & 12.80 & 5.000 & 4.700 & drizzle\tabularnewline
2012-01-02 & 10.90 & 10.60 & 2.800 & 4.500 & rain\tabularnewline
2012-01-03 & 0.8000 & 11.70 & 7.200 & 2.300 & rain\tabularnewline
2012-01-04 & 20.30 & 12.20 & 5.600 & 4.700 & rain\tabularnewline
2012-01-05 & 1.300 & 8.900 & 2.800 & 6.100 & rain\tabularnewline
2012-01-06 & 2.500 & 4.400 & 2.200 & 2.200 & rain\tabularnewline
2012-01-07 & 0.000 & 7.200 & 2.800 & 2.300 & rain\tabularnewline
2012-01-08 & 0.000 & 10.00 & 2.800 & 2.000 & sun\tabularnewline
2012-01-09 & 4.300 & 9.400 & 5.000 & 3.400 & rain\tabularnewline
2012-01-10 & 1.000 & 6.100 & 0.6000 & 3.400 & rain\tabularnewline
2012-01-11 & 0.000 & 6.100 & -1.100 & 5.100 & sun\tabularnewline
2012-01-12 & 0.000 & 6.100 & -1.700 & 1.900 & sun\tabularnewline
2012-01-13 & 0.000 & 5.000 & -2.800 & 1.300 & sun\tabularnewline
2012-01-14 & 4.100 & 4.400 & 0.6000 & 5.300 & snow\tabularnewline
2012-01-15 & 5.300 & 1.100 & -3.300 & 3.200 & snow\tabularnewline
2012-01-16 & 2.500 & 1.700 & -2.800 & 5.000 & snow\tabularnewline
2012-01-17 & 8.100 & 3.300 & 0.000 & 5.600 & snow\tabularnewline
2012-01-18 & 19.80 & 0.000 & -2.800 & 5.000 & snow\tabularnewline
2012-01-19 & 15.20 & -1.100 & -2.800 & 1.600 & snow\tabularnewline
2012-01-20 & 13.50 & 7.200 & -1.100 & 2.300 & snow\tabularnewline
2012-01-21 & 3.000 & 8.300 & 3.300 & 8.200 & rain\tabularnewline
2012-01-22 & 6.100 & 6.700 & 2.200 & 4.800 & rain\tabularnewline
2012-01-23 & 0.000 & 8.300 & 1.100 & 3.600 & rain\tabularnewline
2012-01-24 & 8.600 & 10.00 & 2.200 & 5.100 & rain\tabularnewline
2012-01-25 & 8.100 & 8.900 & 4.400 & 5.400 & rain\tabularnewline
\bottomrule
\end{longtable}

\paragraph{Saving}\label{saving}

Export dataset to a file or output stream can be done by calling
\texttt{api/write-csv!}. Function accepts:

\begin{itemize}
\tightlist
\item
  dataset
\item
  file name with one of the extensions: \texttt{.csv}, \texttt{.tsv},
  \texttt{.csv.gz} and \texttt{.tsv.gz} or output stream
\item
  options:
\item
  \texttt{:separator} - string or separator char.
\end{itemize}

\begin{Shaded}
\begin{Highlighting}[]
\NormalTok{(api/write-csv! ds }\StringTok{"output.tsv.gz"}\NormalTok{)}
\NormalTok{(.exists (clojure.java.io/file }\StringTok{"output.csv.gz"}\NormalTok{))}
\end{Highlighting}
\end{Shaded}

\begin{verbatim}
nil
true
\end{verbatim}

\paragraph{Dataset related functions}\label{dataset-related-functions}

Summary functions about the dataset like number of rows, columns and
basic stats.

\begin{center}\rule{0.5\linewidth}{0.5pt}\end{center}

Number of rows

\begin{Shaded}
\begin{Highlighting}[]
\NormalTok{(api/row-count ds)}
\end{Highlighting}
\end{Shaded}

\begin{verbatim}
1461
\end{verbatim}

\begin{center}\rule{0.5\linewidth}{0.5pt}\end{center}

Number of columns

\begin{Shaded}
\begin{Highlighting}[]
\NormalTok{(api/column-count ds)}
\end{Highlighting}
\end{Shaded}

\begin{verbatim}
6
\end{verbatim}

\begin{center}\rule{0.5\linewidth}{0.5pt}\end{center}

Names of columns.

\begin{Shaded}
\begin{Highlighting}[]
\NormalTok{(api/column-names ds)}
\end{Highlighting}
\end{Shaded}

\begin{verbatim}
("date" "precipitation" "temp_max" "temp_min" "wind" "weather")
\end{verbatim}

\begin{center}\rule{0.5\linewidth}{0.5pt}\end{center}

Shape of the dataset, {[}row count, column count{]}

\begin{Shaded}
\begin{Highlighting}[]
\NormalTok{(api/shape ds)}
\end{Highlighting}
\end{Shaded}

\begin{verbatim}
[1461 6]
\end{verbatim}

\begin{center}\rule{0.5\linewidth}{0.5pt}\end{center}

General info about dataset. There are three variants:

\begin{itemize}
\tightlist
\item
  default - containing information about columns with basic statistics
\item
  \texttt{:basic} - just name, row and column count and information if
  dataset is a result of \texttt{group-by} operation
\item
  \texttt{:columns} - columns' metadata
\end{itemize}

\begin{Shaded}
\begin{Highlighting}[]
\NormalTok{(api/info ds)}
\NormalTok{(api/info ds }\AttributeTok{:basic}\NormalTok{)}
\NormalTok{(api/info ds }\AttributeTok{:columns}\NormalTok{)}
\end{Highlighting}
\end{Shaded}

\url{https://vega.github.io/vega-lite/examples/data/seattle-weather.csv}:
descriptive-stats {[}6 10{]}:

\begin{longtable}[]{@{}llllllllll@{}}
\toprule
\begin{minipage}[b]{0.08\columnwidth}\raggedright\strut
:col-name\strut
\end{minipage} & \begin{minipage}[b]{0.11\columnwidth}\raggedright\strut
:datatype\strut
\end{minipage} & \begin{minipage}[b]{0.06\columnwidth}\raggedright\strut
:n-valid\strut
\end{minipage} & \begin{minipage}[b]{0.07\columnwidth}\raggedright\strut
:n-missing\strut
\end{minipage} & \begin{minipage}[b]{0.07\columnwidth}\raggedright\strut
:mean\strut
\end{minipage} & \begin{minipage}[b]{0.04\columnwidth}\raggedright\strut
:mode\strut
\end{minipage} & \begin{minipage}[b]{0.07\columnwidth}\raggedright\strut
:min\strut
\end{minipage} & \begin{minipage}[b]{0.07\columnwidth}\raggedright\strut
:max\strut
\end{minipage} & \begin{minipage}[b]{0.12\columnwidth}\raggedright\strut
:standard-deviation\strut
\end{minipage} & \begin{minipage}[b]{0.05\columnwidth}\raggedright\strut
:skew\strut
\end{minipage}\tabularnewline
\midrule
\endhead
\begin{minipage}[t]{0.08\columnwidth}\raggedright\strut
date\strut
\end{minipage} & \begin{minipage}[t]{0.11\columnwidth}\raggedright\strut
:packed-local-date\strut
\end{minipage} & \begin{minipage}[t]{0.06\columnwidth}\raggedright\strut
1461\strut
\end{minipage} & \begin{minipage}[t]{0.07\columnwidth}\raggedright\strut
0\strut
\end{minipage} & \begin{minipage}[t]{0.07\columnwidth}\raggedright\strut
2013-12-31\strut
\end{minipage} & \begin{minipage}[t]{0.04\columnwidth}\raggedright\strut
\strut
\end{minipage} & \begin{minipage}[t]{0.07\columnwidth}\raggedright\strut
2012-01-01\strut
\end{minipage} & \begin{minipage}[t]{0.07\columnwidth}\raggedright\strut
2015-12-31\strut
\end{minipage} & \begin{minipage}[t]{0.12\columnwidth}\raggedright\strut
\strut
\end{minipage} & \begin{minipage}[t]{0.05\columnwidth}\raggedright\strut
\strut
\end{minipage}\tabularnewline
\begin{minipage}[t]{0.08\columnwidth}\raggedright\strut
precipitation\strut
\end{minipage} & \begin{minipage}[t]{0.11\columnwidth}\raggedright\strut
:float32\strut
\end{minipage} & \begin{minipage}[t]{0.06\columnwidth}\raggedright\strut
1461\strut
\end{minipage} & \begin{minipage}[t]{0.07\columnwidth}\raggedright\strut
0\strut
\end{minipage} & \begin{minipage}[t]{0.07\columnwidth}\raggedright\strut
3.029\strut
\end{minipage} & \begin{minipage}[t]{0.04\columnwidth}\raggedright\strut
\strut
\end{minipage} & \begin{minipage}[t]{0.07\columnwidth}\raggedright\strut
0.000\strut
\end{minipage} & \begin{minipage}[t]{0.07\columnwidth}\raggedright\strut
55.90\strut
\end{minipage} & \begin{minipage}[t]{0.12\columnwidth}\raggedright\strut
6.680\strut
\end{minipage} & \begin{minipage}[t]{0.05\columnwidth}\raggedright\strut
3.506\strut
\end{minipage}\tabularnewline
\begin{minipage}[t]{0.08\columnwidth}\raggedright\strut
temp\_max\strut
\end{minipage} & \begin{minipage}[t]{0.11\columnwidth}\raggedright\strut
:float32\strut
\end{minipage} & \begin{minipage}[t]{0.06\columnwidth}\raggedright\strut
1461\strut
\end{minipage} & \begin{minipage}[t]{0.07\columnwidth}\raggedright\strut
0\strut
\end{minipage} & \begin{minipage}[t]{0.07\columnwidth}\raggedright\strut
16.44\strut
\end{minipage} & \begin{minipage}[t]{0.04\columnwidth}\raggedright\strut
\strut
\end{minipage} & \begin{minipage}[t]{0.07\columnwidth}\raggedright\strut
-1.600\strut
\end{minipage} & \begin{minipage}[t]{0.07\columnwidth}\raggedright\strut
35.60\strut
\end{minipage} & \begin{minipage}[t]{0.12\columnwidth}\raggedright\strut
7.350\strut
\end{minipage} & \begin{minipage}[t]{0.05\columnwidth}\raggedright\strut
0.2809\strut
\end{minipage}\tabularnewline
\begin{minipage}[t]{0.08\columnwidth}\raggedright\strut
temp\_min\strut
\end{minipage} & \begin{minipage}[t]{0.11\columnwidth}\raggedright\strut
:float32\strut
\end{minipage} & \begin{minipage}[t]{0.06\columnwidth}\raggedright\strut
1461\strut
\end{minipage} & \begin{minipage}[t]{0.07\columnwidth}\raggedright\strut
0\strut
\end{minipage} & \begin{minipage}[t]{0.07\columnwidth}\raggedright\strut
8.235\strut
\end{minipage} & \begin{minipage}[t]{0.04\columnwidth}\raggedright\strut
\strut
\end{minipage} & \begin{minipage}[t]{0.07\columnwidth}\raggedright\strut
-7.100\strut
\end{minipage} & \begin{minipage}[t]{0.07\columnwidth}\raggedright\strut
18.30\strut
\end{minipage} & \begin{minipage}[t]{0.12\columnwidth}\raggedright\strut
5.023\strut
\end{minipage} & \begin{minipage}[t]{0.05\columnwidth}\raggedright\strut
-0.2495\strut
\end{minipage}\tabularnewline
\begin{minipage}[t]{0.08\columnwidth}\raggedright\strut
weather\strut
\end{minipage} & \begin{minipage}[t]{0.11\columnwidth}\raggedright\strut
:string\strut
\end{minipage} & \begin{minipage}[t]{0.06\columnwidth}\raggedright\strut
1461\strut
\end{minipage} & \begin{minipage}[t]{0.07\columnwidth}\raggedright\strut
0\strut
\end{minipage} & \begin{minipage}[t]{0.07\columnwidth}\raggedright\strut
\strut
\end{minipage} & \begin{minipage}[t]{0.04\columnwidth}\raggedright\strut
sun\strut
\end{minipage} & \begin{minipage}[t]{0.07\columnwidth}\raggedright\strut
\strut
\end{minipage} & \begin{minipage}[t]{0.07\columnwidth}\raggedright\strut
\strut
\end{minipage} & \begin{minipage}[t]{0.12\columnwidth}\raggedright\strut
\strut
\end{minipage} & \begin{minipage}[t]{0.05\columnwidth}\raggedright\strut
\strut
\end{minipage}\tabularnewline
\begin{minipage}[t]{0.08\columnwidth}\raggedright\strut
wind\strut
\end{minipage} & \begin{minipage}[t]{0.11\columnwidth}\raggedright\strut
:float32\strut
\end{minipage} & \begin{minipage}[t]{0.06\columnwidth}\raggedright\strut
1461\strut
\end{minipage} & \begin{minipage}[t]{0.07\columnwidth}\raggedright\strut
0\strut
\end{minipage} & \begin{minipage}[t]{0.07\columnwidth}\raggedright\strut
3.241\strut
\end{minipage} & \begin{minipage}[t]{0.04\columnwidth}\raggedright\strut
\strut
\end{minipage} & \begin{minipage}[t]{0.07\columnwidth}\raggedright\strut
0.4000\strut
\end{minipage} & \begin{minipage}[t]{0.07\columnwidth}\raggedright\strut
9.500\strut
\end{minipage} & \begin{minipage}[t]{0.12\columnwidth}\raggedright\strut
1.438\strut
\end{minipage} & \begin{minipage}[t]{0.05\columnwidth}\raggedright\strut
0.8917\strut
\end{minipage}\tabularnewline
\bottomrule
\end{longtable}

\url{https://vega.github.io/vega-lite/examples/data/seattle-weather.csv}
:basic info {[}1 4{]}:

\begin{longtable}[]{@{}llll@{}}
\toprule
\begin{minipage}[b]{0.61\columnwidth}\raggedright\strut
:name\strut
\end{minipage} & \begin{minipage}[b]{0.11\columnwidth}\raggedright\strut
:grouped?\strut
\end{minipage} & \begin{minipage}[b]{0.07\columnwidth}\raggedright\strut
:rows\strut
\end{minipage} & \begin{minipage}[b]{0.10\columnwidth}\raggedright\strut
:columns\strut
\end{minipage}\tabularnewline
\midrule
\endhead
\begin{minipage}[t]{0.61\columnwidth}\raggedright\strut
\url{https://vega.github.io/vega-lite/examples/data/seattle-weather.csv}\strut
\end{minipage} & \begin{minipage}[t]{0.11\columnwidth}\raggedright\strut
false\strut
\end{minipage} & \begin{minipage}[t]{0.07\columnwidth}\raggedright\strut
1461\strut
\end{minipage} & \begin{minipage}[t]{0.10\columnwidth}\raggedright\strut
6\strut
\end{minipage}\tabularnewline
\bottomrule
\end{longtable}

\url{https://vega.github.io/vega-lite/examples/data/seattle-weather.csv}
:column info {[}6 4{]}:

\begin{longtable}[]{@{}llll@{}}
\toprule
:name & :size & :datatype & :categorical?\tabularnewline
\midrule
\endhead
date & 1461 & :packed-local-date &\tabularnewline
precipitation & 1461 & :float32 &\tabularnewline
temp\_max & 1461 & :float32 &\tabularnewline
temp\_min & 1461 & :float32 &\tabularnewline
wind & 1461 & :float32 &\tabularnewline
weather & 1461 & :string & true\tabularnewline
\bottomrule
\end{longtable}

\begin{center}\rule{0.5\linewidth}{0.5pt}\end{center}

Getting a dataset name

\begin{Shaded}
\begin{Highlighting}[]
\NormalTok{(api/dataset-name ds)}
\end{Highlighting}
\end{Shaded}

\begin{verbatim}
"https://vega.github.io/vega-lite/examples/data/seattle-weather.csv"
\end{verbatim}

\begin{center}\rule{0.5\linewidth}{0.5pt}\end{center}

Setting a dataset name (operation is immutable).

\begin{Shaded}
\begin{Highlighting}[]
\NormalTok{(}\KeywordTok{->>} \StringTok{"seattle-weather"}
\NormalTok{     (api/set-dataset-name ds)}
\NormalTok{     (api/dataset-name))}
\end{Highlighting}
\end{Shaded}

\begin{verbatim}
"seattle-weather"
\end{verbatim}

\paragraph{Columns and rows}\label{columns-and-rows}

Get columns and rows as sequences. \texttt{column}, \texttt{columns} and
\texttt{rows} treat grouped dataset as regular one. See \texttt{Groups}
to read more about grouped datasets.

\begin{center}\rule{0.5\linewidth}{0.5pt}\end{center}

Select column.

\begin{Shaded}
\begin{Highlighting}[]
\NormalTok{(ds }\StringTok{"wind"}\NormalTok{)}
\NormalTok{(api/column ds }\StringTok{"date"}\NormalTok{)}
\end{Highlighting}
\end{Shaded}

\begin{verbatim}
#tech.ml.dataset.column<float32>[1461]
wind
[4.700, 4.500, 2.300, 4.700, 6.100, 2.200, 2.300, 2.000, 3.400, 3.400, 5.100, 1.900, 1.300, 5.300, 3.200, 5.000, 5.600, 5.000, 1.600, 2.300, ...]
#tech.ml.dataset.column<packed-local-date>[1461]
date
[2012-01-01, 2012-01-02, 2012-01-03, 2012-01-04, 2012-01-05, 2012-01-06, 2012-01-07, 2012-01-08, 2012-01-09, 2012-01-10, 2012-01-11, 2012-01-12, 2012-01-13, 2012-01-14, 2012-01-15, 2012-01-16, 2012-01-17, 2012-01-18, 2012-01-19, 2012-01-20, ...]
\end{verbatim}

\begin{center}\rule{0.5\linewidth}{0.5pt}\end{center}

Columns as sequence

\begin{Shaded}
\begin{Highlighting}[]
\NormalTok{(}\KeywordTok{take} \DecValTok{2}\NormalTok{ (api/columns ds))}
\end{Highlighting}
\end{Shaded}

\begin{verbatim}
(#tech.ml.dataset.column<packed-local-date>[1461]
date
[2012-01-01, 2012-01-02, 2012-01-03, 2012-01-04, 2012-01-05, 2012-01-06, 2012-01-07, 2012-01-08, 2012-01-09, 2012-01-10, 2012-01-11, 2012-01-12, 2012-01-13, 2012-01-14, 2012-01-15, 2012-01-16, 2012-01-17, 2012-01-18, 2012-01-19, 2012-01-20, ...] #tech.ml.dataset.column<float32>[1461]
precipitation
[0.000, 10.90, 0.8000, 20.30, 1.300, 2.500, 0.000, 0.000, 4.300, 1.000, 0.000, 0.000, 0.000, 4.100, 5.300, 2.500, 8.100, 19.80, 15.20, 13.50, ...])
\end{verbatim}

\begin{center}\rule{0.5\linewidth}{0.5pt}\end{center}

Columns as map

\begin{Shaded}
\begin{Highlighting}[]
\NormalTok{(}\KeywordTok{keys}\NormalTok{ (api/columns ds }\AttributeTok{:as-map}\NormalTok{))}
\end{Highlighting}
\end{Shaded}

\begin{verbatim}
("date" "precipitation" "temp_max" "temp_min" "wind" "weather")
\end{verbatim}

\begin{center}\rule{0.5\linewidth}{0.5pt}\end{center}

Rows as sequence of sequences

\begin{Shaded}
\begin{Highlighting}[]
\NormalTok{(}\KeywordTok{take} \DecValTok{2}\NormalTok{ (api/rows ds))}
\end{Highlighting}
\end{Shaded}

\begin{verbatim}
([#object[java.time.LocalDate 0x70ca466e "2012-01-01"] 0.0 12.8 5.0 4.7 "drizzle"] [#object[java.time.LocalDate 0x506de7eb "2012-01-02"] 10.9 10.6 2.8 4.5 "rain"])
\end{verbatim}

\begin{center}\rule{0.5\linewidth}{0.5pt}\end{center}

Rows as sequence of maps

\begin{Shaded}
\begin{Highlighting}[]
\NormalTok{(clojure.pprint/pprint (}\KeywordTok{take} \DecValTok{2}\NormalTok{ (api/rows ds }\AttributeTok{:as-maps}\NormalTok{)))}
\end{Highlighting}
\end{Shaded}

\begin{verbatim}
({"date" #object[java.time.LocalDate 0x172b080b "2012-01-01"],
  "precipitation" 0.0,
  "temp_min" 5.0,
  "weather" "drizzle",
  "temp_max" 12.8,
  "wind" 4.7}
 {"date" #object[java.time.LocalDate 0x5b83950 "2012-01-02"],
  "precipitation" 10.9,
  "temp_min" 2.8,
  "weather" "rain",
  "temp_max" 10.6,
  "wind" 4.5})
\end{verbatim}

\subsubsection{Group-by}\label{group-by}

Grouping by is an operation which splits dataset into subdatasets and
pack it into new special type of\ldots{} dataset. I distinguish two
types of dataset: regular dataset and grouped dataset. The latter is the
result of grouping.

Grouped dataset is annotated in by \texttt{:grouped?} meta tag and
consist following columns:

\begin{itemize}
\tightlist
\item
  \texttt{:name} - group name or structure
\item
  \texttt{:group-id} - integer assigned to the group
\item
  \texttt{:count} - number of elements in a group
\item
  \texttt{:data} - groups as datasets
\end{itemize}

Almost all functions recognize type of the dataset (grouped or not) and
operate accordingly.

You can't apply reshaping or join/concat functions on grouped datasets.

\paragraph{Grouping}\label{grouping}

Grouping is done by calling \texttt{group-by} function with arguments:

\begin{itemize}
\tightlist
\item
  \texttt{ds} - dataset
\item
  \texttt{grouping-selector} - what to use for grouping
\item
  options:

  \begin{itemize}
  \tightlist
  \item
    \texttt{:result-type} - what to return:

    \begin{itemize}
    \tightlist
    \item
      \texttt{:as-dataset} (default) - return grouped dataset
    \item
      \texttt{:as-indexes} - return rows ids (row number from original
      dataset)
    \item
      \texttt{:as-map} - return map with group names as keys and
      subdataset as values
    \item
      \texttt{:as-seq} - return sequens of subdatasets
    \end{itemize}
  \item
    \texttt{:limit-columns} - list of the columns which should be
    returned during grouping by function.
  \end{itemize}
\end{itemize}

All subdatasets (groups) have set name as the group name, additionally
\texttt{group-id} is in meta.

Grouping can be done by:

\begin{itemize}
\tightlist
\item
  single column name
\item
  seq of column names
\item
  map of keys (group names) and row indexes
\item
  value returned by function taking row as map
\end{itemize}

Note: currently dataset inside dataset is printed recursively so it
renders poorly from markdown. So I will use \texttt{:as-seq} result type
to show just group names and groups.

\begin{center}\rule{0.5\linewidth}{0.5pt}\end{center}

List of columns in groupd dataset

\begin{Shaded}
\begin{Highlighting}[]
\NormalTok{(api/column-names (api/group-by DS }\AttributeTok{:V1}\NormalTok{))}
\end{Highlighting}
\end{Shaded}

\begin{verbatim}
(:name :group-id :count :data)
\end{verbatim}

\begin{center}\rule{0.5\linewidth}{0.5pt}\end{center}

Content of the grouped dataset

\begin{Shaded}
\begin{Highlighting}[]
\NormalTok{(api/columns (api/group-by DS }\AttributeTok{:V1}\NormalTok{) }\AttributeTok{:as-map}\NormalTok{)}
\end{Highlighting}
\end{Shaded}

\begin{verbatim}
{:name #tech.ml.dataset.column<int64>[2]
:name
[1, 2, ], :group-id #tech.ml.dataset.column<int64>[2]
:group-id
[0, 1, ], :count #tech.ml.dataset.column<int32>[2]
:count
[5, 4, ], :data #tech.ml.dataset.column<object>[2]
:data
[1 [5 4]:

| :V1 | :V2 |    :V3 | :V4 |
|-----+-----+--------+-----|
|   1 |   1 | 0.5000 |   A |
|   1 |   3 |  1.500 |   C |
|   1 |   5 |  1.000 |   B |
|   1 |   7 | 0.5000 |   A |
|   1 |   9 |  1.500 |   C |
, 2 [4 4]:

| :V1 | :V2 |    :V3 | :V4 |
|-----+-----+--------+-----|
|   2 |   2 |  1.000 |   B |
|   2 |   4 | 0.5000 |   A |
|   2 |   6 |  1.500 |   C |
|   2 |   8 |  1.000 |   B |
, ]}
\end{verbatim}

\begin{center}\rule{0.5\linewidth}{0.5pt}\end{center}

Grouped dataset as map

\begin{Shaded}
\begin{Highlighting}[]
\NormalTok{(}\KeywordTok{keys}\NormalTok{ (api/group-by DS }\AttributeTok{:V1}\NormalTok{ \{}\AttributeTok{:result-type} \AttributeTok{:as-map}\NormalTok{\}))}
\end{Highlighting}
\end{Shaded}

\begin{verbatim}
(1 2)
\end{verbatim}

\begin{Shaded}
\begin{Highlighting}[]
\NormalTok{(}\KeywordTok{vals}\NormalTok{ (api/group-by DS }\AttributeTok{:V1}\NormalTok{ \{}\AttributeTok{:result-type} \AttributeTok{:as-map}\NormalTok{\}))}
\end{Highlighting}
\end{Shaded}

(1 {[}5 4{]}:

\begin{longtable}[]{@{}llll@{}}
\toprule
:V1 & :V2 & :V3 & :V4\tabularnewline
\midrule
\endhead
1 & 1 & 0.5000 & A\tabularnewline
1 & 3 & 1.500 & C\tabularnewline
1 & 5 & 1.000 & B\tabularnewline
1 & 7 & 0.5000 & A\tabularnewline
1 & 9 & 1.500 & C\tabularnewline
\bottomrule
\end{longtable}

2 {[}4 4{]}:

\begin{longtable}[]{@{}llll@{}}
\toprule
:V1 & :V2 & :V3 & :V4\tabularnewline
\midrule
\endhead
2 & 2 & 1.000 & B\tabularnewline
2 & 4 & 0.5000 & A\tabularnewline
2 & 6 & 1.500 & C\tabularnewline
2 & 8 & 1.000 & B\tabularnewline
\bottomrule
\end{longtable}

)

\begin{center}\rule{0.5\linewidth}{0.5pt}\end{center}

Group dataset as map of indexes (row ids)

\begin{Shaded}
\begin{Highlighting}[]
\NormalTok{(api/group-by DS }\AttributeTok{:V1}\NormalTok{ \{}\AttributeTok{:result-type} \AttributeTok{:as-indexes}\NormalTok{\})}
\end{Highlighting}
\end{Shaded}

\begin{verbatim}
{1 [0 2 4 6 8], 2 [1 3 5 7]}
\end{verbatim}

\begin{center}\rule{0.5\linewidth}{0.5pt}\end{center}

To get groups as sequence or a map can be done from grouped dataset
using \texttt{groups-\textgreater{}seq} and
\texttt{groups-\textgreater{}map} functions.

Groups as seq can be obtained by just accessing \texttt{:data} column.

I will use temporary dataset here.

\begin{Shaded}
\begin{Highlighting}[]
\NormalTok{(}\KeywordTok{let}\NormalTok{ [ds (}\KeywordTok{->}\NormalTok{ \{}\StringTok{"a"}\NormalTok{ [}\DecValTok{1} \DecValTok{1} \DecValTok{2} \DecValTok{2}\NormalTok{]}
              \StringTok{"b"}\NormalTok{ [}\StringTok{"a"} \StringTok{"b"} \StringTok{"c"} \StringTok{"d"}\NormalTok{]\}}
\NormalTok{             (api/dataset)}
\NormalTok{             (api/group-by }\StringTok{"a"}\NormalTok{))]}
\NormalTok{  (}\KeywordTok{seq}\NormalTok{ (ds }\AttributeTok{:data}\NormalTok{))) }\CommentTok{;; seq is not necessary but Markdown treats `:data` as command here}
\end{Highlighting}
\end{Shaded}

(1 {[}2 2{]}:

\begin{longtable}[]{@{}ll@{}}
\toprule
a & b\tabularnewline
\midrule
\endhead
1 & a\tabularnewline
1 & b\tabularnewline
\bottomrule
\end{longtable}

2 {[}2 2{]}:

\begin{longtable}[]{@{}ll@{}}
\toprule
a & b\tabularnewline
\midrule
\endhead
2 & c\tabularnewline
2 & d\tabularnewline
\bottomrule
\end{longtable}

)

\begin{Shaded}
\begin{Highlighting}[]
\NormalTok{(}\KeywordTok{->}\NormalTok{ \{}\StringTok{"a"}\NormalTok{ [}\DecValTok{1} \DecValTok{1} \DecValTok{2} \DecValTok{2}\NormalTok{]}
     \StringTok{"b"}\NormalTok{ [}\StringTok{"a"} \StringTok{"b"} \StringTok{"c"} \StringTok{"d"}\NormalTok{]\}}
\NormalTok{    (api/dataset)}
\NormalTok{    (api/group-by }\StringTok{"a"}\NormalTok{)}
\NormalTok{    (api/groups->seq))}
\end{Highlighting}
\end{Shaded}

(1 {[}2 2{]}:

\begin{longtable}[]{@{}ll@{}}
\toprule
a & b\tabularnewline
\midrule
\endhead
1 & a\tabularnewline
1 & b\tabularnewline
\bottomrule
\end{longtable}

2 {[}2 2{]}:

\begin{longtable}[]{@{}ll@{}}
\toprule
a & b\tabularnewline
\midrule
\endhead
2 & c\tabularnewline
2 & d\tabularnewline
\bottomrule
\end{longtable}

)

\begin{center}\rule{0.5\linewidth}{0.5pt}\end{center}

Groups as map

\begin{Shaded}
\begin{Highlighting}[]
\NormalTok{(}\KeywordTok{->}\NormalTok{ \{}\StringTok{"a"}\NormalTok{ [}\DecValTok{1} \DecValTok{1} \DecValTok{2} \DecValTok{2}\NormalTok{]}
     \StringTok{"b"}\NormalTok{ [}\StringTok{"a"} \StringTok{"b"} \StringTok{"c"} \StringTok{"d"}\NormalTok{]\}}
\NormalTok{    (api/dataset)}
\NormalTok{    (api/group-by }\StringTok{"a"}\NormalTok{)}
\NormalTok{    (api/groups->map))}
\end{Highlighting}
\end{Shaded}

\{1 1 {[}2 2{]}:

\begin{longtable}[]{@{}ll@{}}
\toprule
a & b\tabularnewline
\midrule
\endhead
1 & a\tabularnewline
1 & b\tabularnewline
\bottomrule
\end{longtable}

, 2 2 {[}2 2{]}:

\begin{longtable}[]{@{}ll@{}}
\toprule
a & b\tabularnewline
\midrule
\endhead
2 & c\tabularnewline
2 & d\tabularnewline
\bottomrule
\end{longtable}

\}

\begin{center}\rule{0.5\linewidth}{0.5pt}\end{center}

Grouping by more than one column. You can see that group names are maps.
When ungrouping is done these maps are used to restore column names.

\begin{Shaded}
\begin{Highlighting}[]
\NormalTok{(api/group-by DS [}\AttributeTok{:V1} \AttributeTok{:V3}\NormalTok{] \{}\AttributeTok{:result-type} \AttributeTok{:as-seq}\NormalTok{\})}
\end{Highlighting}
\end{Shaded}

(\{:V3 1.0, :V1 1\} {[}1 4{]}:

\begin{longtable}[]{@{}llll@{}}
\toprule
:V1 & :V2 & :V3 & :V4\tabularnewline
\midrule
\endhead
1 & 5 & 1.000 & B\tabularnewline
\bottomrule
\end{longtable}

\{:V3 0.5, :V1 1\} {[}2 4{]}:

\begin{longtable}[]{@{}llll@{}}
\toprule
:V1 & :V2 & :V3 & :V4\tabularnewline
\midrule
\endhead
1 & 1 & 0.5000 & A\tabularnewline
1 & 7 & 0.5000 & A\tabularnewline
\bottomrule
\end{longtable}

\{:V3 0.5, :V1 2\} {[}1 4{]}:

\begin{longtable}[]{@{}llll@{}}
\toprule
:V1 & :V2 & :V3 & :V4\tabularnewline
\midrule
\endhead
2 & 4 & 0.5000 & A\tabularnewline
\bottomrule
\end{longtable}

\{:V3 1.0, :V1 2\} {[}2 4{]}:

\begin{longtable}[]{@{}llll@{}}
\toprule
:V1 & :V2 & :V3 & :V4\tabularnewline
\midrule
\endhead
2 & 2 & 1.000 & B\tabularnewline
2 & 8 & 1.000 & B\tabularnewline
\bottomrule
\end{longtable}

\{:V3 1.5, :V1 1\} {[}2 4{]}:

\begin{longtable}[]{@{}llll@{}}
\toprule
:V1 & :V2 & :V3 & :V4\tabularnewline
\midrule
\endhead
1 & 3 & 1.500 & C\tabularnewline
1 & 9 & 1.500 & C\tabularnewline
\bottomrule
\end{longtable}

\{:V3 1.5, :V1 2\} {[}1 4{]}:

\begin{longtable}[]{@{}llll@{}}
\toprule
:V1 & :V2 & :V3 & :V4\tabularnewline
\midrule
\endhead
2 & 6 & 1.500 & C\tabularnewline
\bottomrule
\end{longtable}

)

\begin{center}\rule{0.5\linewidth}{0.5pt}\end{center}

Grouping can be done by providing just row indexes. This way you can
assign the same row to more than one group.

\begin{Shaded}
\begin{Highlighting}[]
\NormalTok{(api/group-by DS \{}\StringTok{"group-a"}\NormalTok{ [}\DecValTok{1} \DecValTok{2} \DecValTok{1} \DecValTok{2}\NormalTok{]}
                  \StringTok{"group-b"}\NormalTok{ [}\DecValTok{5} \DecValTok{5} \DecValTok{5} \DecValTok{1}\NormalTok{]\} \{}\AttributeTok{:result-type} \AttributeTok{:as-seq}\NormalTok{\})}
\end{Highlighting}
\end{Shaded}

(group-a {[}4 4{]}:

\begin{longtable}[]{@{}llll@{}}
\toprule
:V1 & :V2 & :V3 & :V4\tabularnewline
\midrule
\endhead
2 & 2 & 1.000 & B\tabularnewline
1 & 3 & 1.500 & C\tabularnewline
2 & 2 & 1.000 & B\tabularnewline
1 & 3 & 1.500 & C\tabularnewline
\bottomrule
\end{longtable}

group-b {[}4 4{]}:

\begin{longtable}[]{@{}llll@{}}
\toprule
:V1 & :V2 & :V3 & :V4\tabularnewline
\midrule
\endhead
2 & 6 & 1.500 & C\tabularnewline
2 & 6 & 1.500 & C\tabularnewline
2 & 6 & 1.500 & C\tabularnewline
2 & 2 & 1.000 & B\tabularnewline
\bottomrule
\end{longtable}

)

\begin{center}\rule{0.5\linewidth}{0.5pt}\end{center}

You can group by a result of gruping function which gets row as map and
should return group name. When map is used as a group name, ungrouping
restore original column names.

\begin{Shaded}
\begin{Highlighting}[]
\NormalTok{(api/group-by DS (}\KeywordTok{fn}\NormalTok{ [row] (}\KeywordTok{*}\NormalTok{ (}\AttributeTok{:V1}\NormalTok{ row)}
\NormalTok{                             (}\AttributeTok{:V3}\NormalTok{ row))) \{}\AttributeTok{:result-type} \AttributeTok{:as-seq}\NormalTok{\})}
\end{Highlighting}
\end{Shaded}

(1.0 {[}2 4{]}:

\begin{longtable}[]{@{}llll@{}}
\toprule
:V1 & :V2 & :V3 & :V4\tabularnewline
\midrule
\endhead
2 & 4 & 0.5000 & A\tabularnewline
1 & 5 & 1.000 & B\tabularnewline
\bottomrule
\end{longtable}

2.0 {[}2 4{]}:

\begin{longtable}[]{@{}llll@{}}
\toprule
:V1 & :V2 & :V3 & :V4\tabularnewline
\midrule
\endhead
2 & 2 & 1.000 & B\tabularnewline
2 & 8 & 1.000 & B\tabularnewline
\bottomrule
\end{longtable}

0.5 {[}2 4{]}:

\begin{longtable}[]{@{}llll@{}}
\toprule
:V1 & :V2 & :V3 & :V4\tabularnewline
\midrule
\endhead
1 & 1 & 0.5000 & A\tabularnewline
1 & 7 & 0.5000 & A\tabularnewline
\bottomrule
\end{longtable}

3.0 {[}1 4{]}:

\begin{longtable}[]{@{}llll@{}}
\toprule
:V1 & :V2 & :V3 & :V4\tabularnewline
\midrule
\endhead
2 & 6 & 1.500 & C\tabularnewline
\bottomrule
\end{longtable}

1.5 {[}2 4{]}:

\begin{longtable}[]{@{}llll@{}}
\toprule
:V1 & :V2 & :V3 & :V4\tabularnewline
\midrule
\endhead
1 & 3 & 1.500 & C\tabularnewline
1 & 9 & 1.500 & C\tabularnewline
\bottomrule
\end{longtable}

)

\begin{center}\rule{0.5\linewidth}{0.5pt}\end{center}

You can use any predicate on column to split dataset into two groups.

\begin{Shaded}
\begin{Highlighting}[]
\NormalTok{(api/group-by DS (}\KeywordTok{comp}\NormalTok{ #(}\KeywordTok{<} \VariableTok{%} \FloatTok{1.0}\NormalTok{) }\AttributeTok{:V3}\NormalTok{) \{}\AttributeTok{:result-type} \AttributeTok{:as-seq}\NormalTok{\})}
\end{Highlighting}
\end{Shaded}

(false {[}6 4{]}:

\begin{longtable}[]{@{}llll@{}}
\toprule
:V1 & :V2 & :V3 & :V4\tabularnewline
\midrule
\endhead
2 & 2 & 1.000 & B\tabularnewline
1 & 3 & 1.500 & C\tabularnewline
1 & 5 & 1.000 & B\tabularnewline
2 & 6 & 1.500 & C\tabularnewline
2 & 8 & 1.000 & B\tabularnewline
1 & 9 & 1.500 & C\tabularnewline
\bottomrule
\end{longtable}

true {[}3 4{]}:

\begin{longtable}[]{@{}llll@{}}
\toprule
:V1 & :V2 & :V3 & :V4\tabularnewline
\midrule
\endhead
1 & 1 & 0.5000 & A\tabularnewline
2 & 4 & 0.5000 & A\tabularnewline
1 & 7 & 0.5000 & A\tabularnewline
\bottomrule
\end{longtable}

)

\begin{center}\rule{0.5\linewidth}{0.5pt}\end{center}

\texttt{juxt} is also helpful

\begin{Shaded}
\begin{Highlighting}[]
\NormalTok{(api/group-by DS (}\KeywordTok{juxt} \AttributeTok{:V1} \AttributeTok{:V3}\NormalTok{) \{}\AttributeTok{:result-type} \AttributeTok{:as-seq}\NormalTok{\})}
\end{Highlighting}
\end{Shaded}

({[}1 1.0{]} {[}1 4{]}:

\begin{longtable}[]{@{}llll@{}}
\toprule
:V1 & :V2 & :V3 & :V4\tabularnewline
\midrule
\endhead
1 & 5 & 1.000 & B\tabularnewline
\bottomrule
\end{longtable}

{[}1 0.5{]} {[}2 4{]}:

\begin{longtable}[]{@{}llll@{}}
\toprule
:V1 & :V2 & :V3 & :V4\tabularnewline
\midrule
\endhead
1 & 1 & 0.5000 & A\tabularnewline
1 & 7 & 0.5000 & A\tabularnewline
\bottomrule
\end{longtable}

{[}2 1.5{]} {[}1 4{]}:

\begin{longtable}[]{@{}llll@{}}
\toprule
:V1 & :V2 & :V3 & :V4\tabularnewline
\midrule
\endhead
2 & 6 & 1.500 & C\tabularnewline
\bottomrule
\end{longtable}

{[}1 1.5{]} {[}2 4{]}:

\begin{longtable}[]{@{}llll@{}}
\toprule
:V1 & :V2 & :V3 & :V4\tabularnewline
\midrule
\endhead
1 & 3 & 1.500 & C\tabularnewline
1 & 9 & 1.500 & C\tabularnewline
\bottomrule
\end{longtable}

{[}2 0.5{]} {[}1 4{]}:

\begin{longtable}[]{@{}llll@{}}
\toprule
:V1 & :V2 & :V3 & :V4\tabularnewline
\midrule
\endhead
2 & 4 & 0.5000 & A\tabularnewline
\bottomrule
\end{longtable}

{[}2 1.0{]} {[}2 4{]}:

\begin{longtable}[]{@{}llll@{}}
\toprule
:V1 & :V2 & :V3 & :V4\tabularnewline
\midrule
\endhead
2 & 2 & 1.000 & B\tabularnewline
2 & 8 & 1.000 & B\tabularnewline
\bottomrule
\end{longtable}

)

\begin{center}\rule{0.5\linewidth}{0.5pt}\end{center}

\texttt{tech.ml.dataset} provides an option to limit columns which are
passed to grouping functions. It's done for performance purposes.

\begin{Shaded}
\begin{Highlighting}[]
\NormalTok{(api/group-by DS }\KeywordTok{identity}\NormalTok{ \{}\AttributeTok{:result-type} \AttributeTok{:as-seq}
                           \AttributeTok{:limit-columns}\NormalTok{ [}\AttributeTok{:V1}\NormalTok{]\})}
\end{Highlighting}
\end{Shaded}

(\{:V1 1\} {[}5 4{]}:

\begin{longtable}[]{@{}llll@{}}
\toprule
:V1 & :V2 & :V3 & :V4\tabularnewline
\midrule
\endhead
1 & 1 & 0.5000 & A\tabularnewline
1 & 3 & 1.500 & C\tabularnewline
1 & 5 & 1.000 & B\tabularnewline
1 & 7 & 0.5000 & A\tabularnewline
1 & 9 & 1.500 & C\tabularnewline
\bottomrule
\end{longtable}

\{:V1 2\} {[}4 4{]}:

\begin{longtable}[]{@{}llll@{}}
\toprule
:V1 & :V2 & :V3 & :V4\tabularnewline
\midrule
\endhead
2 & 2 & 1.000 & B\tabularnewline
2 & 4 & 0.5000 & A\tabularnewline
2 & 6 & 1.500 & C\tabularnewline
2 & 8 & 1.000 & B\tabularnewline
\bottomrule
\end{longtable}

)

\paragraph{Ungrouping}\label{ungrouping}

Ungrouping simply concats all the groups into the dataset. Following
options are possible

\begin{itemize}
\tightlist
\item
  \texttt{:order?} - order groups according to the group name ascending
  order. Default: \texttt{false}
\item
  \texttt{:add-group-as-column} - should group name become a column? If
  yes column is created with provided name (or \texttt{:\$group-name} if
  argument is \texttt{true}). Default: \texttt{nil}.
\item
  \texttt{:add-group-id-as-column} - should group id become a column? If
  yes column is created with provided name (or \texttt{:\$group-id} if
  argument is \texttt{true}). Default: \texttt{nil}.
\item
  \texttt{:dataset-name} - to name resulting dataset. Default:
  \texttt{nil} (\_unnamed)
\end{itemize}

If group name is a map, it will be splitted into separate columns. Be
sure that groups (subdatasets) doesn't contain the same columns already.

If group name is a vector, it will be splitted into separate columns. If
you want to name them, set vector of target column names as
\texttt{:add-group-as-column} argument.

After ungrouping, order of the rows is kept within the groups but groups
are ordered according to the internal storage.

\begin{center}\rule{0.5\linewidth}{0.5pt}\end{center}

Grouping and ungrouping.

\begin{Shaded}
\begin{Highlighting}[]
\NormalTok{(}\KeywordTok{->}\NormalTok{ DS}
\NormalTok{    (api/group-by }\AttributeTok{:V3}\NormalTok{)}
\NormalTok{    (api/ungroup))}
\end{Highlighting}
\end{Shaded}

null {[}9 4{]}:

\begin{longtable}[]{@{}llll@{}}
\toprule
:V1 & :V2 & :V3 & :V4\tabularnewline
\midrule
\endhead
2 & 2 & 1.000 & B\tabularnewline
1 & 5 & 1.000 & B\tabularnewline
2 & 8 & 1.000 & B\tabularnewline
1 & 1 & 0.5000 & A\tabularnewline
2 & 4 & 0.5000 & A\tabularnewline
1 & 7 & 0.5000 & A\tabularnewline
1 & 3 & 1.500 & C\tabularnewline
2 & 6 & 1.500 & C\tabularnewline
1 & 9 & 1.500 & C\tabularnewline
\bottomrule
\end{longtable}

\begin{center}\rule{0.5\linewidth}{0.5pt}\end{center}

Groups sorted by group name and named.

\begin{Shaded}
\begin{Highlighting}[]
\NormalTok{(}\KeywordTok{->}\NormalTok{ DS}
\NormalTok{    (api/group-by }\AttributeTok{:V3}\NormalTok{)}
\NormalTok{    (api/ungroup \{}\AttributeTok{:order}\NormalTok{? }\VariableTok{true}
                  \AttributeTok{:dataset-name} \StringTok{"Ordered by V3"}\NormalTok{\}))}
\end{Highlighting}
\end{Shaded}

Ordered by V3 {[}9 4{]}:

\begin{longtable}[]{@{}llll@{}}
\toprule
:V1 & :V2 & :V3 & :V4\tabularnewline
\midrule
\endhead
1 & 1 & 0.5000 & A\tabularnewline
2 & 4 & 0.5000 & A\tabularnewline
1 & 7 & 0.5000 & A\tabularnewline
2 & 2 & 1.000 & B\tabularnewline
1 & 5 & 1.000 & B\tabularnewline
2 & 8 & 1.000 & B\tabularnewline
1 & 3 & 1.500 & C\tabularnewline
2 & 6 & 1.500 & C\tabularnewline
1 & 9 & 1.500 & C\tabularnewline
\bottomrule
\end{longtable}

\begin{center}\rule{0.5\linewidth}{0.5pt}\end{center}

Let's add group name and id as additional columns

\begin{Shaded}
\begin{Highlighting}[]
\NormalTok{(}\KeywordTok{->}\NormalTok{ DS}
\NormalTok{    (api/group-by (}\KeywordTok{comp}\NormalTok{ #(}\KeywordTok{<} \VariableTok{%} \DecValTok{4}\NormalTok{) }\AttributeTok{:V2}\NormalTok{))}
\NormalTok{    (api/ungroup \{}\AttributeTok{:add-group-as-column} \VariableTok{true}
                  \AttributeTok{:add-group-id-as-column} \VariableTok{true}\NormalTok{\}))}
\end{Highlighting}
\end{Shaded}

null {[}9 6{]}:

\begin{longtable}[]{@{}llllll@{}}
\toprule
:\$group-name & :\$group-id & :V1 & :V2 & :V3 & :V4\tabularnewline
\midrule
\endhead
false & 0 & 2 & 4 & 0.5000 & A\tabularnewline
false & 0 & 1 & 5 & 1.000 & B\tabularnewline
false & 0 & 2 & 6 & 1.500 & C\tabularnewline
false & 0 & 1 & 7 & 0.5000 & A\tabularnewline
false & 0 & 2 & 8 & 1.000 & B\tabularnewline
false & 0 & 1 & 9 & 1.500 & C\tabularnewline
true & 1 & 1 & 1 & 0.5000 & A\tabularnewline
true & 1 & 2 & 2 & 1.000 & B\tabularnewline
true & 1 & 1 & 3 & 1.500 & C\tabularnewline
\bottomrule
\end{longtable}

\begin{center}\rule{0.5\linewidth}{0.5pt}\end{center}

Let's assign different column names

\begin{Shaded}
\begin{Highlighting}[]
\NormalTok{(}\KeywordTok{->}\NormalTok{ DS}
\NormalTok{    (api/group-by (}\KeywordTok{comp}\NormalTok{ #(}\KeywordTok{<} \VariableTok{%} \DecValTok{4}\NormalTok{) }\AttributeTok{:V2}\NormalTok{))}
\NormalTok{    (api/ungroup \{}\AttributeTok{:add-group-as-column} \StringTok{"Is V2 less than 4?"}
                  \AttributeTok{:add-group-id-as-column} \StringTok{"group id"}\NormalTok{\}))}
\end{Highlighting}
\end{Shaded}

null {[}9 6{]}:

\begin{longtable}[]{@{}llllll@{}}
\toprule
Is V2 less than 4? & group id & :V1 & :V2 & :V3 & :V4\tabularnewline
\midrule
\endhead
false & 0 & 2 & 4 & 0.5000 & A\tabularnewline
false & 0 & 1 & 5 & 1.000 & B\tabularnewline
false & 0 & 2 & 6 & 1.500 & C\tabularnewline
false & 0 & 1 & 7 & 0.5000 & A\tabularnewline
false & 0 & 2 & 8 & 1.000 & B\tabularnewline
false & 0 & 1 & 9 & 1.500 & C\tabularnewline
true & 1 & 1 & 1 & 0.5000 & A\tabularnewline
true & 1 & 2 & 2 & 1.000 & B\tabularnewline
true & 1 & 1 & 3 & 1.500 & C\tabularnewline
\bottomrule
\end{longtable}

\begin{center}\rule{0.5\linewidth}{0.5pt}\end{center}

If we group by map, we can automatically create new columns out of group
names.

\begin{Shaded}
\begin{Highlighting}[]
\NormalTok{(}\KeywordTok{->}\NormalTok{ DS}
\NormalTok{    (api/group-by (}\KeywordTok{fn}\NormalTok{ [row] \{}\StringTok{"V1 and V3 multiplied"}\NormalTok{ (}\KeywordTok{*}\NormalTok{ (}\AttributeTok{:V1}\NormalTok{ row)}
\NormalTok{                                                      (}\AttributeTok{:V3}\NormalTok{ row))}
                            \StringTok{"V4 as string"}\NormalTok{ (}\KeywordTok{str}\NormalTok{ (}\AttributeTok{:V4}\NormalTok{ row))\}))}
\NormalTok{    (api/ungroup \{}\AttributeTok{:add-group-as-column} \VariableTok{true}\NormalTok{\}))}
\end{Highlighting}
\end{Shaded}

null {[}9 6{]}:

\begin{longtable}[]{@{}llllll@{}}
\toprule
V1 and V3 multiplied & V4 as string & :V1 & :V2 & :V3 &
:V4\tabularnewline
\midrule
\endhead
3.000 & C & 2 & 6 & 1.500 & C\tabularnewline
1.500 & C & 1 & 3 & 1.500 & C\tabularnewline
1.500 & C & 1 & 9 & 1.500 & C\tabularnewline
1.000 & A & 2 & 4 & 0.5000 & A\tabularnewline
0.5000 & A & 1 & 1 & 0.5000 & A\tabularnewline
0.5000 & A & 1 & 7 & 0.5000 & A\tabularnewline
1.000 & B & 1 & 5 & 1.000 & B\tabularnewline
2.000 & B & 2 & 2 & 1.000 & B\tabularnewline
2.000 & B & 2 & 8 & 1.000 & B\tabularnewline
\bottomrule
\end{longtable}

\begin{center}\rule{0.5\linewidth}{0.5pt}\end{center}

We can add group names without separation

\begin{Shaded}
\begin{Highlighting}[]
\NormalTok{(}\KeywordTok{->}\NormalTok{ DS}
\NormalTok{    (api/group-by (}\KeywordTok{fn}\NormalTok{ [row] \{}\StringTok{"V1 and V3 multiplied"}\NormalTok{ (}\KeywordTok{*}\NormalTok{ (}\AttributeTok{:V1}\NormalTok{ row)}
\NormalTok{                                                      (}\AttributeTok{:V3}\NormalTok{ row))}
                            \StringTok{"V4 as string"}\NormalTok{ (}\KeywordTok{str}\NormalTok{ (}\AttributeTok{:V4}\NormalTok{ row))\}))}
\NormalTok{    (api/ungroup \{}\AttributeTok{:add-group-as-column} \StringTok{"just map"}
                  \AttributeTok{:separate}\NormalTok{? }\VariableTok{false}\NormalTok{\}))}
\end{Highlighting}
\end{Shaded}

null {[}9 5{]}:

\begin{longtable}[]{@{}lllll@{}}
\toprule
just map & :V1 & :V2 & :V3 & :V4\tabularnewline
\midrule
\endhead
\{``V1 and V3 multiplied'' 3.0, ``V4 as string'' ``C''\} & 2 & 6 & 1.500
& C\tabularnewline
\{``V1 and V3 multiplied'' 1.5, ``V4 as string'' ``C''\} & 1 & 3 & 1.500
& C\tabularnewline
\{``V1 and V3 multiplied'' 1.5, ``V4 as string'' ``C''\} & 1 & 9 & 1.500
& C\tabularnewline
\{``V1 and V3 multiplied'' 1.0, ``V4 as string'' ``A''\} & 2 & 4 &
0.5000 & A\tabularnewline
\{``V1 and V3 multiplied'' 0.5, ``V4 as string'' ``A''\} & 1 & 1 &
0.5000 & A\tabularnewline
\{``V1 and V3 multiplied'' 0.5, ``V4 as string'' ``A''\} & 1 & 7 &
0.5000 & A\tabularnewline
\{``V1 and V3 multiplied'' 1.0, ``V4 as string'' ``B''\} & 1 & 5 & 1.000
& B\tabularnewline
\{``V1 and V3 multiplied'' 2.0, ``V4 as string'' ``B''\} & 2 & 2 & 1.000
& B\tabularnewline
\{``V1 and V3 multiplied'' 2.0, ``V4 as string'' ``B''\} & 2 & 8 & 1.000
& B\tabularnewline
\bottomrule
\end{longtable}

\begin{center}\rule{0.5\linewidth}{0.5pt}\end{center}

The same applies to group names as sequences

\begin{Shaded}
\begin{Highlighting}[]
\NormalTok{(}\KeywordTok{->}\NormalTok{ DS}
\NormalTok{    (api/group-by (}\KeywordTok{juxt} \AttributeTok{:V1} \AttributeTok{:V3}\NormalTok{))}
\NormalTok{    (api/ungroup \{}\AttributeTok{:add-group-as-column} \StringTok{"abc"}\NormalTok{\}))}
\end{Highlighting}
\end{Shaded}

null {[}9 6{]}:

\begin{longtable}[]{@{}llllll@{}}
\toprule
:abc-0 & :abc-1 & :V1 & :V2 & :V3 & :V4\tabularnewline
\midrule
\endhead
1 & 1.000 & 1 & 5 & 1.000 & B\tabularnewline
1 & 0.5000 & 1 & 1 & 0.5000 & A\tabularnewline
1 & 0.5000 & 1 & 7 & 0.5000 & A\tabularnewline
2 & 1.500 & 2 & 6 & 1.500 & C\tabularnewline
1 & 1.500 & 1 & 3 & 1.500 & C\tabularnewline
1 & 1.500 & 1 & 9 & 1.500 & C\tabularnewline
2 & 0.5000 & 2 & 4 & 0.5000 & A\tabularnewline
2 & 1.000 & 2 & 2 & 1.000 & B\tabularnewline
2 & 1.000 & 2 & 8 & 1.000 & B\tabularnewline
\bottomrule
\end{longtable}

\begin{center}\rule{0.5\linewidth}{0.5pt}\end{center}

Let's provide column names

\begin{Shaded}
\begin{Highlighting}[]
\NormalTok{(}\KeywordTok{->}\NormalTok{ DS}
\NormalTok{    (api/group-by (}\KeywordTok{juxt} \AttributeTok{:V1} \AttributeTok{:V3}\NormalTok{))}
\NormalTok{    (api/ungroup \{}\AttributeTok{:add-group-as-column}\NormalTok{ [}\StringTok{"v1"} \StringTok{"v3"}\NormalTok{]\}))}
\end{Highlighting}
\end{Shaded}

null {[}9 6{]}:

\begin{longtable}[]{@{}llllll@{}}
\toprule
v1 & v3 & :V1 & :V2 & :V3 & :V4\tabularnewline
\midrule
\endhead
1 & 1.000 & 1 & 5 & 1.000 & B\tabularnewline
1 & 0.5000 & 1 & 1 & 0.5000 & A\tabularnewline
1 & 0.5000 & 1 & 7 & 0.5000 & A\tabularnewline
2 & 1.500 & 2 & 6 & 1.500 & C\tabularnewline
1 & 1.500 & 1 & 3 & 1.500 & C\tabularnewline
1 & 1.500 & 1 & 9 & 1.500 & C\tabularnewline
2 & 0.5000 & 2 & 4 & 0.5000 & A\tabularnewline
2 & 1.000 & 2 & 2 & 1.000 & B\tabularnewline
2 & 1.000 & 2 & 8 & 1.000 & B\tabularnewline
\bottomrule
\end{longtable}

\begin{center}\rule{0.5\linewidth}{0.5pt}\end{center}

Also we can supress separation

\begin{Shaded}
\begin{Highlighting}[]
\NormalTok{(}\KeywordTok{->}\NormalTok{ DS}
\NormalTok{    (api/group-by (}\KeywordTok{juxt} \AttributeTok{:V1} \AttributeTok{:V3}\NormalTok{))}
\NormalTok{    (api/ungroup \{}\AttributeTok{:separate}\NormalTok{? }\VariableTok{false}
                  \AttributeTok{:add-group-as-column} \VariableTok{true}\NormalTok{\}))}
\end{Highlighting}
\end{Shaded}

null {[}9 5{]}:

\begin{longtable}[]{@{}lllll@{}}
\toprule
:\$group-name & :V1 & :V2 & :V3 & :V4\tabularnewline
\midrule
\endhead
{[}1 1.0{]} & 1 & 5 & 1.000 & B\tabularnewline
{[}1 0.5{]} & 1 & 1 & 0.5000 & A\tabularnewline
{[}1 0.5{]} & 1 & 7 & 0.5000 & A\tabularnewline
{[}2 1.5{]} & 2 & 6 & 1.500 & C\tabularnewline
{[}1 1.5{]} & 1 & 3 & 1.500 & C\tabularnewline
{[}1 1.5{]} & 1 & 9 & 1.500 & C\tabularnewline
{[}2 0.5{]} & 2 & 4 & 0.5000 & A\tabularnewline
{[}2 1.0{]} & 2 & 2 & 1.000 & B\tabularnewline
{[}2 1.0{]} & 2 & 8 & 1.000 & B\tabularnewline
\bottomrule
\end{longtable}

\paragraph{Other functions}\label{other-functions}

\subsubsection{Columns}\label{columns}

\subsubsection{Rows}\label{rows}

\subsubsection{Aggregate}\label{aggregate}

\subsubsection{Order}\label{order}

\subsubsection{Unique}\label{unique}

\subsubsection{Missing}\label{missing}

\subsubsection{Join/Split Columns}\label{joinsplit-columns}

\subsubsection{Fold/Unroll Rows}\label{foldunroll-rows}

\subsubsection{Reshape}\label{reshape}

\subsubsection{Join/Concat}\label{joinconcat}

\end{document}
