\documentclass[]{article}
\usepackage{lmodern}
\usepackage{amssymb,amsmath}
\usepackage{ifxetex,ifluatex}
\usepackage{fixltx2e} % provides \textsubscript
\ifnum 0\ifxetex 1\fi\ifluatex 1\fi=0 % if pdftex
  \usepackage[T1]{fontenc}
  \usepackage[utf8]{inputenc}
\else % if luatex or xelatex
  \ifxetex
    \usepackage{mathspec}
  \else
    \usepackage{fontspec}
  \fi
  \defaultfontfeatures{Ligatures=TeX,Scale=MatchLowercase}
\fi
% use upquote if available, for straight quotes in verbatim environments
\IfFileExists{upquote.sty}{\usepackage{upquote}}{}
% use microtype if available
\IfFileExists{microtype.sty}{%
\usepackage[]{microtype}
\UseMicrotypeSet[protrusion]{basicmath} % disable protrusion for tt fonts
}{}
\PassOptionsToPackage{hyphens}{url} % url is loaded by hyperref
\usepackage[unicode=true]{hyperref}
\hypersetup{
            pdftitle={Proposed API for tech.ml.dataset},
            pdfauthor={GenerateMe},
            pdfborder={0 0 0},
            breaklinks=true}
\urlstyle{same}  % don't use monospace font for urls
\usepackage[margin=1in]{geometry}
\usepackage{color}
\usepackage{fancyvrb}
\newcommand{\VerbBar}{|}
\newcommand{\VERB}{\Verb[commandchars=\\\{\}]}
\DefineVerbatimEnvironment{Highlighting}{Verbatim}{commandchars=\\\{\}}
% Add ',fontsize=\small' for more characters per line
\usepackage{framed}
\definecolor{shadecolor}{RGB}{248,248,248}
\newenvironment{Shaded}{\begin{snugshade}}{\end{snugshade}}
\newcommand{\KeywordTok}[1]{\textcolor[rgb]{0.13,0.29,0.53}{\textbf{#1}}}
\newcommand{\DataTypeTok}[1]{\textcolor[rgb]{0.13,0.29,0.53}{#1}}
\newcommand{\DecValTok}[1]{\textcolor[rgb]{0.00,0.00,0.81}{#1}}
\newcommand{\BaseNTok}[1]{\textcolor[rgb]{0.00,0.00,0.81}{#1}}
\newcommand{\FloatTok}[1]{\textcolor[rgb]{0.00,0.00,0.81}{#1}}
\newcommand{\ConstantTok}[1]{\textcolor[rgb]{0.00,0.00,0.00}{#1}}
\newcommand{\CharTok}[1]{\textcolor[rgb]{0.31,0.60,0.02}{#1}}
\newcommand{\SpecialCharTok}[1]{\textcolor[rgb]{0.00,0.00,0.00}{#1}}
\newcommand{\StringTok}[1]{\textcolor[rgb]{0.31,0.60,0.02}{#1}}
\newcommand{\VerbatimStringTok}[1]{\textcolor[rgb]{0.31,0.60,0.02}{#1}}
\newcommand{\SpecialStringTok}[1]{\textcolor[rgb]{0.31,0.60,0.02}{#1}}
\newcommand{\ImportTok}[1]{#1}
\newcommand{\CommentTok}[1]{\textcolor[rgb]{0.56,0.35,0.01}{\textit{#1}}}
\newcommand{\DocumentationTok}[1]{\textcolor[rgb]{0.56,0.35,0.01}{\textbf{\textit{#1}}}}
\newcommand{\AnnotationTok}[1]{\textcolor[rgb]{0.56,0.35,0.01}{\textbf{\textit{#1}}}}
\newcommand{\CommentVarTok}[1]{\textcolor[rgb]{0.56,0.35,0.01}{\textbf{\textit{#1}}}}
\newcommand{\OtherTok}[1]{\textcolor[rgb]{0.56,0.35,0.01}{#1}}
\newcommand{\FunctionTok}[1]{\textcolor[rgb]{0.00,0.00,0.00}{#1}}
\newcommand{\VariableTok}[1]{\textcolor[rgb]{0.00,0.00,0.00}{#1}}
\newcommand{\ControlFlowTok}[1]{\textcolor[rgb]{0.13,0.29,0.53}{\textbf{#1}}}
\newcommand{\OperatorTok}[1]{\textcolor[rgb]{0.81,0.36,0.00}{\textbf{#1}}}
\newcommand{\BuiltInTok}[1]{#1}
\newcommand{\ExtensionTok}[1]{#1}
\newcommand{\PreprocessorTok}[1]{\textcolor[rgb]{0.56,0.35,0.01}{\textit{#1}}}
\newcommand{\AttributeTok}[1]{\textcolor[rgb]{0.77,0.63,0.00}{#1}}
\newcommand{\RegionMarkerTok}[1]{#1}
\newcommand{\InformationTok}[1]{\textcolor[rgb]{0.56,0.35,0.01}{\textbf{\textit{#1}}}}
\newcommand{\WarningTok}[1]{\textcolor[rgb]{0.56,0.35,0.01}{\textbf{\textit{#1}}}}
\newcommand{\AlertTok}[1]{\textcolor[rgb]{0.94,0.16,0.16}{#1}}
\newcommand{\ErrorTok}[1]{\textcolor[rgb]{0.64,0.00,0.00}{\textbf{#1}}}
\newcommand{\NormalTok}[1]{#1}
\usepackage{longtable,booktabs}
% Fix footnotes in tables (requires footnote package)
\IfFileExists{footnote.sty}{\usepackage{footnote}\makesavenoteenv{long table}}{}
\usepackage{graphicx,grffile}
\makeatletter
\def\maxwidth{\ifdim\Gin@nat@width>\linewidth\linewidth\else\Gin@nat@width\fi}
\def\maxheight{\ifdim\Gin@nat@height>\textheight\textheight\else\Gin@nat@height\fi}
\makeatother
% Scale images if necessary, so that they will not overflow the page
% margins by default, and it is still possible to overwrite the defaults
% using explicit options in \includegraphics[width, height, ...]{}
\setkeys{Gin}{width=\maxwidth,height=\maxheight,keepaspectratio}
\IfFileExists{parskip.sty}{%
\usepackage{parskip}
}{% else
\setlength{\parindent}{0pt}
\setlength{\parskip}{6pt plus 2pt minus 1pt}
}
\setlength{\emergencystretch}{3em}  % prevent overfull lines
\providecommand{\tightlist}{%
  \setlength{\itemsep}{0pt}\setlength{\parskip}{0pt}}
\setcounter{secnumdepth}{0}
% Redefines (sub)paragraphs to behave more like sections
\ifx\paragraph\undefined\else
\let\oldparagraph\paragraph
\renewcommand{\paragraph}[1]{\oldparagraph{#1}\mbox{}}
\fi
\ifx\subparagraph\undefined\else
\let\oldsubparagraph\subparagraph
\renewcommand{\subparagraph}[1]{\oldsubparagraph{#1}\mbox{}}
\fi

% set default figure placement to htbp
\makeatletter
\def\fps@figure{htbp}
\makeatother


\title{Proposed API for tech.ml.dataset}
\author{GenerateMe}
\date{2020-05-22}

\begin{document}
\maketitle

\subsection{Introduction}\label{introduction}

\href{https://github.com/techascent/tech.ml.dataset}{tech.ml.dataset} is
a great and fast library which brings columnar dataset to the Clojure.
Chris Nuernberger has been working on this library for last year as a
part of bigger \texttt{tech.ml} stack.

I've started to test the library and help to fix uncovered bugs. My main
goal was to compare functionalities with the other standards from other
platforms. I focused on R solutions:
\href{https://dplyr.tidyverse.org/}{dplyr},
\href{https://tidyr.tidyverse.org/}{tidyr} and
\href{https://rdatatable.gitlab.io/data.table/}{data.table}.

During conversions of the examples I've come up how to reorganized
existing \texttt{tech.ml.dataset} functions into simple to use API. The
main goals were:

\begin{itemize}
\tightlist
\item
  Focus on dataset manipulation functionality, leaving other parts of
  \texttt{tech.ml} like pipelines, datatypes, readers, ML, etc.
\item
  Single entry point for common operations - one function dispatching on
  given arguments.
\item
  \texttt{group-by} results with special kind of dataset - a dataset
  containing subsets created after grouping as a column.
\item
  Most operations recognize regular dataset and grouped dataset and
  process data accordingly.
\item
  One function form to enable thread-first on dataset.
\end{itemize}

All proposed functions are grouped in tabs below. Select group to see
examples and details.

If you want to know more about \texttt{tech.ml.dataset} and
\texttt{tech.ml.datatype} please refer their documentation:

\begin{itemize}
\tightlist
\item
  \href{https://github.com/techascent/tech.datatype/blob/master/docs/cheatsheet.md}{Datatype}
\item
  \href{https://github.com/techascent/tech.datatype/blob/master/docs/datetime.md}{Date/time}
\item
  \href{https://github.com/techascent/tech.ml.dataset/blob/master/docs/walkthrough.md}{Dataset}
\end{itemize}

INFO: The future of this API is not known yet. Two directions are
possible: integration into \texttt{tech.ml} or development under
\texttt{Scicloj} organization. For the time being use this repo if you
want to try. Join the discussion on
\href{https://clojurians.zulipchat.com/\#narrow/stream/236259-tech.2Eml.2Edataset.2Edev/topic/api}{Zulip}

Let's require main namespace and define dataset used in most examples:

\begin{Shaded}
\begin{Highlighting}[]
\NormalTok{(}\KeywordTok{require}\NormalTok{ '[techtest.api }\AttributeTok{:as}\NormalTok{ api])}
\NormalTok{(}\BuiltInTok{def}\FunctionTok{ DS }\NormalTok{(api/dataset \{}\AttributeTok{:V1}\NormalTok{ (}\KeywordTok{take} \DecValTok{9}\NormalTok{ (}\KeywordTok{cycle}\NormalTok{ [}\DecValTok{1} \DecValTok{2}\NormalTok{]))}
                      \AttributeTok{:V2}\NormalTok{ (}\KeywordTok{range} \DecValTok{1} \DecValTok{10}\NormalTok{)}
                      \AttributeTok{:V3}\NormalTok{ (}\KeywordTok{take} \DecValTok{9}\NormalTok{ (}\KeywordTok{cycle}\NormalTok{ [}\FloatTok{0.5} \FloatTok{1.0} \FloatTok{1.5}\NormalTok{]))}
                      \AttributeTok{:V4}\NormalTok{ (}\KeywordTok{take} \DecValTok{9}\NormalTok{ (}\KeywordTok{cycle}\NormalTok{ [}\CharTok{\textbackslash{}A} \CharTok{\textbackslash{}B} \CharTok{\textbackslash{}C}\NormalTok{]))\}))}
\end{Highlighting}
\end{Shaded}

\begin{Shaded}
\begin{Highlighting}[]
\NormalTok{DS}
\end{Highlighting}
\end{Shaded}

\_unnamed {[}9 4{]}:

\begin{longtable}[]{@{}llll@{}}
\toprule
:V1 & :V2 & :V3 & :V4\tabularnewline
\midrule
\endhead
1 & 1 & 0.5000 & A\tabularnewline
2 & 2 & 1.000 & B\tabularnewline
1 & 3 & 1.500 & C\tabularnewline
2 & 4 & 0.5000 & A\tabularnewline
1 & 5 & 1.000 & B\tabularnewline
2 & 6 & 1.500 & C\tabularnewline
1 & 7 & 0.5000 & A\tabularnewline
2 & 8 & 1.000 & B\tabularnewline
1 & 9 & 1.500 & C\tabularnewline
\bottomrule
\end{longtable}

\subsection{Functionality}\label{functionality}

\subsubsection{Dataset}\label{dataset}

Dataset is a special type which can be considered as a map of columns
implemented around \texttt{tech.ml.datatype} library. Each column can be
considered as named sequence of typed data. Supported types include
integers, floats, string, boolean, date/time, objects etc.

\paragraph{Dataset creation}\label{dataset-creation}

Dataset can be created from various of types of Clojure structures and
files:

\begin{itemize}
\tightlist
\item
  single values
\item
  sequence of maps
\item
  map of sequences or values
\item
  sequence of columns (taken from other dataset or created manually)
\item
  sequence of pairs
\item
  file types: raw/gzipped csv/tsv, json, xls(x) taken from local file
  system or URL
\item
  input stream
\end{itemize}

\texttt{api/dataset} accepts:

\begin{itemize}
\tightlist
\item
  data
\item
  options (see documentation of
  \texttt{tech.ml.dataset/-\textgreater{}dataset} function for full
  list):
\item
  \texttt{:dataset-name} - name of the dataset
\item
  \texttt{:num-rows} - number of rows to read from file
\item
  \texttt{:header-row?} - indication if first row in file is a header
\item
  \texttt{:key-fn} - function applied to column names (eg.
  \texttt{keyword}, to convert column names to keywords)
\item
  \texttt{:separator} - column separator
\item
  \texttt{:single-value-column-name} - name of the column when single
  value is provided
\end{itemize}

\begin{center}\rule{0.5\linewidth}{0.5pt}\end{center}

Empty dataset.

\begin{Shaded}
\begin{Highlighting}[]
\NormalTok{(api/dataset)}
\end{Highlighting}
\end{Shaded}

\begin{verbatim}
_unnamed [0 0]:
\end{verbatim}

\begin{center}\rule{0.5\linewidth}{0.5pt}\end{center}

Dataset from single value.

\begin{Shaded}
\begin{Highlighting}[]
\NormalTok{(api/dataset }\DecValTok{999}\NormalTok{)}
\end{Highlighting}
\end{Shaded}

\_unnamed {[}1 1{]}:

\begin{longtable}[]{@{}l@{}}
\toprule
:\$value\tabularnewline
\midrule
\endhead
999\tabularnewline
\bottomrule
\end{longtable}

\begin{center}\rule{0.5\linewidth}{0.5pt}\end{center}

Set column name for single value. Also set the dataset name.

\begin{Shaded}
\begin{Highlighting}[]
\NormalTok{(api/dataset }\DecValTok{999}\NormalTok{ \{}\AttributeTok{:single-value-column-name} \StringTok{"my-single-value"}\NormalTok{\})}
\NormalTok{(api/dataset }\DecValTok{999}\NormalTok{ \{}\AttributeTok{:single-value-column-name} \StringTok{""}
                  \AttributeTok{:dataset-name} \StringTok{"Single value"}\NormalTok{\})}
\end{Highlighting}
\end{Shaded}

\_unnamed {[}1 1{]}:

\begin{longtable}[]{@{}l@{}}
\toprule
my-single-value\tabularnewline
\midrule
\endhead
999\tabularnewline
\bottomrule
\end{longtable}

Single value {[}1 1{]}:

\begin{longtable}[]{@{}l@{}}
\toprule
0\tabularnewline
\midrule
\endhead
999\tabularnewline
\bottomrule
\end{longtable}

\begin{center}\rule{0.5\linewidth}{0.5pt}\end{center}

Sequence of pairs (first = column name, second = value(s)).

\begin{Shaded}
\begin{Highlighting}[]
\NormalTok{(api/dataset [[}\AttributeTok{:A} \DecValTok{33}\NormalTok{] [}\AttributeTok{:B} \DecValTok{5}\NormalTok{] [}\AttributeTok{:C} \AttributeTok{:a}\NormalTok{]])}
\end{Highlighting}
\end{Shaded}

\_unnamed {[}1 3{]}:

\begin{longtable}[]{@{}lll@{}}
\toprule
:A & :B & :C\tabularnewline
\midrule
\endhead
33 & 5 & :a\tabularnewline
\bottomrule
\end{longtable}

\begin{center}\rule{0.5\linewidth}{0.5pt}\end{center}

Not sequential values are repeated row-count number of times.

\begin{Shaded}
\begin{Highlighting}[]
\NormalTok{(api/dataset [[}\AttributeTok{:A}\NormalTok{ [}\DecValTok{1} \DecValTok{2} \DecValTok{3} \DecValTok{4} \DecValTok{5} \DecValTok{6}\NormalTok{]] [}\AttributeTok{:B} \StringTok{"X"}\NormalTok{] [}\AttributeTok{:C} \AttributeTok{:a}\NormalTok{]])}
\end{Highlighting}
\end{Shaded}

\_unnamed {[}6 3{]}:

\begin{longtable}[]{@{}lll@{}}
\toprule
:A & :B & :C\tabularnewline
\midrule
\endhead
1 & X & :a\tabularnewline
2 & X & :a\tabularnewline
3 & X & :a\tabularnewline
4 & X & :a\tabularnewline
5 & X & :a\tabularnewline
6 & X & :a\tabularnewline
\bottomrule
\end{longtable}

\begin{center}\rule{0.5\linewidth}{0.5pt}\end{center}

Dataset created from map (keys = column name, second = value(s)). Works
the same as sequence of pairs.

\begin{Shaded}
\begin{Highlighting}[]
\NormalTok{(api/dataset \{}\AttributeTok{:A} \DecValTok{33}\NormalTok{\})}
\NormalTok{(api/dataset \{}\AttributeTok{:A}\NormalTok{ [}\DecValTok{1} \DecValTok{2} \DecValTok{3}\NormalTok{]\})}
\NormalTok{(api/dataset \{}\AttributeTok{:A}\NormalTok{ [}\DecValTok{3} \DecValTok{4} \DecValTok{5}\NormalTok{] }\AttributeTok{:B} \StringTok{"X"}\NormalTok{\})}
\end{Highlighting}
\end{Shaded}

\_unnamed {[}1 1{]}:

\begin{longtable}[]{@{}l@{}}
\toprule
:A\tabularnewline
\midrule
\endhead
33\tabularnewline
\bottomrule
\end{longtable}

\_unnamed {[}3 1{]}:

\begin{longtable}[]{@{}l@{}}
\toprule
:A\tabularnewline
\midrule
\endhead
1\tabularnewline
2\tabularnewline
3\tabularnewline
\bottomrule
\end{longtable}

\_unnamed {[}3 2{]}:

\begin{longtable}[]{@{}ll@{}}
\toprule
:A & :B\tabularnewline
\midrule
\endhead
3 & X\tabularnewline
4 & X\tabularnewline
5 & X\tabularnewline
\bottomrule
\end{longtable}

\begin{center}\rule{0.5\linewidth}{0.5pt}\end{center}

You can put any value inside a column

\begin{Shaded}
\begin{Highlighting}[]
\NormalTok{(api/dataset \{}\AttributeTok{:A}\NormalTok{ [[}\DecValTok{3} \DecValTok{4} \DecValTok{5}\NormalTok{] [}\AttributeTok{:a} \AttributeTok{:b}\NormalTok{]] }\AttributeTok{:B} \StringTok{"X"}\NormalTok{\})}
\end{Highlighting}
\end{Shaded}

\_unnamed {[}2 2{]}:

\begin{longtable}[]{@{}ll@{}}
\toprule
:A & :B\tabularnewline
\midrule
\endhead
{[}3 4 5{]} & X\tabularnewline
{[}:a :b{]} & X\tabularnewline
\bottomrule
\end{longtable}

\begin{center}\rule{0.5\linewidth}{0.5pt}\end{center}

Sequence of maps

\begin{Shaded}
\begin{Highlighting}[]
\NormalTok{(api/dataset [\{}\AttributeTok{:a} \DecValTok{1} \AttributeTok{:b} \DecValTok{3}\NormalTok{\} \{}\AttributeTok{:b} \DecValTok{2} \AttributeTok{:a} \DecValTok{99}\NormalTok{\}])}
\NormalTok{(api/dataset [\{}\AttributeTok{:a} \DecValTok{1} \AttributeTok{:b}\NormalTok{ [}\DecValTok{1} \DecValTok{2} \DecValTok{3}\NormalTok{]\} \{}\AttributeTok{:a} \DecValTok{2} \AttributeTok{:b}\NormalTok{ [}\DecValTok{3} \DecValTok{4}\NormalTok{]\}])}
\end{Highlighting}
\end{Shaded}

\_unnamed {[}2 2{]}:

\begin{longtable}[]{@{}ll@{}}
\toprule
:a & :b\tabularnewline
\midrule
\endhead
1 & 3\tabularnewline
99 & 2\tabularnewline
\bottomrule
\end{longtable}

\_unnamed {[}2 2{]}:

\begin{longtable}[]{@{}ll@{}}
\toprule
:a & :b\tabularnewline
\midrule
\endhead
1 & {[}1 2 3{]}\tabularnewline
2 & {[}3 4{]}\tabularnewline
\bottomrule
\end{longtable}

\begin{center}\rule{0.5\linewidth}{0.5pt}\end{center}

Missing values are marked by \texttt{nil}

\begin{Shaded}
\begin{Highlighting}[]
\NormalTok{(api/dataset [\{}\AttributeTok{:a} \VariableTok{nil} \AttributeTok{:b} \DecValTok{1}\NormalTok{\} \{}\AttributeTok{:a} \DecValTok{3} \AttributeTok{:b} \DecValTok{4}\NormalTok{\} \{}\AttributeTok{:a} \DecValTok{11}\NormalTok{\}])}
\end{Highlighting}
\end{Shaded}

\_unnamed {[}3 2{]}:

\begin{longtable}[]{@{}ll@{}}
\toprule
:a & :b\tabularnewline
\midrule
\endhead
& 1\tabularnewline
3 & 4\tabularnewline
11 &\tabularnewline
\bottomrule
\end{longtable}

\begin{center}\rule{0.5\linewidth}{0.5pt}\end{center}

Import CSV file

\begin{Shaded}
\begin{Highlighting}[]
\NormalTok{(api/dataset }\StringTok{"data/family.csv"}\NormalTok{)}
\end{Highlighting}
\end{Shaded}

data/family.csv {[}5 5{]}:

\begin{longtable}[]{@{}lllll@{}}
\toprule
family & dob\_child1 & dob\_child2 & gender\_child1 &
gender\_child2\tabularnewline
\midrule
\endhead
1 & 1998-11-26 & 2000-01-29 & 1 & 2\tabularnewline
2 & 1996-06-22 & & 2 &\tabularnewline
3 & 2002-07-11 & 2004-04-05 & 2 & 2\tabularnewline
4 & 2004-10-10 & 2009-08-27 & 1 & 1\tabularnewline
5 & 2000-12-05 & 2005-02-28 & 2 & 1\tabularnewline
\bottomrule
\end{longtable}

\begin{center}\rule{0.5\linewidth}{0.5pt}\end{center}

Import from URL

\begin{Shaded}
\begin{Highlighting}[]
\NormalTok{(}\BuiltInTok{defonce}\FunctionTok{ ds }\NormalTok{(api/dataset }\StringTok{"https://vega.github.io/vega-lite/examples/data/seattle-weather.csv"}\NormalTok{))}
\end{Highlighting}
\end{Shaded}

\begin{Shaded}
\begin{Highlighting}[]
\NormalTok{ds}
\end{Highlighting}
\end{Shaded}

\url{https://vega.github.io/vega-lite/examples/data/seattle-weather.csv}
{[}1461 6{]}:

\begin{longtable}[]{@{}llllll@{}}
\toprule
date & precipitation & temp\_max & temp\_min & wind &
weather\tabularnewline
\midrule
\endhead
2012-01-01 & 0.000 & 12.80 & 5.000 & 4.700 & drizzle\tabularnewline
2012-01-02 & 10.90 & 10.60 & 2.800 & 4.500 & rain\tabularnewline
2012-01-03 & 0.8000 & 11.70 & 7.200 & 2.300 & rain\tabularnewline
2012-01-04 & 20.30 & 12.20 & 5.600 & 4.700 & rain\tabularnewline
2012-01-05 & 1.300 & 8.900 & 2.800 & 6.100 & rain\tabularnewline
2012-01-06 & 2.500 & 4.400 & 2.200 & 2.200 & rain\tabularnewline
2012-01-07 & 0.000 & 7.200 & 2.800 & 2.300 & rain\tabularnewline
2012-01-08 & 0.000 & 10.00 & 2.800 & 2.000 & sun\tabularnewline
2012-01-09 & 4.300 & 9.400 & 5.000 & 3.400 & rain\tabularnewline
2012-01-10 & 1.000 & 6.100 & 0.6000 & 3.400 & rain\tabularnewline
2012-01-11 & 0.000 & 6.100 & -1.100 & 5.100 & sun\tabularnewline
2012-01-12 & 0.000 & 6.100 & -1.700 & 1.900 & sun\tabularnewline
2012-01-13 & 0.000 & 5.000 & -2.800 & 1.300 & sun\tabularnewline
2012-01-14 & 4.100 & 4.400 & 0.6000 & 5.300 & snow\tabularnewline
2012-01-15 & 5.300 & 1.100 & -3.300 & 3.200 & snow\tabularnewline
2012-01-16 & 2.500 & 1.700 & -2.800 & 5.000 & snow\tabularnewline
2012-01-17 & 8.100 & 3.300 & 0.000 & 5.600 & snow\tabularnewline
2012-01-18 & 19.80 & 0.000 & -2.800 & 5.000 & snow\tabularnewline
2012-01-19 & 15.20 & -1.100 & -2.800 & 1.600 & snow\tabularnewline
2012-01-20 & 13.50 & 7.200 & -1.100 & 2.300 & snow\tabularnewline
2012-01-21 & 3.000 & 8.300 & 3.300 & 8.200 & rain\tabularnewline
2012-01-22 & 6.100 & 6.700 & 2.200 & 4.800 & rain\tabularnewline
2012-01-23 & 0.000 & 8.300 & 1.100 & 3.600 & rain\tabularnewline
2012-01-24 & 8.600 & 10.00 & 2.200 & 5.100 & rain\tabularnewline
2012-01-25 & 8.100 & 8.900 & 4.400 & 5.400 & rain\tabularnewline
\bottomrule
\end{longtable}

\paragraph{Saving}\label{saving}

Export dataset to a file or output stream can be done by calling
\texttt{api/write-csv!}. Function accepts:

\begin{itemize}
\tightlist
\item
  dataset
\item
  file name with one of the extensions: \texttt{.csv}, \texttt{.tsv},
  \texttt{.csv.gz} and \texttt{.tsv.gz} or output stream
\item
  options:
\item
  \texttt{:separator} - string or separator char.
\end{itemize}

\begin{Shaded}
\begin{Highlighting}[]
\NormalTok{(api/write-csv! ds }\StringTok{"output.tsv.gz"}\NormalTok{)}
\NormalTok{(.exists (clojure.java.io/file }\StringTok{"output.csv.gz"}\NormalTok{))}
\end{Highlighting}
\end{Shaded}

\begin{verbatim}
nil
true
\end{verbatim}

\paragraph{Dataset related functions}\label{dataset-related-functions}

Summary functions about the dataset like number of rows, columns and
basic stats.

\begin{center}\rule{0.5\linewidth}{0.5pt}\end{center}

Number of rows

\begin{Shaded}
\begin{Highlighting}[]
\NormalTok{(api/row-count ds)}
\end{Highlighting}
\end{Shaded}

\begin{verbatim}
1461
\end{verbatim}

\begin{center}\rule{0.5\linewidth}{0.5pt}\end{center}

Number of columns

\begin{Shaded}
\begin{Highlighting}[]
\NormalTok{(api/column-count ds)}
\end{Highlighting}
\end{Shaded}

\begin{verbatim}
6
\end{verbatim}

\begin{center}\rule{0.5\linewidth}{0.5pt}\end{center}

Shape of the dataset, {[}row count, column count{]}

\begin{Shaded}
\begin{Highlighting}[]
\NormalTok{(api/shape ds)}
\end{Highlighting}
\end{Shaded}

\begin{verbatim}
[1461 6]
\end{verbatim}

\begin{center}\rule{0.5\linewidth}{0.5pt}\end{center}

General info about dataset. There are three variants:

\begin{itemize}
\tightlist
\item
  default - containing information about columns with basic statistics
\item
  \texttt{:basic} - just name, row and column count and information if
  dataset is a result of \texttt{group-by} operation
\item
  \texttt{:columns} - columns' metadata
\end{itemize}

\begin{Shaded}
\begin{Highlighting}[]
\NormalTok{(api/info ds)}
\NormalTok{(api/info ds }\AttributeTok{:basic}\NormalTok{)}
\NormalTok{(api/info ds }\AttributeTok{:columns}\NormalTok{)}
\end{Highlighting}
\end{Shaded}

\url{https://vega.github.io/vega-lite/examples/data/seattle-weather.csv}:
descriptive-stats {[}6 10{]}:

\begin{longtable}[]{@{}llllllllll@{}}
\toprule
\begin{minipage}[b]{0.08\columnwidth}\raggedright\strut
:col-name\strut
\end{minipage} & \begin{minipage}[b]{0.11\columnwidth}\raggedright\strut
:datatype\strut
\end{minipage} & \begin{minipage}[b]{0.06\columnwidth}\raggedright\strut
:n-valid\strut
\end{minipage} & \begin{minipage}[b]{0.07\columnwidth}\raggedright\strut
:n-missing\strut
\end{minipage} & \begin{minipage}[b]{0.07\columnwidth}\raggedright\strut
:mean\strut
\end{minipage} & \begin{minipage}[b]{0.04\columnwidth}\raggedright\strut
:mode\strut
\end{minipage} & \begin{minipage}[b]{0.07\columnwidth}\raggedright\strut
:min\strut
\end{minipage} & \begin{minipage}[b]{0.07\columnwidth}\raggedright\strut
:max\strut
\end{minipage} & \begin{minipage}[b]{0.12\columnwidth}\raggedright\strut
:standard-deviation\strut
\end{minipage} & \begin{minipage}[b]{0.05\columnwidth}\raggedright\strut
:skew\strut
\end{minipage}\tabularnewline
\midrule
\endhead
\begin{minipage}[t]{0.08\columnwidth}\raggedright\strut
date\strut
\end{minipage} & \begin{minipage}[t]{0.11\columnwidth}\raggedright\strut
:packed-local-date\strut
\end{minipage} & \begin{minipage}[t]{0.06\columnwidth}\raggedright\strut
1461\strut
\end{minipage} & \begin{minipage}[t]{0.07\columnwidth}\raggedright\strut
0\strut
\end{minipage} & \begin{minipage}[t]{0.07\columnwidth}\raggedright\strut
2013-12-31\strut
\end{minipage} & \begin{minipage}[t]{0.04\columnwidth}\raggedright\strut
\strut
\end{minipage} & \begin{minipage}[t]{0.07\columnwidth}\raggedright\strut
2012-01-01\strut
\end{minipage} & \begin{minipage}[t]{0.07\columnwidth}\raggedright\strut
2015-12-31\strut
\end{minipage} & \begin{minipage}[t]{0.12\columnwidth}\raggedright\strut
\strut
\end{minipage} & \begin{minipage}[t]{0.05\columnwidth}\raggedright\strut
\strut
\end{minipage}\tabularnewline
\begin{minipage}[t]{0.08\columnwidth}\raggedright\strut
precipitation\strut
\end{minipage} & \begin{minipage}[t]{0.11\columnwidth}\raggedright\strut
:float32\strut
\end{minipage} & \begin{minipage}[t]{0.06\columnwidth}\raggedright\strut
1461\strut
\end{minipage} & \begin{minipage}[t]{0.07\columnwidth}\raggedright\strut
0\strut
\end{minipage} & \begin{minipage}[t]{0.07\columnwidth}\raggedright\strut
3.029\strut
\end{minipage} & \begin{minipage}[t]{0.04\columnwidth}\raggedright\strut
\strut
\end{minipage} & \begin{minipage}[t]{0.07\columnwidth}\raggedright\strut
0.000\strut
\end{minipage} & \begin{minipage}[t]{0.07\columnwidth}\raggedright\strut
55.90\strut
\end{minipage} & \begin{minipage}[t]{0.12\columnwidth}\raggedright\strut
6.680\strut
\end{minipage} & \begin{minipage}[t]{0.05\columnwidth}\raggedright\strut
3.506\strut
\end{minipage}\tabularnewline
\begin{minipage}[t]{0.08\columnwidth}\raggedright\strut
temp\_max\strut
\end{minipage} & \begin{minipage}[t]{0.11\columnwidth}\raggedright\strut
:float32\strut
\end{minipage} & \begin{minipage}[t]{0.06\columnwidth}\raggedright\strut
1461\strut
\end{minipage} & \begin{minipage}[t]{0.07\columnwidth}\raggedright\strut
0\strut
\end{minipage} & \begin{minipage}[t]{0.07\columnwidth}\raggedright\strut
16.44\strut
\end{minipage} & \begin{minipage}[t]{0.04\columnwidth}\raggedright\strut
\strut
\end{minipage} & \begin{minipage}[t]{0.07\columnwidth}\raggedright\strut
-1.600\strut
\end{minipage} & \begin{minipage}[t]{0.07\columnwidth}\raggedright\strut
35.60\strut
\end{minipage} & \begin{minipage}[t]{0.12\columnwidth}\raggedright\strut
7.350\strut
\end{minipage} & \begin{minipage}[t]{0.05\columnwidth}\raggedright\strut
0.2809\strut
\end{minipage}\tabularnewline
\begin{minipage}[t]{0.08\columnwidth}\raggedright\strut
temp\_min\strut
\end{minipage} & \begin{minipage}[t]{0.11\columnwidth}\raggedright\strut
:float32\strut
\end{minipage} & \begin{minipage}[t]{0.06\columnwidth}\raggedright\strut
1461\strut
\end{minipage} & \begin{minipage}[t]{0.07\columnwidth}\raggedright\strut
0\strut
\end{minipage} & \begin{minipage}[t]{0.07\columnwidth}\raggedright\strut
8.235\strut
\end{minipage} & \begin{minipage}[t]{0.04\columnwidth}\raggedright\strut
\strut
\end{minipage} & \begin{minipage}[t]{0.07\columnwidth}\raggedright\strut
-7.100\strut
\end{minipage} & \begin{minipage}[t]{0.07\columnwidth}\raggedright\strut
18.30\strut
\end{minipage} & \begin{minipage}[t]{0.12\columnwidth}\raggedright\strut
5.023\strut
\end{minipage} & \begin{minipage}[t]{0.05\columnwidth}\raggedright\strut
-0.2495\strut
\end{minipage}\tabularnewline
\begin{minipage}[t]{0.08\columnwidth}\raggedright\strut
weather\strut
\end{minipage} & \begin{minipage}[t]{0.11\columnwidth}\raggedright\strut
:string\strut
\end{minipage} & \begin{minipage}[t]{0.06\columnwidth}\raggedright\strut
1461\strut
\end{minipage} & \begin{minipage}[t]{0.07\columnwidth}\raggedright\strut
0\strut
\end{minipage} & \begin{minipage}[t]{0.07\columnwidth}\raggedright\strut
\strut
\end{minipage} & \begin{minipage}[t]{0.04\columnwidth}\raggedright\strut
sun\strut
\end{minipage} & \begin{minipage}[t]{0.07\columnwidth}\raggedright\strut
\strut
\end{minipage} & \begin{minipage}[t]{0.07\columnwidth}\raggedright\strut
\strut
\end{minipage} & \begin{minipage}[t]{0.12\columnwidth}\raggedright\strut
\strut
\end{minipage} & \begin{minipage}[t]{0.05\columnwidth}\raggedright\strut
\strut
\end{minipage}\tabularnewline
\begin{minipage}[t]{0.08\columnwidth}\raggedright\strut
wind\strut
\end{minipage} & \begin{minipage}[t]{0.11\columnwidth}\raggedright\strut
:float32\strut
\end{minipage} & \begin{minipage}[t]{0.06\columnwidth}\raggedright\strut
1461\strut
\end{minipage} & \begin{minipage}[t]{0.07\columnwidth}\raggedright\strut
0\strut
\end{minipage} & \begin{minipage}[t]{0.07\columnwidth}\raggedright\strut
3.241\strut
\end{minipage} & \begin{minipage}[t]{0.04\columnwidth}\raggedright\strut
\strut
\end{minipage} & \begin{minipage}[t]{0.07\columnwidth}\raggedright\strut
0.4000\strut
\end{minipage} & \begin{minipage}[t]{0.07\columnwidth}\raggedright\strut
9.500\strut
\end{minipage} & \begin{minipage}[t]{0.12\columnwidth}\raggedright\strut
1.438\strut
\end{minipage} & \begin{minipage}[t]{0.05\columnwidth}\raggedright\strut
0.8917\strut
\end{minipage}\tabularnewline
\bottomrule
\end{longtable}

\url{https://vega.github.io/vega-lite/examples/data/seattle-weather.csv}
:basic info {[}1 4{]}:

\begin{longtable}[]{@{}llll@{}}
\toprule
\begin{minipage}[b]{0.61\columnwidth}\raggedright\strut
:name\strut
\end{minipage} & \begin{minipage}[b]{0.11\columnwidth}\raggedright\strut
:grouped?\strut
\end{minipage} & \begin{minipage}[b]{0.07\columnwidth}\raggedright\strut
:rows\strut
\end{minipage} & \begin{minipage}[b]{0.10\columnwidth}\raggedright\strut
:columns\strut
\end{minipage}\tabularnewline
\midrule
\endhead
\begin{minipage}[t]{0.61\columnwidth}\raggedright\strut
\url{https://vega.github.io/vega-lite/examples/data/seattle-weather.csv}\strut
\end{minipage} & \begin{minipage}[t]{0.11\columnwidth}\raggedright\strut
false\strut
\end{minipage} & \begin{minipage}[t]{0.07\columnwidth}\raggedright\strut
1461\strut
\end{minipage} & \begin{minipage}[t]{0.10\columnwidth}\raggedright\strut
6\strut
\end{minipage}\tabularnewline
\bottomrule
\end{longtable}

\url{https://vega.github.io/vega-lite/examples/data/seattle-weather.csv}
:column info {[}6 4{]}:

\begin{longtable}[]{@{}llll@{}}
\toprule
:name & :size & :datatype & :categorical?\tabularnewline
\midrule
\endhead
date & 1461 & :packed-local-date &\tabularnewline
precipitation & 1461 & :float32 &\tabularnewline
temp\_max & 1461 & :float32 &\tabularnewline
temp\_min & 1461 & :float32 &\tabularnewline
wind & 1461 & :float32 &\tabularnewline
weather & 1461 & :string & true\tabularnewline
\bottomrule
\end{longtable}

\begin{center}\rule{0.5\linewidth}{0.5pt}\end{center}

Getting a dataset name

\begin{Shaded}
\begin{Highlighting}[]
\NormalTok{(api/dataset-name ds)}
\end{Highlighting}
\end{Shaded}

\begin{verbatim}
"https://vega.github.io/vega-lite/examples/data/seattle-weather.csv"
\end{verbatim}

\begin{center}\rule{0.5\linewidth}{0.5pt}\end{center}

Setting a dataset name (operation is immutable).

\begin{Shaded}
\begin{Highlighting}[]
\NormalTok{(}\KeywordTok{->>} \StringTok{"seattle-weather"}
\NormalTok{     (api/set-dataset-name ds)}
\NormalTok{     (api/dataset-name))}
\end{Highlighting}
\end{Shaded}

\begin{verbatim}
"seattle-weather"
\end{verbatim}

\paragraph{Columns and rows}\label{columns-and-rows}

Get columns and rows as sequences. \texttt{column}, \texttt{columns} and
\texttt{rows} treat grouped dataset as regular one. See \texttt{Groups}
to read more about grouped datasets.

\begin{center}\rule{0.5\linewidth}{0.5pt}\end{center}

Select column.

\begin{Shaded}
\begin{Highlighting}[]
\NormalTok{(ds }\StringTok{"wind"}\NormalTok{)}
\NormalTok{(api/column ds }\StringTok{"date"}\NormalTok{)}
\end{Highlighting}
\end{Shaded}

\begin{verbatim}
#tech.ml.dataset.column<float32>[1461]
wind
[4.700, 4.500, 2.300, 4.700, 6.100, 2.200, 2.300, 2.000, 3.400, 3.400, 5.100, 1.900, 1.300, 5.300, 3.200, 5.000, 5.600, 5.000, 1.600, 2.300, ...]
#tech.ml.dataset.column<packed-local-date>[1461]
date
[2012-01-01, 2012-01-02, 2012-01-03, 2012-01-04, 2012-01-05, 2012-01-06, 2012-01-07, 2012-01-08, 2012-01-09, 2012-01-10, 2012-01-11, 2012-01-12, 2012-01-13, 2012-01-14, 2012-01-15, 2012-01-16, 2012-01-17, 2012-01-18, 2012-01-19, 2012-01-20, ...]
\end{verbatim}

\begin{center}\rule{0.5\linewidth}{0.5pt}\end{center}

Columns as sequence

\begin{Shaded}
\begin{Highlighting}[]
\NormalTok{(}\KeywordTok{take} \DecValTok{2}\NormalTok{ (api/columns ds))}
\end{Highlighting}
\end{Shaded}

\begin{verbatim}
(#tech.ml.dataset.column<packed-local-date>[1461]
date
[2012-01-01, 2012-01-02, 2012-01-03, 2012-01-04, 2012-01-05, 2012-01-06, 2012-01-07, 2012-01-08, 2012-01-09, 2012-01-10, 2012-01-11, 2012-01-12, 2012-01-13, 2012-01-14, 2012-01-15, 2012-01-16, 2012-01-17, 2012-01-18, 2012-01-19, 2012-01-20, ...] #tech.ml.dataset.column<float32>[1461]
precipitation
[0.000, 10.90, 0.8000, 20.30, 1.300, 2.500, 0.000, 0.000, 4.300, 1.000, 0.000, 0.000, 0.000, 4.100, 5.300, 2.500, 8.100, 19.80, 15.20, 13.50, ...])
\end{verbatim}

\begin{center}\rule{0.5\linewidth}{0.5pt}\end{center}

Columns as map

\begin{Shaded}
\begin{Highlighting}[]
\NormalTok{(}\KeywordTok{keys}\NormalTok{ (api/columns ds }\AttributeTok{:as-map}\NormalTok{))}
\end{Highlighting}
\end{Shaded}

\begin{verbatim}
("date" "precipitation" "temp_max" "temp_min" "wind" "weather")
\end{verbatim}

\begin{center}\rule{0.5\linewidth}{0.5pt}\end{center}

Rows as sequence of sequences

\begin{Shaded}
\begin{Highlighting}[]
\NormalTok{(}\KeywordTok{take} \DecValTok{2}\NormalTok{ (api/rows ds))}
\end{Highlighting}
\end{Shaded}

\begin{verbatim}
([#object[java.time.LocalDate 0x4fd2177e "2012-01-01"] 0.0 12.8 5.0 4.7 "drizzle"] [#object[java.time.LocalDate 0x2eea1cca "2012-01-02"] 10.9 10.6 2.8 4.5 "rain"])
\end{verbatim}

\begin{center}\rule{0.5\linewidth}{0.5pt}\end{center}

Rows as sequence of maps

\begin{Shaded}
\begin{Highlighting}[]
\NormalTok{(clojure.pprint/pprint (}\KeywordTok{take} \DecValTok{2}\NormalTok{ (api/rows ds }\AttributeTok{:as-maps}\NormalTok{)))}
\end{Highlighting}
\end{Shaded}

\begin{verbatim}
({"date" #object[java.time.LocalDate 0x3e89a0d6 "2012-01-01"],
  "precipitation" 0.0,
  "temp_min" 5.0,
  "weather" "drizzle",
  "temp_max" 12.8,
  "wind" 4.7}
 {"date" #object[java.time.LocalDate 0xd7dd8f1 "2012-01-02"],
  "precipitation" 10.9,
  "temp_min" 2.8,
  "weather" "rain",
  "temp_max" 10.6,
  "wind" 4.5})
\end{verbatim}

\subsubsection{Group-by}\label{group-by}

Grouping by is an operation which splits dataset into subdatasets and
pack it into new special type of\ldots{} dataset. I distinguish two
types of dataset: regular dataset and grouped dataset. The latter is the
result of grouping.

Grouped dataset is annotated in by \texttt{:grouped?} meta tag and
consist following columns:

\begin{itemize}
\tightlist
\item
  \texttt{:name} - group name or structure
\item
  \texttt{:group-id} - integer assigned to the group
\item
  \texttt{:data} - groups as datasets
\end{itemize}

Almost all functions recognize type of the dataset (grouped or not) and
operate accordingly.

You can't apply reshaping or join/concat functions on grouped datasets.

\paragraph{Grouping}\label{grouping}

Grouping is done by calling \texttt{group-by} function with arguments:

\begin{itemize}
\tightlist
\item
  \texttt{ds} - dataset
\item
  \texttt{grouping-selector} - what to use for grouping
\item
  options:

  \begin{itemize}
  \tightlist
  \item
    \texttt{:result-type} - what to return:

    \begin{itemize}
    \tightlist
    \item
      \texttt{:as-dataset} (default) - return grouped dataset
    \item
      \texttt{:as-indexes} - return rows ids (row number from original
      dataset)
    \item
      \texttt{:as-map} - return map with group names as keys and
      subdataset as values
    \item
      \texttt{:as-seq} - return sequens of subdatasets
    \end{itemize}
  \item
    \texttt{:limit-columns} - list of the columns which should be
    returned during grouping by function.
  \end{itemize}
\end{itemize}

All subdatasets (groups) have set name as the group name, additionally
\texttt{group-id} is in meta.

Grouping can be done by:

\begin{itemize}
\tightlist
\item
  single column name
\item
  seq of column names
\item
  map of keys (group names) and row indexes
\item
  value returned by function taking row as map
\end{itemize}

Note: currently dataset inside dataset is printed recursively so it
renders poorly from markdown. So I will use \texttt{:as-seq} result type
to show just group names and groups.

\begin{center}\rule{0.5\linewidth}{0.5pt}\end{center}

List of columns in groupd dataset

\begin{Shaded}
\begin{Highlighting}[]
\NormalTok{(api/column-names (api/group-by DS }\AttributeTok{:V1}\NormalTok{))}
\end{Highlighting}
\end{Shaded}

\begin{verbatim}
(:name :group-id :data)
\end{verbatim}

\begin{center}\rule{0.5\linewidth}{0.5pt}\end{center}

Content of the grouped dataset

\begin{Shaded}
\begin{Highlighting}[]
\NormalTok{(api/columns (api/group-by DS }\AttributeTok{:V1}\NormalTok{) }\AttributeTok{:as-map}\NormalTok{)}
\end{Highlighting}
\end{Shaded}

\begin{verbatim}
{:name #tech.ml.dataset.column<int64>[2]
:name
[1, 2, ], :group-id #tech.ml.dataset.column<int64>[2]
:group-id
[0, 1, ], :data #tech.ml.dataset.column<object>[2]
:data
[1 [5 4]:

| :V1 | :V2 |    :V3 | :V4 |
|-----+-----+--------+-----|
|   1 |   1 | 0.5000 |   A |
|   1 |   3 |  1.500 |   C |
|   1 |   5 |  1.000 |   B |
|   1 |   7 | 0.5000 |   A |
|   1 |   9 |  1.500 |   C |
, 2 [4 4]:

| :V1 | :V2 |    :V3 | :V4 |
|-----+-----+--------+-----|
|   2 |   2 |  1.000 |   B |
|   2 |   4 | 0.5000 |   A |
|   2 |   6 |  1.500 |   C |
|   2 |   8 |  1.000 |   B |
, ]}
\end{verbatim}

\begin{center}\rule{0.5\linewidth}{0.5pt}\end{center}

Grouped dataset as map

\begin{Shaded}
\begin{Highlighting}[]
\NormalTok{(}\KeywordTok{keys}\NormalTok{ (api/group-by DS }\AttributeTok{:V1}\NormalTok{ \{}\AttributeTok{:result-type} \AttributeTok{:as-map}\NormalTok{\}))}
\end{Highlighting}
\end{Shaded}

\begin{verbatim}
(1 2)
\end{verbatim}

\begin{Shaded}
\begin{Highlighting}[]
\NormalTok{(}\KeywordTok{vals}\NormalTok{ (api/group-by DS }\AttributeTok{:V1}\NormalTok{ \{}\AttributeTok{:result-type} \AttributeTok{:as-map}\NormalTok{\}))}
\end{Highlighting}
\end{Shaded}

(1 {[}5 4{]}:

\begin{longtable}[]{@{}llll@{}}
\toprule
:V1 & :V2 & :V3 & :V4\tabularnewline
\midrule
\endhead
1 & 1 & 0.5000 & A\tabularnewline
1 & 3 & 1.500 & C\tabularnewline
1 & 5 & 1.000 & B\tabularnewline
1 & 7 & 0.5000 & A\tabularnewline
1 & 9 & 1.500 & C\tabularnewline
\bottomrule
\end{longtable}

2 {[}4 4{]}:

\begin{longtable}[]{@{}llll@{}}
\toprule
:V1 & :V2 & :V3 & :V4\tabularnewline
\midrule
\endhead
2 & 2 & 1.000 & B\tabularnewline
2 & 4 & 0.5000 & A\tabularnewline
2 & 6 & 1.500 & C\tabularnewline
2 & 8 & 1.000 & B\tabularnewline
\bottomrule
\end{longtable}

)

\begin{center}\rule{0.5\linewidth}{0.5pt}\end{center}

Group dataset as map of indexes (row ids)

\begin{Shaded}
\begin{Highlighting}[]
\NormalTok{(api/group-by DS }\AttributeTok{:V1}\NormalTok{ \{}\AttributeTok{:result-type} \AttributeTok{:as-indexes}\NormalTok{\})}
\end{Highlighting}
\end{Shaded}

\begin{verbatim}
{1 [0 2 4 6 8], 2 [1 3 5 7]}
\end{verbatim}

\begin{center}\rule{0.5\linewidth}{0.5pt}\end{center}

To get groups as sequence or a map can be done from grouped dataset
using \texttt{groups-\textgreater{}seq} and
\texttt{groups-\textgreater{}map} functions.

Groups as seq can be obtained by just accessing \texttt{:data} column.

I will use temporary dataset here.

\begin{Shaded}
\begin{Highlighting}[]
\NormalTok{(}\KeywordTok{let}\NormalTok{ [ds (}\KeywordTok{->}\NormalTok{ \{}\StringTok{"a"}\NormalTok{ [}\DecValTok{1} \DecValTok{1} \DecValTok{2} \DecValTok{2}\NormalTok{]}
              \StringTok{"b"}\NormalTok{ [}\StringTok{"a"} \StringTok{"b"} \StringTok{"c"} \StringTok{"d"}\NormalTok{]\}}
\NormalTok{             (api/dataset)}
\NormalTok{             (api/group-by }\StringTok{"a"}\NormalTok{))]}
\NormalTok{  (}\KeywordTok{seq}\NormalTok{ (ds }\AttributeTok{:data}\NormalTok{))) }\CommentTok{;; seq is not necessary but Markdown treats `:data` as command here}
\end{Highlighting}
\end{Shaded}

(1 {[}2 2{]}:

\begin{longtable}[]{@{}ll@{}}
\toprule
a & b\tabularnewline
\midrule
\endhead
1 & a\tabularnewline
1 & b\tabularnewline
\bottomrule
\end{longtable}

2 {[}2 2{]}:

\begin{longtable}[]{@{}ll@{}}
\toprule
a & b\tabularnewline
\midrule
\endhead
2 & c\tabularnewline
2 & d\tabularnewline
\bottomrule
\end{longtable}

)

\begin{Shaded}
\begin{Highlighting}[]
\NormalTok{(}\KeywordTok{->}\NormalTok{ \{}\StringTok{"a"}\NormalTok{ [}\DecValTok{1} \DecValTok{1} \DecValTok{2} \DecValTok{2}\NormalTok{]}
     \StringTok{"b"}\NormalTok{ [}\StringTok{"a"} \StringTok{"b"} \StringTok{"c"} \StringTok{"d"}\NormalTok{]\}}
\NormalTok{    (api/dataset)}
\NormalTok{    (api/group-by }\StringTok{"a"}\NormalTok{)}
\NormalTok{    (api/groups->seq))}
\end{Highlighting}
\end{Shaded}

(1 {[}2 2{]}:

\begin{longtable}[]{@{}ll@{}}
\toprule
a & b\tabularnewline
\midrule
\endhead
1 & a\tabularnewline
1 & b\tabularnewline
\bottomrule
\end{longtable}

2 {[}2 2{]}:

\begin{longtable}[]{@{}ll@{}}
\toprule
a & b\tabularnewline
\midrule
\endhead
2 & c\tabularnewline
2 & d\tabularnewline
\bottomrule
\end{longtable}

)

\begin{center}\rule{0.5\linewidth}{0.5pt}\end{center}

Groups as map

\begin{Shaded}
\begin{Highlighting}[]
\NormalTok{(}\KeywordTok{->}\NormalTok{ \{}\StringTok{"a"}\NormalTok{ [}\DecValTok{1} \DecValTok{1} \DecValTok{2} \DecValTok{2}\NormalTok{]}
     \StringTok{"b"}\NormalTok{ [}\StringTok{"a"} \StringTok{"b"} \StringTok{"c"} \StringTok{"d"}\NormalTok{]\}}
\NormalTok{    (api/dataset)}
\NormalTok{    (api/group-by }\StringTok{"a"}\NormalTok{)}
\NormalTok{    (api/groups->map))}
\end{Highlighting}
\end{Shaded}

\{1 1 {[}2 2{]}:

\begin{longtable}[]{@{}ll@{}}
\toprule
a & b\tabularnewline
\midrule
\endhead
1 & a\tabularnewline
1 & b\tabularnewline
\bottomrule
\end{longtable}

, 2 2 {[}2 2{]}:

\begin{longtable}[]{@{}ll@{}}
\toprule
a & b\tabularnewline
\midrule
\endhead
2 & c\tabularnewline
2 & d\tabularnewline
\bottomrule
\end{longtable}

\}

\begin{center}\rule{0.5\linewidth}{0.5pt}\end{center}

Grouping by more than one column. You can see that group names are maps.
When ungrouping is done these maps are used to restore column names.

\begin{Shaded}
\begin{Highlighting}[]
\NormalTok{(api/group-by DS [}\AttributeTok{:V1} \AttributeTok{:V3}\NormalTok{] \{}\AttributeTok{:result-type} \AttributeTok{:as-seq}\NormalTok{\})}
\end{Highlighting}
\end{Shaded}

(\{:V3 1.0, :V1 1\} {[}1 4{]}:

\begin{longtable}[]{@{}llll@{}}
\toprule
:V1 & :V2 & :V3 & :V4\tabularnewline
\midrule
\endhead
1 & 5 & 1.000 & B\tabularnewline
\bottomrule
\end{longtable}

\{:V3 0.5, :V1 1\} {[}2 4{]}:

\begin{longtable}[]{@{}llll@{}}
\toprule
:V1 & :V2 & :V3 & :V4\tabularnewline
\midrule
\endhead
1 & 1 & 0.5000 & A\tabularnewline
1 & 7 & 0.5000 & A\tabularnewline
\bottomrule
\end{longtable}

\{:V3 0.5, :V1 2\} {[}1 4{]}:

\begin{longtable}[]{@{}llll@{}}
\toprule
:V1 & :V2 & :V3 & :V4\tabularnewline
\midrule
\endhead
2 & 4 & 0.5000 & A\tabularnewline
\bottomrule
\end{longtable}

\{:V3 1.0, :V1 2\} {[}2 4{]}:

\begin{longtable}[]{@{}llll@{}}
\toprule
:V1 & :V2 & :V3 & :V4\tabularnewline
\midrule
\endhead
2 & 2 & 1.000 & B\tabularnewline
2 & 8 & 1.000 & B\tabularnewline
\bottomrule
\end{longtable}

\{:V3 1.5, :V1 1\} {[}2 4{]}:

\begin{longtable}[]{@{}llll@{}}
\toprule
:V1 & :V2 & :V3 & :V4\tabularnewline
\midrule
\endhead
1 & 3 & 1.500 & C\tabularnewline
1 & 9 & 1.500 & C\tabularnewline
\bottomrule
\end{longtable}

\{:V3 1.5, :V1 2\} {[}1 4{]}:

\begin{longtable}[]{@{}llll@{}}
\toprule
:V1 & :V2 & :V3 & :V4\tabularnewline
\midrule
\endhead
2 & 6 & 1.500 & C\tabularnewline
\bottomrule
\end{longtable}

)

\begin{center}\rule{0.5\linewidth}{0.5pt}\end{center}

Grouping can be done by providing just row indexes. This way you can
assign the same row to more than one group.

\begin{Shaded}
\begin{Highlighting}[]
\NormalTok{(api/group-by DS \{}\StringTok{"group-a"}\NormalTok{ [}\DecValTok{1} \DecValTok{2} \DecValTok{1} \DecValTok{2}\NormalTok{]}
                  \StringTok{"group-b"}\NormalTok{ [}\DecValTok{5} \DecValTok{5} \DecValTok{5} \DecValTok{1}\NormalTok{]\} \{}\AttributeTok{:result-type} \AttributeTok{:as-seq}\NormalTok{\})}
\end{Highlighting}
\end{Shaded}

(group-a {[}4 4{]}:

\begin{longtable}[]{@{}llll@{}}
\toprule
:V1 & :V2 & :V3 & :V4\tabularnewline
\midrule
\endhead
2 & 2 & 1.000 & B\tabularnewline
1 & 3 & 1.500 & C\tabularnewline
2 & 2 & 1.000 & B\tabularnewline
1 & 3 & 1.500 & C\tabularnewline
\bottomrule
\end{longtable}

group-b {[}4 4{]}:

\begin{longtable}[]{@{}llll@{}}
\toprule
:V1 & :V2 & :V3 & :V4\tabularnewline
\midrule
\endhead
2 & 6 & 1.500 & C\tabularnewline
2 & 6 & 1.500 & C\tabularnewline
2 & 6 & 1.500 & C\tabularnewline
2 & 2 & 1.000 & B\tabularnewline
\bottomrule
\end{longtable}

)

\begin{center}\rule{0.5\linewidth}{0.5pt}\end{center}

You can group by a result of gruping function which gets row as map and
should return group name. When map is used as a group name, ungrouping
restore original column names.

\begin{Shaded}
\begin{Highlighting}[]
\NormalTok{(api/group-by DS (}\KeywordTok{fn}\NormalTok{ [row] (}\KeywordTok{*}\NormalTok{ (}\AttributeTok{:V1}\NormalTok{ row)}
\NormalTok{                             (}\AttributeTok{:V3}\NormalTok{ row))) \{}\AttributeTok{:result-type} \AttributeTok{:as-seq}\NormalTok{\})}
\end{Highlighting}
\end{Shaded}

(1.0 {[}2 4{]}:

\begin{longtable}[]{@{}llll@{}}
\toprule
:V1 & :V2 & :V3 & :V4\tabularnewline
\midrule
\endhead
2 & 4 & 0.5000 & A\tabularnewline
1 & 5 & 1.000 & B\tabularnewline
\bottomrule
\end{longtable}

2.0 {[}2 4{]}:

\begin{longtable}[]{@{}llll@{}}
\toprule
:V1 & :V2 & :V3 & :V4\tabularnewline
\midrule
\endhead
2 & 2 & 1.000 & B\tabularnewline
2 & 8 & 1.000 & B\tabularnewline
\bottomrule
\end{longtable}

0.5 {[}2 4{]}:

\begin{longtable}[]{@{}llll@{}}
\toprule
:V1 & :V2 & :V3 & :V4\tabularnewline
\midrule
\endhead
1 & 1 & 0.5000 & A\tabularnewline
1 & 7 & 0.5000 & A\tabularnewline
\bottomrule
\end{longtable}

3.0 {[}1 4{]}:

\begin{longtable}[]{@{}llll@{}}
\toprule
:V1 & :V2 & :V3 & :V4\tabularnewline
\midrule
\endhead
2 & 6 & 1.500 & C\tabularnewline
\bottomrule
\end{longtable}

1.5 {[}2 4{]}:

\begin{longtable}[]{@{}llll@{}}
\toprule
:V1 & :V2 & :V3 & :V4\tabularnewline
\midrule
\endhead
1 & 3 & 1.500 & C\tabularnewline
1 & 9 & 1.500 & C\tabularnewline
\bottomrule
\end{longtable}

)

\begin{center}\rule{0.5\linewidth}{0.5pt}\end{center}

You can use any predicate on column to split dataset into two groups.

\begin{Shaded}
\begin{Highlighting}[]
\NormalTok{(api/group-by DS (}\KeywordTok{comp}\NormalTok{ #(}\KeywordTok{<} \VariableTok{%} \FloatTok{1.0}\NormalTok{) }\AttributeTok{:V3}\NormalTok{) \{}\AttributeTok{:result-type} \AttributeTok{:as-seq}\NormalTok{\})}
\end{Highlighting}
\end{Shaded}

(false {[}6 4{]}:

\begin{longtable}[]{@{}llll@{}}
\toprule
:V1 & :V2 & :V3 & :V4\tabularnewline
\midrule
\endhead
2 & 2 & 1.000 & B\tabularnewline
1 & 3 & 1.500 & C\tabularnewline
1 & 5 & 1.000 & B\tabularnewline
2 & 6 & 1.500 & C\tabularnewline
2 & 8 & 1.000 & B\tabularnewline
1 & 9 & 1.500 & C\tabularnewline
\bottomrule
\end{longtable}

true {[}3 4{]}:

\begin{longtable}[]{@{}llll@{}}
\toprule
:V1 & :V2 & :V3 & :V4\tabularnewline
\midrule
\endhead
1 & 1 & 0.5000 & A\tabularnewline
2 & 4 & 0.5000 & A\tabularnewline
1 & 7 & 0.5000 & A\tabularnewline
\bottomrule
\end{longtable}

)

\begin{center}\rule{0.5\linewidth}{0.5pt}\end{center}

\texttt{juxt} is also helpful

\begin{Shaded}
\begin{Highlighting}[]
\NormalTok{(api/group-by DS (}\KeywordTok{juxt} \AttributeTok{:V1} \AttributeTok{:V3}\NormalTok{) \{}\AttributeTok{:result-type} \AttributeTok{:as-seq}\NormalTok{\})}
\end{Highlighting}
\end{Shaded}

({[}1 1.0{]} {[}1 4{]}:

\begin{longtable}[]{@{}llll@{}}
\toprule
:V1 & :V2 & :V3 & :V4\tabularnewline
\midrule
\endhead
1 & 5 & 1.000 & B\tabularnewline
\bottomrule
\end{longtable}

{[}1 0.5{]} {[}2 4{]}:

\begin{longtable}[]{@{}llll@{}}
\toprule
:V1 & :V2 & :V3 & :V4\tabularnewline
\midrule
\endhead
1 & 1 & 0.5000 & A\tabularnewline
1 & 7 & 0.5000 & A\tabularnewline
\bottomrule
\end{longtable}

{[}2 1.5{]} {[}1 4{]}:

\begin{longtable}[]{@{}llll@{}}
\toprule
:V1 & :V2 & :V3 & :V4\tabularnewline
\midrule
\endhead
2 & 6 & 1.500 & C\tabularnewline
\bottomrule
\end{longtable}

{[}1 1.5{]} {[}2 4{]}:

\begin{longtable}[]{@{}llll@{}}
\toprule
:V1 & :V2 & :V3 & :V4\tabularnewline
\midrule
\endhead
1 & 3 & 1.500 & C\tabularnewline
1 & 9 & 1.500 & C\tabularnewline
\bottomrule
\end{longtable}

{[}2 0.5{]} {[}1 4{]}:

\begin{longtable}[]{@{}llll@{}}
\toprule
:V1 & :V2 & :V3 & :V4\tabularnewline
\midrule
\endhead
2 & 4 & 0.5000 & A\tabularnewline
\bottomrule
\end{longtable}

{[}2 1.0{]} {[}2 4{]}:

\begin{longtable}[]{@{}llll@{}}
\toprule
:V1 & :V2 & :V3 & :V4\tabularnewline
\midrule
\endhead
2 & 2 & 1.000 & B\tabularnewline
2 & 8 & 1.000 & B\tabularnewline
\bottomrule
\end{longtable}

)

\begin{center}\rule{0.5\linewidth}{0.5pt}\end{center}

\texttt{tech.ml.dataset} provides an option to limit columns which are
passed to grouping functions. It's done for performance purposes.

\begin{Shaded}
\begin{Highlighting}[]
\NormalTok{(api/group-by DS }\KeywordTok{identity}\NormalTok{ \{}\AttributeTok{:result-type} \AttributeTok{:as-seq}
                           \AttributeTok{:limit-columns}\NormalTok{ [}\AttributeTok{:V1}\NormalTok{]\})}
\end{Highlighting}
\end{Shaded}

(\{:V1 1\} {[}5 4{]}:

\begin{longtable}[]{@{}llll@{}}
\toprule
:V1 & :V2 & :V3 & :V4\tabularnewline
\midrule
\endhead
1 & 1 & 0.5000 & A\tabularnewline
1 & 3 & 1.500 & C\tabularnewline
1 & 5 & 1.000 & B\tabularnewline
1 & 7 & 0.5000 & A\tabularnewline
1 & 9 & 1.500 & C\tabularnewline
\bottomrule
\end{longtable}

\{:V1 2\} {[}4 4{]}:

\begin{longtable}[]{@{}llll@{}}
\toprule
:V1 & :V2 & :V3 & :V4\tabularnewline
\midrule
\endhead
2 & 2 & 1.000 & B\tabularnewline
2 & 4 & 0.5000 & A\tabularnewline
2 & 6 & 1.500 & C\tabularnewline
2 & 8 & 1.000 & B\tabularnewline
\bottomrule
\end{longtable}

)

\paragraph{Ungrouping}\label{ungrouping}

Ungrouping simply concats all the groups into the dataset. Following
options are possible

\begin{itemize}
\tightlist
\item
  \texttt{:order?} - order groups according to the group name ascending
  order. Default: \texttt{false}
\item
  \texttt{:add-group-as-column} - should group name become a column? If
  yes column is created with provided name (or \texttt{:\$group-name} if
  argument is \texttt{true}). Default: \texttt{nil}.
\item
  \texttt{:add-group-id-as-column} - should group id become a column? If
  yes column is created with provided name (or \texttt{:\$group-id} if
  argument is \texttt{true}). Default: \texttt{nil}.
\item
  \texttt{:dataset-name} - to name resulting dataset. Default:
  \texttt{nil} (\_unnamed)
\end{itemize}

If group name is a map, it will be splitted into separate columns. Be
sure that groups (subdatasets) doesn't contain the same columns already.

If group name is a vector, it will be splitted into separate columns. If
you want to name them, set vector of target column names as
\texttt{:add-group-as-column} argument.

After ungrouping, order of the rows is kept within the groups but groups
are ordered according to the internal storage.

\begin{center}\rule{0.5\linewidth}{0.5pt}\end{center}

Grouping and ungrouping.

\begin{Shaded}
\begin{Highlighting}[]
\NormalTok{(}\KeywordTok{->}\NormalTok{ DS}
\NormalTok{    (api/group-by }\AttributeTok{:V3}\NormalTok{)}
\NormalTok{    (api/ungroup))}
\end{Highlighting}
\end{Shaded}

\_unnamed {[}9 4{]}:

\begin{longtable}[]{@{}llll@{}}
\toprule
:V1 & :V2 & :V3 & :V4\tabularnewline
\midrule
\endhead
2 & 2 & 1.000 & B\tabularnewline
1 & 5 & 1.000 & B\tabularnewline
2 & 8 & 1.000 & B\tabularnewline
1 & 1 & 0.5000 & A\tabularnewline
2 & 4 & 0.5000 & A\tabularnewline
1 & 7 & 0.5000 & A\tabularnewline
1 & 3 & 1.500 & C\tabularnewline
2 & 6 & 1.500 & C\tabularnewline
1 & 9 & 1.500 & C\tabularnewline
\bottomrule
\end{longtable}

\begin{center}\rule{0.5\linewidth}{0.5pt}\end{center}

Groups sorted by group name and named.

\begin{Shaded}
\begin{Highlighting}[]
\NormalTok{(}\KeywordTok{->}\NormalTok{ DS}
\NormalTok{    (api/group-by }\AttributeTok{:V3}\NormalTok{)}
\NormalTok{    (api/ungroup \{}\AttributeTok{:order}\NormalTok{? }\VariableTok{true}
                  \AttributeTok{:dataset-name} \StringTok{"Ordered by V3"}\NormalTok{\}))}
\end{Highlighting}
\end{Shaded}

Ordered by V3 {[}9 4{]}:

\begin{longtable}[]{@{}llll@{}}
\toprule
:V1 & :V2 & :V3 & :V4\tabularnewline
\midrule
\endhead
1 & 1 & 0.5000 & A\tabularnewline
2 & 4 & 0.5000 & A\tabularnewline
1 & 7 & 0.5000 & A\tabularnewline
2 & 2 & 1.000 & B\tabularnewline
1 & 5 & 1.000 & B\tabularnewline
2 & 8 & 1.000 & B\tabularnewline
1 & 3 & 1.500 & C\tabularnewline
2 & 6 & 1.500 & C\tabularnewline
1 & 9 & 1.500 & C\tabularnewline
\bottomrule
\end{longtable}

\begin{center}\rule{0.5\linewidth}{0.5pt}\end{center}

Let's add group name and id as additional columns

\begin{Shaded}
\begin{Highlighting}[]
\NormalTok{(}\KeywordTok{->}\NormalTok{ DS}
\NormalTok{    (api/group-by (}\KeywordTok{comp}\NormalTok{ #(}\KeywordTok{<} \VariableTok{%} \DecValTok{4}\NormalTok{) }\AttributeTok{:V2}\NormalTok{))}
\NormalTok{    (api/ungroup \{}\AttributeTok{:add-group-as-column} \VariableTok{true}
                  \AttributeTok{:add-group-id-as-column} \VariableTok{true}\NormalTok{\}))}
\end{Highlighting}
\end{Shaded}

\_unnamed {[}9 6{]}:

\begin{longtable}[]{@{}llllll@{}}
\toprule
:\$group-name & :\$group-id & :V1 & :V2 & :V3 & :V4\tabularnewline
\midrule
\endhead
false & 0 & 2 & 4 & 0.5000 & A\tabularnewline
false & 0 & 1 & 5 & 1.000 & B\tabularnewline
false & 0 & 2 & 6 & 1.500 & C\tabularnewline
false & 0 & 1 & 7 & 0.5000 & A\tabularnewline
false & 0 & 2 & 8 & 1.000 & B\tabularnewline
false & 0 & 1 & 9 & 1.500 & C\tabularnewline
true & 1 & 1 & 1 & 0.5000 & A\tabularnewline
true & 1 & 2 & 2 & 1.000 & B\tabularnewline
true & 1 & 1 & 3 & 1.500 & C\tabularnewline
\bottomrule
\end{longtable}

\begin{center}\rule{0.5\linewidth}{0.5pt}\end{center}

Let's assign different column names

\begin{Shaded}
\begin{Highlighting}[]
\NormalTok{(}\KeywordTok{->}\NormalTok{ DS}
\NormalTok{    (api/group-by (}\KeywordTok{comp}\NormalTok{ #(}\KeywordTok{<} \VariableTok{%} \DecValTok{4}\NormalTok{) }\AttributeTok{:V2}\NormalTok{))}
\NormalTok{    (api/ungroup \{}\AttributeTok{:add-group-as-column} \StringTok{"Is V2 less than 4?"}
                  \AttributeTok{:add-group-id-as-column} \StringTok{"group id"}\NormalTok{\}))}
\end{Highlighting}
\end{Shaded}

\_unnamed {[}9 6{]}:

\begin{longtable}[]{@{}llllll@{}}
\toprule
Is V2 less than 4? & group id & :V1 & :V2 & :V3 & :V4\tabularnewline
\midrule
\endhead
false & 0 & 2 & 4 & 0.5000 & A\tabularnewline
false & 0 & 1 & 5 & 1.000 & B\tabularnewline
false & 0 & 2 & 6 & 1.500 & C\tabularnewline
false & 0 & 1 & 7 & 0.5000 & A\tabularnewline
false & 0 & 2 & 8 & 1.000 & B\tabularnewline
false & 0 & 1 & 9 & 1.500 & C\tabularnewline
true & 1 & 1 & 1 & 0.5000 & A\tabularnewline
true & 1 & 2 & 2 & 1.000 & B\tabularnewline
true & 1 & 1 & 3 & 1.500 & C\tabularnewline
\bottomrule
\end{longtable}

\begin{center}\rule{0.5\linewidth}{0.5pt}\end{center}

If we group by map, we can automatically create new columns out of group
names.

\begin{Shaded}
\begin{Highlighting}[]
\NormalTok{(}\KeywordTok{->}\NormalTok{ DS}
\NormalTok{    (api/group-by (}\KeywordTok{fn}\NormalTok{ [row] \{}\StringTok{"V1 and V3 multiplied"}\NormalTok{ (}\KeywordTok{*}\NormalTok{ (}\AttributeTok{:V1}\NormalTok{ row)}
\NormalTok{                                                      (}\AttributeTok{:V3}\NormalTok{ row))}
                            \StringTok{"V4 as string"}\NormalTok{ (}\KeywordTok{str}\NormalTok{ (}\AttributeTok{:V4}\NormalTok{ row))\}))}
\NormalTok{    (api/ungroup \{}\AttributeTok{:add-group-as-column} \VariableTok{true}\NormalTok{\}))}
\end{Highlighting}
\end{Shaded}

\_unnamed {[}9 6{]}:

\begin{longtable}[]{@{}llllll@{}}
\toprule
V1 and V3 multiplied & V4 as string & :V1 & :V2 & :V3 &
:V4\tabularnewline
\midrule
\endhead
3.000 & C & 2 & 6 & 1.500 & C\tabularnewline
1.500 & C & 1 & 3 & 1.500 & C\tabularnewline
1.500 & C & 1 & 9 & 1.500 & C\tabularnewline
1.000 & A & 2 & 4 & 0.5000 & A\tabularnewline
0.5000 & A & 1 & 1 & 0.5000 & A\tabularnewline
0.5000 & A & 1 & 7 & 0.5000 & A\tabularnewline
1.000 & B & 1 & 5 & 1.000 & B\tabularnewline
2.000 & B & 2 & 2 & 1.000 & B\tabularnewline
2.000 & B & 2 & 8 & 1.000 & B\tabularnewline
\bottomrule
\end{longtable}

\begin{center}\rule{0.5\linewidth}{0.5pt}\end{center}

We can add group names without separation

\begin{Shaded}
\begin{Highlighting}[]
\NormalTok{(}\KeywordTok{->}\NormalTok{ DS}
\NormalTok{    (api/group-by (}\KeywordTok{fn}\NormalTok{ [row] \{}\StringTok{"V1 and V3 multiplied"}\NormalTok{ (}\KeywordTok{*}\NormalTok{ (}\AttributeTok{:V1}\NormalTok{ row)}
\NormalTok{                                                      (}\AttributeTok{:V3}\NormalTok{ row))}
                            \StringTok{"V4 as string"}\NormalTok{ (}\KeywordTok{str}\NormalTok{ (}\AttributeTok{:V4}\NormalTok{ row))\}))}
\NormalTok{    (api/ungroup \{}\AttributeTok{:add-group-as-column} \StringTok{"just map"}
                  \AttributeTok{:separate}\NormalTok{? }\VariableTok{false}\NormalTok{\}))}
\end{Highlighting}
\end{Shaded}

\_unnamed {[}9 5{]}:

\begin{longtable}[]{@{}lllll@{}}
\toprule
just map & :V1 & :V2 & :V3 & :V4\tabularnewline
\midrule
\endhead
\{``V1 and V3 multiplied'' 3.0, ``V4 as string'' ``C''\} & 2 & 6 & 1.500
& C\tabularnewline
\{``V1 and V3 multiplied'' 1.5, ``V4 as string'' ``C''\} & 1 & 3 & 1.500
& C\tabularnewline
\{``V1 and V3 multiplied'' 1.5, ``V4 as string'' ``C''\} & 1 & 9 & 1.500
& C\tabularnewline
\{``V1 and V3 multiplied'' 1.0, ``V4 as string'' ``A''\} & 2 & 4 &
0.5000 & A\tabularnewline
\{``V1 and V3 multiplied'' 0.5, ``V4 as string'' ``A''\} & 1 & 1 &
0.5000 & A\tabularnewline
\{``V1 and V3 multiplied'' 0.5, ``V4 as string'' ``A''\} & 1 & 7 &
0.5000 & A\tabularnewline
\{``V1 and V3 multiplied'' 1.0, ``V4 as string'' ``B''\} & 1 & 5 & 1.000
& B\tabularnewline
\{``V1 and V3 multiplied'' 2.0, ``V4 as string'' ``B''\} & 2 & 2 & 1.000
& B\tabularnewline
\{``V1 and V3 multiplied'' 2.0, ``V4 as string'' ``B''\} & 2 & 8 & 1.000
& B\tabularnewline
\bottomrule
\end{longtable}

\begin{center}\rule{0.5\linewidth}{0.5pt}\end{center}

The same applies to group names as sequences

\begin{Shaded}
\begin{Highlighting}[]
\NormalTok{(}\KeywordTok{->}\NormalTok{ DS}
\NormalTok{    (api/group-by (}\KeywordTok{juxt} \AttributeTok{:V1} \AttributeTok{:V3}\NormalTok{))}
\NormalTok{    (api/ungroup \{}\AttributeTok{:add-group-as-column} \StringTok{"abc"}\NormalTok{\}))}
\end{Highlighting}
\end{Shaded}

\_unnamed {[}9 6{]}:

\begin{longtable}[]{@{}llllll@{}}
\toprule
:abc-0 & :abc-1 & :V1 & :V2 & :V3 & :V4\tabularnewline
\midrule
\endhead
1 & 1.000 & 1 & 5 & 1.000 & B\tabularnewline
1 & 0.5000 & 1 & 1 & 0.5000 & A\tabularnewline
1 & 0.5000 & 1 & 7 & 0.5000 & A\tabularnewline
2 & 1.500 & 2 & 6 & 1.500 & C\tabularnewline
1 & 1.500 & 1 & 3 & 1.500 & C\tabularnewline
1 & 1.500 & 1 & 9 & 1.500 & C\tabularnewline
2 & 0.5000 & 2 & 4 & 0.5000 & A\tabularnewline
2 & 1.000 & 2 & 2 & 1.000 & B\tabularnewline
2 & 1.000 & 2 & 8 & 1.000 & B\tabularnewline
\bottomrule
\end{longtable}

\begin{center}\rule{0.5\linewidth}{0.5pt}\end{center}

Let's provide column names

\begin{Shaded}
\begin{Highlighting}[]
\NormalTok{(}\KeywordTok{->}\NormalTok{ DS}
\NormalTok{    (api/group-by (}\KeywordTok{juxt} \AttributeTok{:V1} \AttributeTok{:V3}\NormalTok{))}
\NormalTok{    (api/ungroup \{}\AttributeTok{:add-group-as-column}\NormalTok{ [}\StringTok{"v1"} \StringTok{"v3"}\NormalTok{]\}))}
\end{Highlighting}
\end{Shaded}

\_unnamed {[}9 6{]}:

\begin{longtable}[]{@{}llllll@{}}
\toprule
v1 & v3 & :V1 & :V2 & :V3 & :V4\tabularnewline
\midrule
\endhead
1 & 1.000 & 1 & 5 & 1.000 & B\tabularnewline
1 & 0.5000 & 1 & 1 & 0.5000 & A\tabularnewline
1 & 0.5000 & 1 & 7 & 0.5000 & A\tabularnewline
2 & 1.500 & 2 & 6 & 1.500 & C\tabularnewline
1 & 1.500 & 1 & 3 & 1.500 & C\tabularnewline
1 & 1.500 & 1 & 9 & 1.500 & C\tabularnewline
2 & 0.5000 & 2 & 4 & 0.5000 & A\tabularnewline
2 & 1.000 & 2 & 2 & 1.000 & B\tabularnewline
2 & 1.000 & 2 & 8 & 1.000 & B\tabularnewline
\bottomrule
\end{longtable}

\begin{center}\rule{0.5\linewidth}{0.5pt}\end{center}

Also we can supress separation

\begin{Shaded}
\begin{Highlighting}[]
\NormalTok{(}\KeywordTok{->}\NormalTok{ DS}
\NormalTok{    (api/group-by (}\KeywordTok{juxt} \AttributeTok{:V1} \AttributeTok{:V3}\NormalTok{))}
\NormalTok{    (api/ungroup \{}\AttributeTok{:separate}\NormalTok{? }\VariableTok{false}
                  \AttributeTok{:add-group-as-column} \VariableTok{true}\NormalTok{\}))}
\CommentTok{;; => _unnamed [9 5]:}
\end{Highlighting}
\end{Shaded}

\_unnamed {[}9 5{]}:

\begin{longtable}[]{@{}lllll@{}}
\toprule
:\$group-name & :V1 & :V2 & :V3 & :V4\tabularnewline
\midrule
\endhead
{[}1 1.0{]} & 1 & 5 & 1.000 & B\tabularnewline
{[}1 0.5{]} & 1 & 1 & 0.5000 & A\tabularnewline
{[}1 0.5{]} & 1 & 7 & 0.5000 & A\tabularnewline
{[}2 1.5{]} & 2 & 6 & 1.500 & C\tabularnewline
{[}1 1.5{]} & 1 & 3 & 1.500 & C\tabularnewline
{[}1 1.5{]} & 1 & 9 & 1.500 & C\tabularnewline
{[}2 0.5{]} & 2 & 4 & 0.5000 & A\tabularnewline
{[}2 1.0{]} & 2 & 2 & 1.000 & B\tabularnewline
{[}2 1.0{]} & 2 & 8 & 1.000 & B\tabularnewline
\bottomrule
\end{longtable}

\paragraph{Other functions}\label{other-functions}

To check if dataset is grouped or not just use \texttt{grouped?}
function.

\begin{Shaded}
\begin{Highlighting}[]
\NormalTok{(api/grouped? DS)}
\end{Highlighting}
\end{Shaded}

\begin{verbatim}
nil
\end{verbatim}

\begin{Shaded}
\begin{Highlighting}[]
\NormalTok{(api/grouped? (api/group-by DS }\AttributeTok{:V1}\NormalTok{))}
\end{Highlighting}
\end{Shaded}

\begin{verbatim}
true
\end{verbatim}

\begin{center}\rule{0.5\linewidth}{0.5pt}\end{center}

If you want to remove grouping annotation (to make all the functions
work as with regular dataset) you can use \texttt{unmark-group} or
\texttt{as-regular-dataset} (alias) functions.

It can be important when you want to remove some groups (rows) from
grouped dataset using \texttt{drop-rows} or something like that.

\begin{Shaded}
\begin{Highlighting}[]
\NormalTok{(}\KeywordTok{->}\NormalTok{ DS}
\NormalTok{    (api/group-by }\AttributeTok{:V1}\NormalTok{)}
\NormalTok{    (api/as-regular-dataset)}
\NormalTok{    (api/grouped?))}
\end{Highlighting}
\end{Shaded}

\begin{verbatim}
nil
\end{verbatim}

\begin{center}\rule{0.5\linewidth}{0.5pt}\end{center}

This is considered internal.

If you want to implement your own mapping function on grouped dataset
you can call \texttt{process-group-data} and pass function operating on
datasets. Result should be a dataset to have ungrouping working.

\begin{Shaded}
\begin{Highlighting}[]
\NormalTok{(}\KeywordTok{->}\NormalTok{ DS}
\NormalTok{    (api/group-by }\AttributeTok{:V1}\NormalTok{)}
\NormalTok{    (api/process-group-data #(}\KeywordTok{str} \StringTok{"Shape: "}\NormalTok{ (}\KeywordTok{vector}\NormalTok{ (api/row-count }\VariableTok\NormalTok{))))}
\NormalTok{    (api/as-regular-dataset))}
\end{Highlighting}
\end{Shaded}

\_unnamed {[}2 3{]}:

\begin{longtable}[]{@{}lll@{}}
\toprule
:name & :group-id & :data\tabularnewline
\midrule
\endhead
1 & 0 & Shape: {[}5 4{]}\tabularnewline
2 & 1 & Shape: {[}4 4{]}\tabularnewline
\bottomrule
\end{longtable}

\subsubsection{Columns}\label{columns}

Column is a special \texttt{tech.ml.dataset} structure based on
\texttt{tech.ml.datatype} library. For our purposes we cat treat columns
as typed and named sequence bound to particular dataset.

Type of the data is inferred from a sequence during column creation.

\paragraph{Names}\label{names}

To select dataset columns or column names \texttt{columns-selector} is
used. \texttt{columns-selector} can be one of the following:

\begin{itemize}
\tightlist
\item
  \texttt{:all} keyword - selects all columns
\item
  column name - for single column
\item
  sequence of column names - for collection of columns
\item
  regex - to apply pattern on column names or datatype
\item
  filter predicate - to filter column names or datatype
\end{itemize}

Column name can be anything.

\texttt{column-names} function returns names according to
\texttt{columns-selector} and optional \texttt{meta-filed}.
\texttt{meta-field} is one of the following:

\begin{itemize}
\tightlist
\item
  \texttt{:name} (default) - to operate on column names
\item
  \texttt{:datatype} - to operated on column types
\item
  \texttt{:all} - if you want to process all metadata
\end{itemize}

\begin{center}\rule{0.5\linewidth}{0.5pt}\end{center}

To select all column names you can use \texttt{column-names} function.

\begin{Shaded}
\begin{Highlighting}[]
\NormalTok{(api/column-names DS)}
\end{Highlighting}
\end{Shaded}

\begin{verbatim}
(:V1 :V2 :V3 :V4)
\end{verbatim}

or

\begin{Shaded}
\begin{Highlighting}[]
\NormalTok{(api/column-names DS }\AttributeTok{:all}\NormalTok{)}
\end{Highlighting}
\end{Shaded}

\begin{verbatim}
(:V1 :V2 :V3 :V4)
\end{verbatim}

In case you want to select column which has name \texttt{:all} (or is
sequence or map), put it into a vector. Below code returns empty
sequence since there is no such column in the dataset.

\begin{Shaded}
\begin{Highlighting}[]
\NormalTok{(api/column-names DS [}\AttributeTok{:all}\NormalTok{])}
\end{Highlighting}
\end{Shaded}

\begin{verbatim}
()
\end{verbatim}

\begin{center}\rule{0.5\linewidth}{0.5pt}\end{center}

Obviously selecting single name returns it's name if available

\begin{Shaded}
\begin{Highlighting}[]
\NormalTok{(api/column-names DS }\AttributeTok{:V1}\NormalTok{)}
\NormalTok{(api/column-names DS }\StringTok{"no such column"}\NormalTok{)}
\end{Highlighting}
\end{Shaded}

\begin{verbatim}
(:V1)
()
\end{verbatim}

\begin{center}\rule{0.5\linewidth}{0.5pt}\end{center}

Select sequence of column names.

\begin{Shaded}
\begin{Highlighting}[]
\NormalTok{(api/column-names DS [}\AttributeTok{:V1} \StringTok{"V2"} \AttributeTok{:V3} \AttributeTok{:V4} \AttributeTok{:V5}\NormalTok{])}
\end{Highlighting}
\end{Shaded}

\begin{verbatim}
(:V1 :V3 :V4)
\end{verbatim}

\begin{center}\rule{0.5\linewidth}{0.5pt}\end{center}

Select names based on regex, columns ends with \texttt{1} or \texttt{4}

\begin{Shaded}
\begin{Highlighting}[]
\NormalTok{(api/column-names DS }\SpecialStringTok{#".*[14]"}\NormalTok{)}
\end{Highlighting}
\end{Shaded}

\begin{verbatim}
(:V1 :V4)
\end{verbatim}

\begin{center}\rule{0.5\linewidth}{0.5pt}\end{center}

Select names based on regex operating on type of the column (to check
what are the column types, call \texttt{(api/info\ DS\ :columns)}. Here
we want to get integer columns only.

\begin{Shaded}
\begin{Highlighting}[]
\NormalTok{(api/column-names DS }\SpecialStringTok{#"^:int.*"} \AttributeTok{:datatype}\NormalTok{)}
\end{Highlighting}
\end{Shaded}

\begin{verbatim}
(:V1 :V2)
\end{verbatim}

\begin{center}\rule{0.5\linewidth}{0.5pt}\end{center}

And finally we can use predicate to select names. Let's select double
precision columns.

\begin{Shaded}
\begin{Highlighting}[]
\NormalTok{(api/column-names DS #(}\KeywordTok{=} \AttributeTok{:float64} \VariableTok{%}\NormalTok{) }\AttributeTok{:datatype}\NormalTok{)}
\end{Highlighting}
\end{Shaded}

\begin{verbatim}
(:V3)
\end{verbatim}

\begin{center}\rule{0.5\linewidth}{0.5pt}\end{center}

If you want to select all columns but given, use \texttt{complement}
function. Works only on a predicate.

\begin{Shaded}
\begin{Highlighting}[]
\NormalTok{(api/column-names DS (}\KeywordTok{complement}\NormalTok{ #\{}\AttributeTok{:V1}\NormalTok{\}))}
\NormalTok{(api/column-names DS (}\KeywordTok{complement}\NormalTok{ #(}\KeywordTok{=} \AttributeTok{:float64} \VariableTok{%}\NormalTok{)) }\AttributeTok{:datatype}\NormalTok{)}
\end{Highlighting}
\end{Shaded}

\begin{verbatim}
(:V2 :V3 :V4)
(:V1 :V2 :V4)
\end{verbatim}

\begin{center}\rule{0.5\linewidth}{0.5pt}\end{center}

You can select column names based on all column metadata at once by
using \texttt{:all} metadata selector. Below we want to select column
names ending with \texttt{1} which have \texttt{long} datatype.

\begin{Shaded}
\begin{Highlighting}[]
\NormalTok{(api/column-names DS (}\KeywordTok{fn}\NormalTok{ [}\KeywordTok{meta}\NormalTok{]}
\NormalTok{                       (}\KeywordTok{and}\NormalTok{ (}\KeywordTok{=} \AttributeTok{:int64}\NormalTok{ (}\AttributeTok{:datatype} \KeywordTok{meta}\NormalTok{))}
\NormalTok{                            (clojure.string/ends-with? (}\AttributeTok{:name} \KeywordTok{meta}\NormalTok{) }\StringTok{"1"}\NormalTok{))) }\AttributeTok{:all}\NormalTok{)}
\end{Highlighting}
\end{Shaded}

\begin{verbatim}
(:V1)
\end{verbatim}

\paragraph{Select}\label{select}

\texttt{select-columns} creates dataset with columns selected by
\texttt{columns-selector} as described above. Function works on regular
and grouped dataset.

\begin{center}\rule{0.5\linewidth}{0.5pt}\end{center}

Select only float64 columns

\begin{Shaded}
\begin{Highlighting}[]
\NormalTok{(api/select-columns DS #(}\KeywordTok{=} \AttributeTok{:float64} \VariableTok{%}\NormalTok{) }\AttributeTok{:datatype}\NormalTok{)}
\end{Highlighting}
\end{Shaded}

\_unnamed {[}9 1{]}:

\begin{longtable}[]{@{}l@{}}
\toprule
:V3\tabularnewline
\midrule
\endhead
0.5000\tabularnewline
1.000\tabularnewline
1.500\tabularnewline
0.5000\tabularnewline
1.000\tabularnewline
1.500\tabularnewline
0.5000\tabularnewline
1.000\tabularnewline
1.500\tabularnewline
\bottomrule
\end{longtable}

\begin{center}\rule{0.5\linewidth}{0.5pt}\end{center}

Select all but \texttt{:V1} columns

\begin{Shaded}
\begin{Highlighting}[]
\NormalTok{(api/select-columns DS (}\KeywordTok{complement}\NormalTok{ #\{}\AttributeTok{:V1}\NormalTok{\}))}
\end{Highlighting}
\end{Shaded}

\_unnamed {[}9 3{]}:

\begin{longtable}[]{@{}lll@{}}
\toprule
:V2 & :V3 & :V4\tabularnewline
\midrule
\endhead
1 & 0.5000 & A\tabularnewline
2 & 1.000 & B\tabularnewline
3 & 1.500 & C\tabularnewline
4 & 0.5000 & A\tabularnewline
5 & 1.000 & B\tabularnewline
6 & 1.500 & C\tabularnewline
7 & 0.5000 & A\tabularnewline
8 & 1.000 & B\tabularnewline
9 & 1.500 & C\tabularnewline
\bottomrule
\end{longtable}

\begin{center}\rule{0.5\linewidth}{0.5pt}\end{center}

If we have grouped data set, column selection is applied to every group
separately.

\begin{Shaded}
\begin{Highlighting}[]
\NormalTok{(}\KeywordTok{->}\NormalTok{ DS}
\NormalTok{    (api/group-by }\AttributeTok{:V1}\NormalTok{)}
\NormalTok{    (api/select-columns [}\AttributeTok{:V2} \AttributeTok{:V3}\NormalTok{])}
\NormalTok{    (api/groups->map))}
\end{Highlighting}
\end{Shaded}

\{1 1 {[}5 2{]}:

\begin{longtable}[]{@{}ll@{}}
\toprule
:V2 & :V3\tabularnewline
\midrule
\endhead
1 & 0.5000\tabularnewline
3 & 1.500\tabularnewline
5 & 1.000\tabularnewline
7 & 0.5000\tabularnewline
9 & 1.500\tabularnewline
\bottomrule
\end{longtable}

, 2 2 {[}4 2{]}:

\begin{longtable}[]{@{}ll@{}}
\toprule
:V2 & :V3\tabularnewline
\midrule
\endhead
2 & 1.000\tabularnewline
4 & 0.5000\tabularnewline
6 & 1.500\tabularnewline
8 & 1.000\tabularnewline
\bottomrule
\end{longtable}

\}

\paragraph{Drop}\label{drop}

\texttt{drop-columns} creates dataset with removed columns.

\begin{center}\rule{0.5\linewidth}{0.5pt}\end{center}

Drop float64 columns

\begin{Shaded}
\begin{Highlighting}[]
\NormalTok{(api/drop-columns DS #(}\KeywordTok{=} \AttributeTok{:float64} \VariableTok{%}\NormalTok{) }\AttributeTok{:datatype}\NormalTok{)}
\end{Highlighting}
\end{Shaded}

\_unnamed {[}9 3{]}:

\begin{longtable}[]{@{}lll@{}}
\toprule
:V1 & :V2 & :V4\tabularnewline
\midrule
\endhead
1 & 1 & A\tabularnewline
2 & 2 & B\tabularnewline
1 & 3 & C\tabularnewline
2 & 4 & A\tabularnewline
1 & 5 & B\tabularnewline
2 & 6 & C\tabularnewline
1 & 7 & A\tabularnewline
2 & 8 & B\tabularnewline
1 & 9 & C\tabularnewline
\bottomrule
\end{longtable}

\begin{center}\rule{0.5\linewidth}{0.5pt}\end{center}

Drop all columns but \texttt{:V1} and \texttt{:V2}

\begin{Shaded}
\begin{Highlighting}[]
\NormalTok{(api/drop-columns DS (}\KeywordTok{complement}\NormalTok{ #\{}\AttributeTok{:V1} \AttributeTok{:V2}\NormalTok{\}))}
\end{Highlighting}
\end{Shaded}

\_unnamed {[}9 2{]}:

\begin{longtable}[]{@{}ll@{}}
\toprule
:V1 & :V2\tabularnewline
\midrule
\endhead
1 & 1\tabularnewline
2 & 2\tabularnewline
1 & 3\tabularnewline
2 & 4\tabularnewline
1 & 5\tabularnewline
2 & 6\tabularnewline
1 & 7\tabularnewline
2 & 8\tabularnewline
1 & 9\tabularnewline
\bottomrule
\end{longtable}

\begin{center}\rule{0.5\linewidth}{0.5pt}\end{center}

If we have grouped data set, column selection is applied to every group
separately. Selected columns are dropped.

\begin{Shaded}
\begin{Highlighting}[]
\NormalTok{(}\KeywordTok{->}\NormalTok{ DS}
\NormalTok{    (api/group-by }\AttributeTok{:V1}\NormalTok{)}
\NormalTok{    (api/drop-columns [}\AttributeTok{:V2} \AttributeTok{:V3}\NormalTok{])}
\NormalTok{    (api/groups->map))}
\end{Highlighting}
\end{Shaded}

\{1 1 {[}5 2{]}:

\begin{longtable}[]{@{}ll@{}}
\toprule
:V1 & :V4\tabularnewline
\midrule
\endhead
1 & A\tabularnewline
1 & C\tabularnewline
1 & B\tabularnewline
1 & A\tabularnewline
1 & C\tabularnewline
\bottomrule
\end{longtable}

, 2 2 {[}4 2{]}:

\begin{longtable}[]{@{}ll@{}}
\toprule
:V1 & :V4\tabularnewline
\midrule
\endhead
2 & B\tabularnewline
2 & A\tabularnewline
2 & C\tabularnewline
2 & B\tabularnewline
\bottomrule
\end{longtable}

\}

\paragraph{Rename}\label{rename}

If you want to rename colums use \texttt{rename-columns} and pass map
where keys are old names, values new ones.

\begin{Shaded}
\begin{Highlighting}[]
\NormalTok{(api/rename-columns DS \{}\AttributeTok{:V1} \StringTok{"v1"}
                        \AttributeTok{:V2} \StringTok{"v2"}
                        \AttributeTok{:V3}\NormalTok{ [}\DecValTok{1} \DecValTok{2} \DecValTok{3}\NormalTok{]}
                        \AttributeTok{:V4}\NormalTok{ (Object.)\})}
\end{Highlighting}
\end{Shaded}

\_unnamed {[}9 4{]}:

\begin{longtable}[]{@{}llll@{}}
\toprule
v1 & v2 & {[}1 2 3{]} &
\href{mailto:java.lang.Object@2b0acee5}{\nolinkurl{java.lang.Object@2b0acee5}}\tabularnewline
\midrule
\endhead
1 & 1 & 0.5000 & A\tabularnewline
2 & 2 & 1.000 & B\tabularnewline
1 & 3 & 1.500 & C\tabularnewline
2 & 4 & 0.5000 & A\tabularnewline
1 & 5 & 1.000 & B\tabularnewline
2 & 6 & 1.500 & C\tabularnewline
1 & 7 & 0.5000 & A\tabularnewline
2 & 8 & 1.000 & B\tabularnewline
1 & 9 & 1.500 & C\tabularnewline
\bottomrule
\end{longtable}

\begin{center}\rule{0.5\linewidth}{0.5pt}\end{center}

Function works on grouped dataset

\begin{Shaded}
\begin{Highlighting}[]
\NormalTok{(}\KeywordTok{->}\NormalTok{ DS}
\NormalTok{    (api/group-by }\AttributeTok{:V1}\NormalTok{)}
\NormalTok{    (api/rename-columns \{}\AttributeTok{:V1} \StringTok{"v1"}
                         \AttributeTok{:V2} \StringTok{"v2"}
                         \AttributeTok{:V3}\NormalTok{ [}\DecValTok{1} \DecValTok{2} \DecValTok{3}\NormalTok{]}
                         \AttributeTok{:V4}\NormalTok{ (Object.)\})}
\NormalTok{    (api/groups->map))}
\end{Highlighting}
\end{Shaded}

\{1 1 {[}5 4{]}:

\begin{longtable}[]{@{}llll@{}}
\toprule
v1 & v2 & {[}1 2 3{]} &
\href{mailto:java.lang.Object@60b39df5}{\nolinkurl{java.lang.Object@60b39df5}}\tabularnewline
\midrule
\endhead
1 & 1 & 0.5000 & A\tabularnewline
1 & 3 & 1.500 & C\tabularnewline
1 & 5 & 1.000 & B\tabularnewline
1 & 7 & 0.5000 & A\tabularnewline
1 & 9 & 1.500 & C\tabularnewline
\bottomrule
\end{longtable}

, 2 2 {[}4 4{]}:

\begin{longtable}[]{@{}llll@{}}
\toprule
v1 & v2 & {[}1 2 3{]} &
\href{mailto:java.lang.Object@60b39df5}{\nolinkurl{java.lang.Object@60b39df5}}\tabularnewline
\midrule
\endhead
2 & 2 & 1.000 & B\tabularnewline
2 & 4 & 0.5000 & A\tabularnewline
2 & 6 & 1.500 & C\tabularnewline
2 & 8 & 1.000 & B\tabularnewline
\bottomrule
\end{longtable}

\}

\paragraph{Add or update}\label{add-or-update}

To add (or update existing) column call \texttt{add-or-update-column}
function. Function accepts:

\begin{itemize}
\tightlist
\item
  \texttt{ds} - a dataset
\item
  \texttt{column-name} - if it's existing column name, column will be
  replaced
\item
  \texttt{column} - can be column (from other dataset), sequence, single
  value or function. Too big columns are always trimmed. Too small are
  cycled or extended with missing values (according to
  \texttt{size-strategy} argument)
\item
  \texttt{size-strategy} (optional) - when new column is shorter than
  dataset row count, following strategies are applied:

  \begin{itemize}
  \tightlist
  \item
    \texttt{:cycle} (default) - repeat data
  \item
    \texttt{:na} - append missing values
  \end{itemize}
\end{itemize}

Function works on grouped dataset.

\begin{center}\rule{0.5\linewidth}{0.5pt}\end{center}

Add single value as column

\begin{Shaded}
\begin{Highlighting}[]
\NormalTok{(api/add-or-update-column DS }\AttributeTok{:V5} \StringTok{"X"}\NormalTok{)}
\end{Highlighting}
\end{Shaded}

\_unnamed {[}9 5{]}:

\begin{longtable}[]{@{}lllll@{}}
\toprule
:V1 & :V2 & :V3 & :V4 & :V5\tabularnewline
\midrule
\endhead
1 & 1 & 0.5000 & A & X\tabularnewline
2 & 2 & 1.000 & B & X\tabularnewline
1 & 3 & 1.500 & C & X\tabularnewline
2 & 4 & 0.5000 & A & X\tabularnewline
1 & 5 & 1.000 & B & X\tabularnewline
2 & 6 & 1.500 & C & X\tabularnewline
1 & 7 & 0.5000 & A & X\tabularnewline
2 & 8 & 1.000 & B & X\tabularnewline
1 & 9 & 1.500 & C & X\tabularnewline
\bottomrule
\end{longtable}

\begin{center}\rule{0.5\linewidth}{0.5pt}\end{center}

Replace one column (column is trimmed)

\begin{Shaded}
\begin{Highlighting}[]
\NormalTok{(api/add-or-update-column DS }\AttributeTok{:V1}\NormalTok{ (}\KeywordTok{repeatedly} \KeywordTok{rand}\NormalTok{))}
\end{Highlighting}
\end{Shaded}

\_unnamed {[}9 4{]}:

\begin{longtable}[]{@{}llll@{}}
\toprule
:V1 & :V2 & :V3 & :V4\tabularnewline
\midrule
\endhead
0.3120 & 1 & 0.5000 & A\tabularnewline
0.4625 & 2 & 1.000 & B\tabularnewline
0.4346 & 3 & 1.500 & C\tabularnewline
0.05148 & 4 & 0.5000 & A\tabularnewline
0.2961 & 5 & 1.000 & B\tabularnewline
0.1465 & 6 & 1.500 & C\tabularnewline
0.6781 & 7 & 0.5000 & A\tabularnewline
0.008779 & 8 & 1.000 & B\tabularnewline
0.4155 & 9 & 1.500 & C\tabularnewline
\bottomrule
\end{longtable}

\begin{center}\rule{0.5\linewidth}{0.5pt}\end{center}

Copy column

\begin{Shaded}
\begin{Highlighting}[]
\NormalTok{(api/add-or-update-column DS }\AttributeTok{:V5}\NormalTok{ (DS }\AttributeTok{:V1}\NormalTok{))}
\end{Highlighting}
\end{Shaded}

\_unnamed {[}9 5{]}:

\begin{longtable}[]{@{}lllll@{}}
\toprule
:V1 & :V2 & :V3 & :V4 & :V5\tabularnewline
\midrule
\endhead
1 & 1 & 0.5000 & A & 1\tabularnewline
2 & 2 & 1.000 & B & 2\tabularnewline
1 & 3 & 1.500 & C & 1\tabularnewline
2 & 4 & 0.5000 & A & 2\tabularnewline
1 & 5 & 1.000 & B & 1\tabularnewline
2 & 6 & 1.500 & C & 2\tabularnewline
1 & 7 & 0.5000 & A & 1\tabularnewline
2 & 8 & 1.000 & B & 2\tabularnewline
1 & 9 & 1.500 & C & 1\tabularnewline
\bottomrule
\end{longtable}

\begin{center}\rule{0.5\linewidth}{0.5pt}\end{center}

When function is used, argument is whole dataset and the result should
be column, sequence or single value

\begin{Shaded}
\begin{Highlighting}[]
\NormalTok{(api/add-or-update-column DS }\AttributeTok{:row-count}\NormalTok{ api/row-count) }
\end{Highlighting}
\end{Shaded}

\_unnamed {[}9 5{]}:

\begin{longtable}[]{@{}lllll@{}}
\toprule
:V1 & :V2 & :V3 & :V4 & :row-count\tabularnewline
\midrule
\endhead
1 & 1 & 0.5000 & A & 9\tabularnewline
2 & 2 & 1.000 & B & 9\tabularnewline
1 & 3 & 1.500 & C & 9\tabularnewline
2 & 4 & 0.5000 & A & 9\tabularnewline
1 & 5 & 1.000 & B & 9\tabularnewline
2 & 6 & 1.500 & C & 9\tabularnewline
1 & 7 & 0.5000 & A & 9\tabularnewline
2 & 8 & 1.000 & B & 9\tabularnewline
1 & 9 & 1.500 & C & 9\tabularnewline
\bottomrule
\end{longtable}

\begin{center}\rule{0.5\linewidth}{0.5pt}\end{center}

Above example run on grouped dataset, applies function on each group
separately.

\begin{Shaded}
\begin{Highlighting}[]
\NormalTok{(}\KeywordTok{->}\NormalTok{ DS}
\NormalTok{    (api/group-by }\AttributeTok{:V1}\NormalTok{)}
\NormalTok{    (api/add-or-update-column }\AttributeTok{:row-count}\NormalTok{ api/row-count)}
\NormalTok{    (api/ungroup))}
\end{Highlighting}
\end{Shaded}

\_unnamed {[}9 5{]}:

\begin{longtable}[]{@{}lllll@{}}
\toprule
:V1 & :V2 & :V3 & :V4 & :row-count\tabularnewline
\midrule
\endhead
1 & 1 & 0.5000 & A & 5\tabularnewline
1 & 3 & 1.500 & C & 5\tabularnewline
1 & 5 & 1.000 & B & 5\tabularnewline
1 & 7 & 0.5000 & A & 5\tabularnewline
1 & 9 & 1.500 & C & 5\tabularnewline
2 & 2 & 1.000 & B & 4\tabularnewline
2 & 4 & 0.5000 & A & 4\tabularnewline
2 & 6 & 1.500 & C & 4\tabularnewline
2 & 8 & 1.000 & B & 4\tabularnewline
\bottomrule
\end{longtable}

\begin{center}\rule{0.5\linewidth}{0.5pt}\end{center}

When column which is added is longer than row count in dataset, column
is trimmed. When column is shorter, it's cycled or missing values are
appended.

\begin{Shaded}
\begin{Highlighting}[]
\NormalTok{(api/add-or-update-column DS }\AttributeTok{:V5}\NormalTok{ [}\AttributeTok{:r} \AttributeTok{:b}\NormalTok{])}
\end{Highlighting}
\end{Shaded}

\_unnamed {[}9 5{]}:

\begin{longtable}[]{@{}lllll@{}}
\toprule
:V1 & :V2 & :V3 & :V4 & :V5\tabularnewline
\midrule
\endhead
1 & 1 & 0.5000 & A & :r\tabularnewline
2 & 2 & 1.000 & B & :b\tabularnewline
1 & 3 & 1.500 & C & :r\tabularnewline
2 & 4 & 0.5000 & A & :b\tabularnewline
1 & 5 & 1.000 & B & :r\tabularnewline
2 & 6 & 1.500 & C & :b\tabularnewline
1 & 7 & 0.5000 & A & :r\tabularnewline
2 & 8 & 1.000 & B & :b\tabularnewline
1 & 9 & 1.500 & C & :r\tabularnewline
\bottomrule
\end{longtable}

\begin{Shaded}
\begin{Highlighting}[]
\NormalTok{(api/add-or-update-column DS }\AttributeTok{:V5}\NormalTok{ [}\AttributeTok{:r} \AttributeTok{:b}\NormalTok{] }\AttributeTok{:na}\NormalTok{)}
\end{Highlighting}
\end{Shaded}

\_unnamed {[}9 5{]}:

\begin{longtable}[]{@{}lllll@{}}
\toprule
:V1 & :V2 & :V3 & :V4 & :V5\tabularnewline
\midrule
\endhead
1 & 1 & 0.5000 & A & :r\tabularnewline
2 & 2 & 1.000 & B & :b\tabularnewline
1 & 3 & 1.500 & C &\tabularnewline
2 & 4 & 0.5000 & A &\tabularnewline
1 & 5 & 1.000 & B &\tabularnewline
2 & 6 & 1.500 & C &\tabularnewline
1 & 7 & 0.5000 & A &\tabularnewline
2 & 8 & 1.000 & B &\tabularnewline
1 & 9 & 1.500 & C &\tabularnewline
\bottomrule
\end{longtable}

\begin{center}\rule{0.5\linewidth}{0.5pt}\end{center}

Tha same applies for grouped dataset

\begin{Shaded}
\begin{Highlighting}[]
\NormalTok{(}\KeywordTok{->}\NormalTok{ DS}
\NormalTok{    (api/group-by }\AttributeTok{:V3}\NormalTok{)}
\NormalTok{    (api/add-or-update-column }\AttributeTok{:V5}\NormalTok{ [}\AttributeTok{:r} \AttributeTok{:b}\NormalTok{] }\AttributeTok{:na}\NormalTok{)}
\NormalTok{    (api/ungroup))}
\end{Highlighting}
\end{Shaded}

\_unnamed {[}9 5{]}:

\begin{longtable}[]{@{}lllll@{}}
\toprule
:V1 & :V2 & :V3 & :V4 & :V5\tabularnewline
\midrule
\endhead
2 & 2 & 1.000 & B & :r\tabularnewline
1 & 5 & 1.000 & B & :b\tabularnewline
2 & 8 & 1.000 & B &\tabularnewline
1 & 1 & 0.5000 & A & :r\tabularnewline
2 & 4 & 0.5000 & A & :b\tabularnewline
1 & 7 & 0.5000 & A &\tabularnewline
1 & 3 & 1.500 & C & :r\tabularnewline
2 & 6 & 1.500 & C & :b\tabularnewline
1 & 9 & 1.500 & C &\tabularnewline
\bottomrule
\end{longtable}

\begin{center}\rule{0.5\linewidth}{0.5pt}\end{center}

Let's use other column to fill groups

\begin{Shaded}
\begin{Highlighting}[]
\NormalTok{(}\KeywordTok{->}\NormalTok{ DS}
\NormalTok{    (api/group-by }\AttributeTok{:V3}\NormalTok{)}
\NormalTok{    (api/add-or-update-column }\AttributeTok{:V5}\NormalTok{ (DS }\AttributeTok{:V2}\NormalTok{))}
\NormalTok{    (api/ungroup))}
\end{Highlighting}
\end{Shaded}

\_unnamed {[}9 5{]}:

\begin{longtable}[]{@{}lllll@{}}
\toprule
:V1 & :V2 & :V3 & :V4 & :V5\tabularnewline
\midrule
\endhead
2 & 2 & 1.000 & B & 1\tabularnewline
1 & 5 & 1.000 & B & 2\tabularnewline
2 & 8 & 1.000 & B & 3\tabularnewline
1 & 1 & 0.5000 & A & 1\tabularnewline
2 & 4 & 0.5000 & A & 2\tabularnewline
1 & 7 & 0.5000 & A & 3\tabularnewline
1 & 3 & 1.500 & C & 1\tabularnewline
2 & 6 & 1.500 & C & 2\tabularnewline
1 & 9 & 1.500 & C & 3\tabularnewline
\bottomrule
\end{longtable}

\begin{center}\rule{0.5\linewidth}{0.5pt}\end{center}

In case you want to add or update several columns you can call
\texttt{add-or-update-columns} and provide map where keys are column
names, vals are columns.

\begin{Shaded}
\begin{Highlighting}[]
\NormalTok{(api/add-or-update-columns DS \{}\AttributeTok{:V1}\NormalTok{ #(}\KeywordTok{map} \KeywordTok{inc}\NormalTok{ (}\VariableTok \AttributeTok{:V4}\NormalTok{))}
                               \AttributeTok{:V6} \DecValTok{11}\NormalTok{\})}
\end{Highlighting}
\end{Shaded}

\_unnamed {[}9 6{]}:

\begin{longtable}[]{@{}llllll@{}}
\toprule
:V1 & :V2 & :V3 & :V4 & :V5 & :V6\tabularnewline
\midrule
\endhead
2 & 1 & 0.5000 & A & :A & 11\tabularnewline
3 & 2 & 1.000 & B & :B & 11\tabularnewline
2 & 3 & 1.500 & C & :C & 11\tabularnewline
3 & 4 & 0.5000 & A & :A & 11\tabularnewline
2 & 5 & 1.000 & B & :B & 11\tabularnewline
3 & 6 & 1.500 & C & :C & 11\tabularnewline
2 & 7 & 0.5000 & A & :A & 11\tabularnewline
3 & 8 & 1.000 & B & :B & 11\tabularnewline
2 & 9 & 1.500 & C & :C & 11\tabularnewline
\bottomrule
\end{longtable}

\paragraph{Map}\label{map}

The other way of creating or updating column is to map columns as
regular \texttt{map} function. The arity of mapping function should be
the same as number of selected columns.

Arguments:

\begin{itemize}
\tightlist
\item
  \texttt{ds} - dataset
\item
  \texttt{column-name} - target column name
\item
  \texttt{map-fn} - mapping function
\item
  \texttt{columns-selector} - columns selected
\item
  \texttt{meta-field} (optional) - column selector option
\end{itemize}

\begin{center}\rule{0.5\linewidth}{0.5pt}\end{center}

Let's add numerical columns together

\begin{Shaded}
\begin{Highlighting}[]
\NormalTok{(api/map-columns DS }\AttributeTok{:sum-of-numbers}\NormalTok{ (}\KeywordTok{fn}\NormalTok{ [& rows]}
\NormalTok{                                      (}\KeywordTok{reduce} \KeywordTok{+}\NormalTok{ rows)) #\{}\AttributeTok{:int64} \AttributeTok{:float64}\NormalTok{\} }\AttributeTok{:datatype}\NormalTok{)}
\end{Highlighting}
\end{Shaded}

\_unnamed {[}9 5{]}:

\begin{longtable}[]{@{}lllll@{}}
\toprule
:V1 & :V2 & :V3 & :V4 & :sum-of-numbers\tabularnewline
\midrule
\endhead
1 & 1 & 0.5000 & A & 2.500\tabularnewline
2 & 2 & 1.000 & B & 5.000\tabularnewline
1 & 3 & 1.500 & C & 5.500\tabularnewline
2 & 4 & 0.5000 & A & 6.500\tabularnewline
1 & 5 & 1.000 & B & 7.000\tabularnewline
2 & 6 & 1.500 & C & 9.500\tabularnewline
1 & 7 & 0.5000 & A & 8.500\tabularnewline
2 & 8 & 1.000 & B & 11.00\tabularnewline
1 & 9 & 1.500 & C & 11.50\tabularnewline
\bottomrule
\end{longtable}

The same works on grouped dataset

\begin{Shaded}
\begin{Highlighting}[]
\NormalTok{(}\KeywordTok{->}\NormalTok{ DS}
\NormalTok{    (api/group-by }\AttributeTok{:V4}\NormalTok{)}
\NormalTok{    (api/map-columns }\AttributeTok{:sum-of-numbers}\NormalTok{ (}\KeywordTok{fn}\NormalTok{ [& rows]}
\NormalTok{                                       (}\KeywordTok{reduce} \KeywordTok{+}\NormalTok{ rows)) #\{}\AttributeTok{:int64} \AttributeTok{:float64}\NormalTok{\} }\AttributeTok{:datatype}\NormalTok{)}
\NormalTok{    (api/ungroup))}
\end{Highlighting}
\end{Shaded}

\_unnamed {[}9 5{]}:

\begin{longtable}[]{@{}lllll@{}}
\toprule
:V1 & :V2 & :V3 & :V4 & :sum-of-numbers\tabularnewline
\midrule
\endhead
1 & 1 & 0.5000 & A & 2.500\tabularnewline
2 & 4 & 0.5000 & A & 6.500\tabularnewline
1 & 7 & 0.5000 & A & 8.500\tabularnewline
2 & 2 & 1.000 & B & 5.000\tabularnewline
1 & 5 & 1.000 & B & 7.000\tabularnewline
2 & 8 & 1.000 & B & 11.00\tabularnewline
1 & 3 & 1.500 & C & 5.500\tabularnewline
2 & 6 & 1.500 & C & 9.500\tabularnewline
1 & 9 & 1.500 & C & 11.50\tabularnewline
\bottomrule
\end{longtable}

\paragraph{Reorder}\label{reorder}

To reorder columns use columns selectors to choose what columns go
first. The unseleted columns are appended to the end.

\begin{Shaded}
\begin{Highlighting}[]
\NormalTok{(api/reorder-columns DS }\AttributeTok{:V4}\NormalTok{ [}\AttributeTok{:V3} \AttributeTok{:V2}\NormalTok{] }\AttributeTok{:V1}\NormalTok{)}
\end{Highlighting}
\end{Shaded}

\_unnamed {[}9 4{]}:

\begin{longtable}[]{@{}llll@{}}
\toprule
:V4 & :V2 & :V3 & :V1\tabularnewline
\midrule
\endhead
A & 1 & 0.5000 & 1\tabularnewline
B & 2 & 1.000 & 2\tabularnewline
C & 3 & 1.500 & 1\tabularnewline
A & 4 & 0.5000 & 2\tabularnewline
B & 5 & 1.000 & 1\tabularnewline
C & 6 & 1.500 & 2\tabularnewline
A & 7 & 0.5000 & 1\tabularnewline
B & 8 & 1.000 & 2\tabularnewline
C & 9 & 1.500 & 1\tabularnewline
\bottomrule
\end{longtable}

\begin{center}\rule{0.5\linewidth}{0.5pt}\end{center}

This function doesn't let you select meta field, so you have to call
\texttt{column-names} in such case. Below we want to add integer columns
at the end.

\begin{Shaded}
\begin{Highlighting}[]
\NormalTok{(api/reorder-columns DS (api/column-names DS (}\KeywordTok{complement}\NormalTok{ #\{}\AttributeTok{:int64}\NormalTok{\}) }\AttributeTok{:datatype}\NormalTok{))}
\end{Highlighting}
\end{Shaded}

\_unnamed {[}9 4{]}:

\begin{longtable}[]{@{}llll@{}}
\toprule
:V3 & :V4 & :V1 & :V2\tabularnewline
\midrule
\endhead
0.5000 & A & 1 & 1\tabularnewline
1.000 & B & 2 & 2\tabularnewline
1.500 & C & 1 & 3\tabularnewline
0.5000 & A & 2 & 4\tabularnewline
1.000 & B & 1 & 5\tabularnewline
1.500 & C & 2 & 6\tabularnewline
0.5000 & A & 1 & 7\tabularnewline
1.000 & B & 2 & 8\tabularnewline
1.500 & C & 1 & 9\tabularnewline
\bottomrule
\end{longtable}

\paragraph{Type conversion}\label{type-conversion}

To convert column into given datatype can be done using
\texttt{convert-column-type} function. Not all the types can be
converted automatically also some types require slow parsing (every
conversion from string). In case where conversion is not possible you
can pass conversion function.

Arguments:

\begin{itemize}
\tightlist
\item
  \texttt{ds} - dataset
\item
  Two options:

  \begin{itemize}
  \tightlist
  \item
    \texttt{coltype-map} in case when you want to convert several
    columns, keys are column names, vals are new types
  \item
    \texttt{colname} and \texttt{new-type} - column name and new
    datatype
  \end{itemize}
\end{itemize}

\texttt{new-type} can be:

\begin{itemize}
\tightlist
\item
  a type like \texttt{:int64} or \texttt{:string}
\item
  or pair of datetime and conversion function
\end{itemize}

After conversion additional infomation is given on problematic values

\begin{center}\rule{0.5\linewidth}{0.5pt}\end{center}

Basic conversion

\begin{Shaded}
\begin{Highlighting}[]
\NormalTok{(}\KeywordTok{->}\NormalTok{ DS}
\NormalTok{    (api/convert-column-type }\AttributeTok{:V1} \AttributeTok{:float64}\NormalTok{)}
\NormalTok{    (api/info }\AttributeTok{:columns}\NormalTok{))}
\end{Highlighting}
\end{Shaded}

\_unnamed :column info {[}4 5{]}:

\begin{longtable}[]{@{}lllll@{}}
\toprule
:name & :size & :datatype & :unparsed-indexes &
:unparsed-data\tabularnewline
\midrule
\endhead
:V1 & 9 & :float64 & \{\} & {[}{]}\tabularnewline
:V2 & 9 & :int64 & &\tabularnewline
:V3 & 9 & :float64 & &\tabularnewline
:V4 & 9 & :object & &\tabularnewline
\bottomrule
\end{longtable}

\begin{center}\rule{0.5\linewidth}{0.5pt}\end{center}

Using custom converter. Let's treat \texttt{:V4} as haxadecimal values.
See that this way we can map column to any value.

\begin{Shaded}
\begin{Highlighting}[]
\NormalTok{(}\KeywordTok{->}\NormalTok{ DS}
\NormalTok{    (api/convert-column-type }\AttributeTok{:V4}\NormalTok{ [}\AttributeTok{:int16}\NormalTok{ #(Integer/parseInt (}\KeywordTok{str} \VariableTok{%}\NormalTok{) }\DecValTok{16}\NormalTok{)]))}
\end{Highlighting}
\end{Shaded}

\_unnamed {[}9 4{]}:

\begin{longtable}[]{@{}llll@{}}
\toprule
:V1 & :V2 & :V3 & :V4\tabularnewline
\midrule
\endhead
1 & 1 & 0.5000 & 10\tabularnewline
2 & 2 & 1.000 & 11\tabularnewline
1 & 3 & 1.500 & 12\tabularnewline
2 & 4 & 0.5000 & 10\tabularnewline
1 & 5 & 1.000 & 11\tabularnewline
2 & 6 & 1.500 & 12\tabularnewline
1 & 7 & 0.5000 & 10\tabularnewline
2 & 8 & 1.000 & 11\tabularnewline
1 & 9 & 1.500 & 12\tabularnewline
\bottomrule
\end{longtable}

\begin{center}\rule{0.5\linewidth}{0.5pt}\end{center}

You can process several columns at once

\begin{Shaded}
\begin{Highlighting}[]
\NormalTok{(}\KeywordTok{->}\NormalTok{ DS}
\NormalTok{    (api/convert-column-type \{}\AttributeTok{:V1} \AttributeTok{:float64}
                              \AttributeTok{:V2} \AttributeTok{:object}
                              \AttributeTok{:V3}\NormalTok{ [}\AttributeTok{:boolean}\NormalTok{ #(}\KeywordTok{<} \VariableTok{%} \FloatTok{1.0}\NormalTok{)]}
                              \AttributeTok{:V4} \AttributeTok{:string}\NormalTok{\})}
\NormalTok{    (api/info }\AttributeTok{:columns}\NormalTok{))}
\end{Highlighting}
\end{Shaded}

\_unnamed :column info {[}4 6{]}:

\begin{longtable}[]{@{}llllll@{}}
\toprule
:name & :size & :datatype & :unparsed-indexes & :unparsed-data &
:categorical?\tabularnewline
\midrule
\endhead
:V1 & 9 & :float64 & \{\} & {[}{]} &\tabularnewline
:V2 & 9 & :object & \{\} & {[}{]} &\tabularnewline
:V3 & 9 & :boolean & \{\} & {[}{]} &\tabularnewline
:V4 & 9 & :string & \{\} & {[}{]} & true\tabularnewline
\bottomrule
\end{longtable}

\begin{center}\rule{0.5\linewidth}{0.5pt}\end{center}

Function works on the grouped dataset

\begin{Shaded}
\begin{Highlighting}[]
\NormalTok{(}\KeywordTok{->}\NormalTok{ DS}
\NormalTok{    (api/group-by }\AttributeTok{:V1}\NormalTok{)}
\NormalTok{    (api/convert-column-type }\AttributeTok{:V1} \AttributeTok{:float32}\NormalTok{)}
\NormalTok{    (api/ungroup)}
\NormalTok{    (api/info }\AttributeTok{:columns}\NormalTok{))}
\end{Highlighting}
\end{Shaded}

\_unnamed :column info {[}4 5{]}:

\begin{longtable}[]{@{}lllll@{}}
\toprule
:name & :size & :datatype & :unparsed-indexes &
:unparsed-data\tabularnewline
\midrule
\endhead
:V1 & 9 & :float32 & \{\} & {[}{]}\tabularnewline
:V2 & 9 & :int64 & &\tabularnewline
:V3 & 9 & :float64 & &\tabularnewline
:V4 & 9 & :object & &\tabularnewline
\bottomrule
\end{longtable}

\subsubsection{Rows}\label{rows}

\paragraph{Select}\label{select-1}

\paragraph{Drop}\label{drop-1}

\paragraph{Other}\label{other}

\subsubsection{Aggregate}\label{aggregate}

\subsubsection{Order}\label{order}

\subsubsection{Unique}\label{unique}

\paragraph{Strategies}\label{strategies}

\subsubsection{Missing}\label{missing}

\subsubsection{Select}\label{select-2}

\subsubsection{Drop}\label{drop-2}

\subsubsection{Replace}\label{replace}

\subsubsection{Join/Separate Columns}\label{joinseparate-columns}

\paragraph{Join}\label{join}

\paragraph{Separate}\label{separate}

\subsubsection{Fold/Unroll Rows}\label{foldunroll-rows}

\paragraph{Fold}\label{fold}

\paragraph{Unroll}\label{unroll}

\subsubsection{Reshape}\label{reshape}

\paragraph{Longer}\label{longer}

\paragraph{Wider}\label{wider}

\subsubsection{Join/Concat}\label{joinconcat}

\paragraph{Left}\label{left}

\paragraph{Right}\label{right}

\paragraph{Inner}\label{inner}

\paragraph{Hash}\label{hash}

\paragraph{Concat}\label{concat}

\subsubsection{Functions}\label{functions}

\end{document}
